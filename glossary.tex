\newglossaryentry{domain-knowledge}{%
  name={domain knowledge},%
  description={valid knowledge used to refer to an area of human endeavour, an autonomous computer activity, or other specialized discipline}}

\newacronym[description={(Environmental Noise Directive) \emph{DIRECTIVE 2002/49/EC OF THE EUROPEAN PARLIAMENT AND OF THE COUNCIL of 25 June 2002 relating to the assessment and management of environmental noise}. Policy directive within the EU setting out priorities and requirements of member nations for ensuring health and environmental protection as it relates to noise. Incorporates requirements for agglomerations to produce noise maps and identify and preserve quiet areas.}]{end}{END}{Environmental Noise Directive}

\newglossaryentry{isopl}{name={ISOPleasant},description={The value along the primary pleasantness dimension of the soundscape circumplex, calculated via a trigonometric projection of the other \gls{paq}s, as defined in \citet{ISO12913_3_2019IOS}}}

\newglossaryentry{isoev}{name={ISOEventful},description={The value along the primary eventfulness dimension of the soundscape circumplex, calculated via a trigonometric projection of the other \gls{paq}s, as defined in \citet{ISO12913_3_2019IOS}}}

% Psychoacoustic features
%TODO: Write descriptions of psychoacoustic features
\newglossaryentry{laeq}{name={$L_{Aeq}$},description={}}
\newglossaryentry{n5}{name={$N_{5}$},description={Psychoacoustic Loudness}}
\newglossaryentry{s}{name={$S$},description={Psychoacoustic Sharpness}}
\newglossaryentry{r}{name={$R$},description={Psychoacoustic Roughness}}
\newglossaryentry{iu}{name={$I$},description={Impulsiveness}}
\newglossaryentry{fs}{name={$FS$},description={Fluctuation Strength}}
\newglossaryentry{tu}{name={$T$},description={Tonality}}
\newglossaryentry{pa}{name={$PA$},description={Zwicker Psychoacoustic Annoyance}}
\newglossaryentry{la10la90}{name={$L_{A10}-L_{A90}$},description={}}
\newglossaryentry{lcla}{name={$L_{Ceq}-L_{Aeq}$},description={}}
\newglossaryentry{ra}{name={$RA$},description={Relative Approach}}

\newglossaryentry{environmental-unit}{
  name={environmental unit},
  description={An area within a public space in which environmental factors are consistent and which is typically perceived to constitute a single distinct area.}
}


\newacronym{ssid}{SSID}{Soundscape Indices}
\newacronym{covid19}{COVID-19}{Coronavirus disease of 2019}
\newacronym{slm}{SLM}{Sound Level Meter}
\newacronym{amb}{AMB}{Ambisonic recording}
\newacronym{bin}{BIN}{Binaural}
\newacronym{pic}{PIC}{Site pictures}
\newacronym{vid}{VID}{360\degree Video}
\newacronym{env}{ENV}{Environmental factors}
\newacronym{que}{QUE}{Questionnaires}
\newacronym{ssqp}{SSQP}{Swedish Soundscape Quality Protocol}
\newacronym{paq}{PAQ}{Perceived Affective Quality}
\newacronym{who5}{WHO-5}{WHO Well-being Index}
\newacronym{vr}{VR}{Virtual Reality}
\newacronym{aic}{AIC}{Akaike Information Criterion}
\newacronym{vif}{VIF}{Variance Inflation Factor}
\newacronym{mae}{MAE}{Mean Absolute Error}