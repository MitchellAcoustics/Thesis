\chapter{$1/f$ Analysis of complex sound environments}
\label{app:overf}

\draft{Massive edits to make}

\section{Abstract}

It has been proposed that the temporal structure of acoustic environments exhibit a $1/f^\alpha$ behaviour, like many other time series in nature. To investigate the relative randomness within urban soundscapes, the slope coefficient ($\alpha$) of the temporal structure of a series of psychoacoustic metrics is calculated for recordings made in ?15 locations across Europe. The first stage of this analysis involves identifying at which frequency regimes this $\alpha_{f}$ metric exhibits different behaviours for the various psychoacoustic metrics. Finally, the relationships between this temporal structure and the assessed soundscape pleasantness and eventfulness are examined and the most significant features are identified.

\section{Introduction}

\textbf{General justification for soundscape (no more than 3 sentences?).}
The impact of noise pollution on general health and well-being has been recognised for decades and various strategies have been developed to incorporate contributing effects on perception when considering the impact of the noise environment and have developed and improved tools for assessing the perceived quality of a sound environment, known as the soundscape.

The soundscape approach to assessing and addressing urban noise pollution aims to take a more holistic view of the sound environment than more traditional methods. This is typically done through in situ questionnaires, soundwalks, and interviews, as laid out recently in Section 5 of \citet{ISO12913Part2}.

\subsection{Perceptual Attributes}
The model of soundscape characterization described by Axelsson et al. \cite{Axelsson2010principal} and recently standardised by ISO 12913-2 \citep{ISO12913Part2} comprises 8 perceptual attributes. These perceptual attributes form a 2-dimensional circumplex axis in which Pleasant/Unpleasant is orthogonal to Eventful/Uneventful, which is subsequently 45\degree{} offset from the orthogonal axis containing Exciting/Monotonous
and Chaotic/Calm. Any measured soundscape can thus be plotted on the circumplex axis according to its rated perceptual attributes.

\subsection{Importance of temporal structure on the sound environment}

Previous work on identifying the key acoustic and psychoacoustic parameters has so far yielded conclusive results only in the realm of soundscape identification \cite{Rychtarikova2013Soundscape} and has indicated that traditional acoustic parameters and analysis are insufficient metrics for subjective assessment of a soundscape's overall pleasantness \cite{Aletta2014Towards}. However, progress has been made in determining the relationships between (psycho)acoustic parameters and more targeted perceptual attributes such as vibrancy and eventfulness \cite{Aletta2018Towards, Jeon2011Non}.


Over the years, attempts to improve or reform traditional methods of noise assessment have produced a multitude of available noise indicators. A vital subset of these have provided information regarding the time series of the acoustic environment, typically serving one or both of two primary purposes: (1) to generalise sound levels or the response to sound levels over an extended period of time, and (2) to indicate the amount of variability of the sound level over the time period.

Those which fall into only the first category includes 24 hour metrics such as Community Noise Equivalent Level (CNEL) and the Day-Night  or Day-Evening-Night Levels (L\textsubscript{dn}
and L\textsubscript{den}, respectively); also included are levels meant for work-safety assessments, such as Sound Exposure Level (SEL).
\draft{Note: Psychoacoustic metrics were developed for industrial assessments of single source, product noise. This limits their applicability to multi-source complex sounde environments. }
\draft{Discuss current methods of assessing urban noise, i.e. Leq, Ldn, SEL, Ln, and how they handle temporal structure. Discussion of statistical methods e.g. Venaklasen method for CALGreen?}
\draft{Discussion of perceptual attributes and their collection and uses. Relate perceptual attributes to acoustical metrics above. Any discussion of temporality in perceptual attributes?}


\subsubsection{Introduce and review items of previous research in the area (obligatory).}

\draft{Music and Self-organised criticality}
\cite{Jeon2011Non} investigated the idea of an overall temporal structure in music,

Key points:
\begin{itemize}
  \item Found a 1/f structure in the spectral density of fluctuations of the audio power of many musical selections and English speech.
  \item 1/f structure of audio power holds down to $5 * 10^{-4}$ Hz (2000s period).
  \item 1/f structure of frequency fluctuations of music holds down to the inverse to the length of the sample.
  \item 1/f structure of frequency fluctuations of English speech show different characteristics, with a break of around 0.1s, the length of a single syllable.
        \begin{quote}
          The spectral density $S_v(f)$ of a quantity $V(t)$ fluctuating with time $t$ is a \textbf{measure of the mean squared variation} $<V^2>$ in a unit bandwidth centered on the frequency $f$.
        \end{quote}


\end{itemize}


\citet{deCoensel20031f} were the first to investigate the presence of this $1/f$ structure in environmental sound sources. Using a selection of 6 recordings from quiet rural areas in Flanders, Belgium and 12 from urban spaces in Ghent, Belgium, the authors examine the log-log spectra of A-weighted sound pressure, psychoacoustic loudness, and pitch fluctuations (i.e. zero-crossing rate) of both rural and urban soundscapes.




\cite{Botteldooren2006temporal} presented a temporal indicator based on this \(1/f\) structure. However, ...

\citet{Yang2015Presence} investigated the presence of a $1/f^{\alpha}$ behaviour in 102 recordings made in the countryside, natural parks, and urban areas in England between 1994 and 2010. They extended the previous investigations beyond simply considering the $\alpha$ value, to also consider whether different frequency regimes (the authors use the term 'range' instead) demonstrated different behaviours and to include the quadratic deviation from the best-fitted straight line. By visual inspection of the $1/f$ plots of the various psychoacoustic metrics, the authors noted two breaks in the frequency regimes, at $10^{-1}$ Hz and at $10^0$ Hz

Through these three primary studies various approaches to understanding this behaviour in soundscapes have been advanced. However, as de Coensel, Botteldooren, and de Muer explicitly stated, none of them have attempted to "link our observations to perception of the soundscape by the human observer". This is where this study picks up.


\subsection{Justification}
Characterising the dynamic nature of the sound environment, and the influence this will have on perception, is key to extending our assessments beyond what can be assessed in person by researchers.

To bring soundscape back to its roots as a consideration of sound environments as akin to a musical composition, when considering and discussing a piece of music, the three primary components are Melody, Harmony, and Rhythm. In this way, we could consider melody as the foreground, sound mark, or keynote sounds; harmony as the background or ambient sound; and rhythm as the temporal structure of the sound environment, both of the foreground and background sounds. Although the first two domains are fairly well researched in the soundscape literature, there is a lack of understanding of the influence of temporal structure on perception in soundscape.

Rhythm and structure in music can be distinguished into several levels or domains: the short-term is the rhythm of individual notes; the medium-term is the rhythm and structure of musical phrases (which could include the repetition of short-term rhythms); the long-term is the overall structure of a piece and organisation of movements. In the same way, the temporal structure of a sound environment can be distinguished into several domains: short-term (10s?) being the time structure of individual sound events (e.g. repeated beeps or knocking); the medium-term ...

In order to move towards an understanding of the temporal structure of complex, real-world soundscapes and towards the eventual development of a metric to describe this temporal structure, an analysis of the $1/f^\alpha$ structures of urban soundscapes has been conducted. By drawing from the large Soundscape Indices (SSID) database \citep{Mitchell2020Soundscape}, which includes 1,600 socio-acoustic surveys and accompanying 30s binaural recordings characteristics of the $1/f^\alpha$ structure in the time series of calculated psychoacoustic metrics are identified. Correlations between this structure and the soundscape perception as assessed by the in-situ surveys are then calculated to identify the influence the temporal structure may have on perception in real-world urban soundscapes.

\subsection{Frequency ranges, intervals, and regimes}

zBZoth De Coensel and Yang identified frequency ranges or intervals over which the behaviour of the spectrum differs. \citet{deCoensel20031f} divided the frequency interval into two parts: $I_1 = [0.002 Hz, 0.2 Hz]$ and $I_2 = [0.2 Hz, 5 Hz]$. $I_1$ which ranges from 200ms to 5s is therefore taken to represents the behaviour of the source itself, while $I_2$, which ranges from 5s to 10min is influenced by behaviours \emph{between} sources.

If this hypothesis is correct, wherein the high frequency interval is \emph{within} sources and the low frequency interval is \emph{between} sources, we would then expect to see similar interval separation across several different metrics.


\section{Methods}
\subsection{Survey collection}

\paragraph{SSID Protocol} This study makes use of the database collected as part of the (SSID) project. This SSID database consists of soundscape assessments and corresponding recordings carried out in 17 locations across the United Kingdom, the Netherlands, Spain, and Italy. The \citet{ISO12913Part1} series was consulted for reporting on soundscape data. \citet{ISO12913Part2} and \citet{ISO12913Part3} were used as the basis of the data collection protocol and data analysis used. For a full description of the SSID Protocol and of the publicly available database, please see \citet{Mitchell2020Soundscape} and \citet{Mitchell2021Database}.

\subparagraph{Binaural Recorder}

\subparagraph{Sessions}

\paragraph{Sound sources and PAQ questions}

\section{Analysis}
The temporal variation of several psychoacoustic parameters has been analyzed for each of the included recordings. Based on previous soundscape literature \citep{yang}

\subsection{Recording Processing}
\paragraph{Psychoacoustic calc}

\subparagraph{Channel max}

\subsection{$1/f^\alpha$ Analysis}
\paragraph{Power Spectrum analysis} Welch's method for noisy signals. nperseg determination.
\paragraph{Slope ($\alpha$) calculation} Regression calculation for the given regimes.

\paragraph{Regime Breaks}

The regime break is defined as the point in the spectrum at which there is a noticeable change in the slope. This change indicates that the regime at frequencies below the break has a different $\alpha$ from the regime at frequencies above the break and therefore exhibits a different characteristic randomness behaviour.

Figure \ref{fig:regime_breaks} The spectra for all 1,000+ recordings is plotted, then the geometric mean line of these spectra is plotted in order to visualise the general pattern. The a mean line is used as it is otherwise impractical to investigate the behaviour across such a large number of samples. The geometric mean is used to account for the log scale relationship for the y-axis.



This is done for each parameter independently.

\subsection{Correlation Analysis}

Partial Distance Correlation using dcor package from Python. Or Mutual information. With control for SessionID.

\draft{Only explain the method of correlation analysis, leave justification (i.e. non-linear) for the results}


\section{Results \& Discussion}

\subsection{Identified time and frequency regimes}

\begin{figure}
  \centering
  % \includegraphics{}
  \caption{Set of plots including the average line for each of the parameters, with a dotted line indicating the identified regime break.}
  \label{fig:regime_breaks}
\end{figure}


\paragraph{Discuss} What does the regime above and below the break represent? How do we interpret this? How is it useful?

\subsection{Slope values}
\begin{figure}
  \centering
  % \includegraphics{}
  \caption{Single plot of a sample power spectrum, with the regression lines of the regimes plotted.}
  \label{fig:spectral_slope}
\end{figure}

\paragraph{Not -1 slope, not pink noise}

\begin{table}[]
  \centering
  \begin{tabular}{c|c}
     & \\
     &
  \end{tabular}
  \caption{Summary statistics of slopes?}
  \label{tab:my_label}
\end{table}

\paragraph{Discussion} Problems with using fft versus periodogram resulted in incorrect interpretation of previous results.


\section{Correlations}

\paragraph{Explain why MI is better for this.}

\begin{table}[]
  \centering
  \begin{tabular}{c|c|c|c|c|c}
    Rank & ISOPleasant & ISOEventful & Traffic & Natural & Human \\
    1    & SIL (.41)   & SIL (.43)   & ...     & ...     & ...   \\
    2    & ...         & ...         & ...     & ...     & ...   \\
  \end{tabular}
  \caption{Correlation rankings between psychoacoustic and temporal features, and perceptual features}
  \label{tab:my_label}
\end{table}

\begin{figure}
  \centering
  \caption{Non-linear scatterplot for highly correlated feature(s)}
  \label{fig:scatterplot}
\end{figure}

This is where the nrmse / deviation discussion / limitation can go.

\section{Concluding Remarks}

This theme appeared in literature 10 years ago, then just went quiet.

