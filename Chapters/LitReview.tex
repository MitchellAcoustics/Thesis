\chapter{Literature Review}
\label{ch:lit}

\section{Impact of Urban Noise on Health and Wellbeing}

  \emph{Give a full formal background to why noise control is important for public health}.
  % https://www.euro.who.int/__data/assets/pdf_file/0008/383921/noise-guidelines-eng.pdf?ua=1

\section{Current Methods of Assessing and Addressing Urban Noise}

  \subsection{Acoustical Parameters}

  \subsection{ISO Environmental Acoustics Standards}
    \emph{ISO 1996-1, esp sections on annoyance, e.g. Annex F, G, H}

  \subsection{EU Noise Mapping}

  \subsection{Shortcomings}

\section{Soundscape Studies}

  \subsection{Soundscape Descriptors and Indices}

  \subsection{World Soundscape Project}

  \subsection{Swedish Soundscape Quality Protocol}


  \subsection{Demographic differences}
    Several studies have attempted to study the degree to which personal and demographic factors influence a person's soundscape perception. In some conceptions \citep{Kou2020Contexts} % cite: add Erfanian 2020
    these personal factors are classed as 'contextual' soundscape indicators - features which influence or, in a modelling context, be used as independent variables to predict the value of a soundscape descriptor. The personal factors help to create a personal soundscape interpretation model which is individual to each person.

    In this way, a person's individual state-of-mind, ethnic identity, educational background, gender identity, etc. form a pseudo-deterministic framework %! what a load of crap
    through which the physical inputs from their environment are filtered. Clearly, many of these personal factors could never be measured and even those which are measurable will have wide ranges of legitimate effects, however estimating the degree and type of effect they may have can both help us better predict individual soundscape assessments and understand how group identities influence sound perception.

    %!: Need to include earlier, more foundational studies into demographic factors

    \paragraph*{Section on Erfanian et al. 2020, Psychological Well-being}

    \paragraph*{Low-income and minority evidence} %! I think this section will need to be heavily revised for phrasing and content. I'm not happy with how I'm discussing under-represented groups.
    A consistent limitation of soundscape studies investigating the influence of personal factors is a sampling bias towards majority ethnicities (typically White British for UK studies and ethnic Chinese for Chinese studies) and middle-class and highly educated groups. % ! Hoo boy citation definitely needed
    This results in not only incomplete information about how demographics influence soundscape perception, but also represents a systemic under-representation of certain environments. While it may be unclear to what extent ethnicity and social class internally influence a person's perception, it is clear that these groups are exposed to different sound environments % cite: socio-economic studies - Huan 2019? Jian ~2015?
    and therefore studies which do not include under-represented groups are also by definition not including those sound environments which those groups inhabit.

    A recent study by \cite{Kou2020Contexts} was successful in making inroads in these under-represented environments by studying the Humboldt Park neighbourhood in Chicago, USA. Their study included
    % TODO: Finish summarising results from Kou2020

\section{Existing Predictive Models}

  \citep{Lionello2020Review}

  \subsection{Models based on non-acoustic data sources}
    \citep{Verma2020Predicting}, \citep{Gasco2020Social}
