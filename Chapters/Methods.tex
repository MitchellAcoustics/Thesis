\chapter{Methodology Part 2 - Data Analysis and Modelling}
\label{chap:methods}




%%%%%%%%%%%%%%%%%%%%%%%%%%%%%
\draft{Why questionnaires and modelling?}

\section{Questionnaires}

 The full protocol developed for this thesis is outlined in \cref{chap:protocol}. The development and presentation of this protocol involved a substantial development and testing phase, and represents a novel advancement in soundscape survey methodology. Therefore it was submitted and published as a peer-reviewed journal article in MDPI Applied Sciences as \citet{Mitchell2020Soundscape} and is presented as a stand-alone chapter within this thesis. This section therefore presents those methods not associated with the data collection procedure, i.e. the analysis and statistical methods used.

 \subsection{Likert Responses}

 \subsection{Circumplex Projection}

\section{Psychoacoustics and Auditory Perception}

 \subsection{Psychoacoustic Parameters}

   \subsubsection{Loudness}
   \emph{Zwicker and Fastl, Chap 8, see Mendeley notes and python-acoustics development notes.}

\section{Machine Learning and Regression Techniques}
Machine learning approaches are typically divided into three broad categories: supervised, unsupervised, and reinforcement learning. In supervised learning, the training data consists of input-output pairs and learns a model which can map from the inputs to the outputs. In unsupervised learning, no corresponding output data is available to the training model, thus it learns patterns in the input without feedback. Reinforcement learning does not necessarily begin with training data, instead the learning agent is given a series of reinforcements in the form of rewards and punishments \citep{StuartRussell2021}. Reinforcement learning will not be used in this thesis. Unsupervised learning has been applied to a limited degree to the acoustic data collected in several sound environments which will be expanded upon later. 

The majority of this thesis is therefore focussed on creating a supervised learning model wherein the input data are the result of measurements and the output data are the perceptual assessments of the soundscapes. In the context of this thesis, there are two primary types of supervised machine learning models - regression and classification. Regression is applied when the output is a continuous number (e.g. temperature) whereas classification is used when the output is a finite set of values. 

As will be further expanded upon in \cref{ch:circumplex}, using the trigonometric projection method provided in \citet{ISO12913Part3} enables us to transform the 8 Likert scale PAQ values into a pair of coordinate values. This transformation has a few beneficial effects for applying standard modelling techniques to soundscape data. First, it simplifies and reduces the target problem; rather than needing to model eight separate responses, we are now focussed on only two. Second, it transforms the data from ordinal responses on a 1 to 5 scale into continuous values between -1 and +1. While it is clearly possible to model ordinal outputs through classification, the methods are often less familiar and more complicated than dealing with a more standard regression problem. For those outside of machine learning (i.e. designers, engineers, etc.) regression methods, especially linear regression, are already familiar and interpretable while methods of classification and ordinal modelling are typically less familiar. By applying the \gls{iso} projection to each individual's soundscape assessment, we generate a vector of output values which can be matched up to physical data measured for each individual. This creates the sort of input-output pair vector necessary for supervised regression learning.

\subsection{Multi-level Linear Regression}

Multi-level regression modelling is a technique commonly used in fields such as psychology \cit{42, 43 from Orga paper}, and \draft{other examples}. \glspl{mlm} are particularly useful when data is organised at one or more levels or groups. The concept behind \glspl{mlm} can be built up starting from simple linear regression, as given by:

\begin{equation}
  y_i = X_i\beta + \epsilon_i
\end{equation}

For a classical multiple linear regression, we expand the $k$ coefficients out as so:

\begin{equation}
  y_i = \beta_i X_{i1} + \ldots + \beta_k X_{ik} + \epsilon_i \text{ for } i = 1, \ldots, n,
\end{equation}

where the errors $\epsilon_i$ have independent normal distributions with mean 0 and standard deviation $\sigma$

Their primary feature is the ability to have coefficients and intercepts which are allowed to vary depending on the group \citep{Gelman2006Multilevel}. This can take three forms:

\begin{enumerate}
  \item Random intercepts
  \item Random slopes
  \item Random intercept and random slopes
\end{enumerate}

In a random intercept structure, the intercept for each input feature is allowed to vary according to the second level. This structure assumes that the linear relationship between each input feature and the output is consistent across the second level groups, but that the zero point (the intercept) is different. This is expressed mathematically as:

\begin{equation}
  %TODO: insert random intercept equation from Gelman 2006.
\end{equation}

In the context of auditory perception studies, this is most appropriate for repeated measures experimental designs (as will be demonstrated in \cref{ch:whostudy,ch:mlmann}). A repeated measures study is one in all participants experience all levels of the independent variables and provide some response in terms of the output variable. In other words, each participant constitutes a group in the model and they respond to all of the input variables \citep{Kristjansson2007Multilevel}. In this case, the \gls{mlm} framework is used to account for starting differences between respondents; populations are expected to demonstrate similar behaviours in response to a given stimulus, but may have differing initial starting points, i.e. different intercepts for each analysed feature. The \gls{mlm} framework using a varying intercept for each participant allows this initial difference among individuals to be accounted for while also highlighting the overall relationship between e.g. acoustic features and annoyance ratings for a given sound. 

Random slope structures take the opposite assumption; each level shares the same intercept, while the coefficients for each feature are allowed to vary depending on the group. This assumes that different groups will have a different relationship between the input features and the output, but that these relationships may begin at a different threshold. This structure appears to be less commonly used than random intercept models. This can be mathematically described as:

\begin{equation}
  %TODO: insert random slope equation from Gelman 2006
\end{equation}



The structure inherent within the \gls{isd} means that this approach is particularly appropriate. In order to further demonstrate the structure and use of an \gls{mlm}, I'll further describe it in terms of the \gls{isd} data, where the most obvious second level for this \gls{mlm} is the location (a categorical variable defined by the LocationID). 

%TODO: Finish mlm section

\paragraph*{Repeated Measures}


 \subsection{Feature Selection}
  %  \subsubsection{Mutual Information}
  %  \draft{It appears that mutual information is related to the Bayes formula. I still need to read more into this, but it appears based on relative and overlapping probability distributions between the variables in question.}
  %  \paragraph*{From scholarpedia:}
  %  % http://www.scholarpedia.org/article/Mutual_information
  %  \draft{Based on entropy, where the uncertainty about a variable can be expressed as "the number of yes/no questions it takes to guess a random variable, given knowledge of the underlying distribution and taking the optimal question-asking strategy". "The mutual information is therefore the \emph{reduction} in uncertainty about variable $X$, or the expected reduction in the number of yes/no questions needed to guess $X$ after observing $Y$.". }

  %  \draft{"Mutual Information is just one way among many of measuring how related two variables are. However, it is a measure ideally suited for analyzing communication channels. Abstractly, a communication channel can be visualized as a transmission medium which receives an input $x$ and produces an output $y$. If the channel is \emph{noiseless}, the output will be equal to the input. However, in general, the transmission medium is noisy and an input $x$ is converted to an output $y$ with probability $P_{Y|X}(y|x)$. }
  %  \misc{This seems very useful for my conception of sound perception / auditory processing, where the perception system is a noisy communication channel.}

  %  \subsubsection{Conditional Mutual Information}
  %  The Mutual Information between two variables, given another variable as a control.

%  \subsection{Clustering Analysis}
%    \paragraph{K-means}
%    \paragraph{nbclust}

