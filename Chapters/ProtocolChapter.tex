\chapter[The SSID Protocol]{The Soundscape Indices (SSID) Protocol: A Method for Urban Soundscape Surveys -- Questionnaires with Acoustical and Contextual Information}

Conducting urban soundscape studies on a scale large enough to form a machine learning dataset presents a unique challenge. The standardised methods of conducting soundscape surveys \citep{ISO12913_2_2018IOS} are labour-intensive, time-consuming, and provide limited information about the acoustical and environmental context.

\section*{Abstract}

A protocol for characterizing urban soundscapes for use in the design of \gls{ssid} and general urban research as implemented under the European Research Council (ERC)-funded SSID project is described in detail. The protocol consists of two stages: (1) a Recording Stage to collect audio-visual recordings for further analysis and for use in laboratory experiments, and (2) a Questionnaire Stage to collect in-situ soundscape assessments via a questionnaire method paired with acoustic data collection. Key adjustments and improvements have been made to enable the collation of data gathered from research groups around the world. The data collected under this protocol will form a large-scale, international soundscape database.

\section{Introduction}
%NOTE: This intro might work better to form the thesis intro. Then jump into the protocol at "Purpose" and design

%NOTE: Could replace with an intro which ends up justifying the development of the protocol as its own stand-alone scientific work.

Soundscape studies strive to understand the perception of a sound environment, in context, including acoustic, (non-acoustic) environmental, contextual, and persona factors. These factors combine together to form a person's soundscape in complex interacting ways \cit{1}. In order to predict how people would perceive an acoustic environment, it is essential to identify the underlying acoustic and non-acoustic properties of soundscape.

The soundscape community is undergoing a period of increased methodological standardisation in order to better coordinate and communicate the findings of the field. This process has resulted in many operational tools designed to assess and understand how sound environments are perceived and apply this to shape modern noise control engineering approaches. Important topics which have been identified throughout this process are soundscape 'descriptors', 'indicators', and 'indices'. \citet{Aletta2016Soundscape} defined soundscape descriptors as "measures of how people perceive the acoustic environment"; soundscape indicators as "measures used to predict the value of a soundscape descriptor"; and soundscape indices can then be defined as "single value scales derived from either descriptors or indicators that allow for comparison across soundscapes" \citep{Aletta2018Towards}.

This conception has recently been formalised and expanded upon with the adoption of the recent ISO 12913 standard series \citep{ISO12913_1_2014IOS,ISO12913_2_2018IOS,ISO12913_3_2019IOS}. ISO 12913 Part 1 sets out the definition and conception of Soundscape, defining it as the "acoustic environment as perceived or experienced and / or understood by a person or people, in context". Here, the soundscape is separated from the idea of an acoustic environment, which encompasses all of the sound which is experienced by the receiver, including any acoustically modifying effects of the environment. In contrast, the soundscape considers the acoustic environment, but also considers the impact of non-acoustic elements, such as the listener's context and the visual setting, and how these interact with the acoustic environment to influence the listener's perception.

The ISO/TS 12913-2:2018 is the current reference document addressing data collection and reporting requirements in soundscape studies. In terms of methods, the ISO document covers two main approaches, namely: soundwalks combined with questionnaires (Methods A and B) and narrative interviews (Method C) \citep{ISO12913_2_2018IOS}, which relate to on-site and off-site data collection, accordingly. Part 3 of the ISO 12913 series builds on Part 2 and provides guidelines for analysing data gathered using only those methods \citep{ISO12913_3_2019IOS}. However, the range of possible methodological approaches to soundscape data collection is much broader and it includes, for instance, laboratory experiments \citep{Aletta2016Soundscape,Sun2019Classification,Oberman2018Towards}, pseudo-randomized experience sampling \citep{Craig2017Experience}, and even non-participatory studies \citep{Lavia2018Non}. The protocol described in this paper was designed having in mind the need for a relatively large soundscape dataset that could be used for design and modelling purposes, thus trying to expand the scope of soundwalks that typically deal with much smaller samples of participants \citep{Engel2018Review}. For the sake of comparability and standardization with these methods, we chose to refer to the soundscape attributes reported in the ISO Part 2 (Method A).

Several studies prior to the formalisation of the ISO standards on soundscape demonstrated the general, but inadequate, relationship between traditional acoustic metrics, such as $L_{Aeq}$, with the subjective evaluation of the soundscape \cit{1, 12-15}. These have typically aimed to address the existing gap between traditional environmental acoustics metrics and the experience of the sound environment. \citep{Yang2005Acoustic} showed that, when the sound level is 'lower than a certain value, say 70 dB(A)', there is no longer a significant change in the evaluation of acoustic comfort as the sound level changes. However, the perceived sound level does continue to change along with the measured sound level, showing that (1) measured sound level is not enough to predict soundscape descriptors such as 'acoustic comfort', and (2) there is a complex relationship between perceived sound level and soundscape descriptors which is mediated by other factors.

% NOTE I've written this intro section in a few places. I'm going to move on to the protocol specific stuff and decide what to do with the above later.

\section{Purpose}

The \gls{ssid} Protocol was designed to achieve two primary goals:
\begin{enumerate}
  \item gather in-situ soundscape assessments from the public, which can be further analysed and utilised in designing a soundscape index;
  \item conduct recordings needed to reproduce the audio-visual environment of a location in a laboratory setting for conducting controlled experiments on soundscape.
\end{enumerate}

These two goals represent two levels of data required for developing a general soundscape model. The first enables large scale data collection, resulting in a database with thousands of perceptual responses and their corresponding quantitative data which can be statistically analysed on a large scale, or used for training in machine learning modelling. In-situ assessments also represent the most holistic assessment, ensuring all factors that influence the soundscape are present, including those which cannot be reproduced elsewhere.

However, there are questions that cannot be practically addressed in-situ, such as soundscape assessment of less- or un-populated areas, the influence of mismatched acoustic and visual cues, physiological and neural responses to various soundscapes, and so on \cit{31}. Laboratory experiments with controlled environments are required to address these aspects. Toward the development of a coherent \gls{ssid}, therefore, it is important that these two forms of data are collected simultaneously and with compatible methods, such that the results of the two approaches can be confidently combined and compared. In addition, since this protocol is intended to be used for the creation of a large-scale international database with additions carried out by several different and remote teams, it has been designed for efficiency, scalability, and information redundancy.

\section{Protocol Design and Equipment}

The first goal is achieved by conducting in-situ questionnaires using a slightly altered version of Method A (questionnaire) from Annex C of the ISO/TS 12913-2:2018 technical specification \citep{ISO12913_2_2018IOS} collected either via handheld tablets or paper copies of the questionnaire. Typically, a minimum of 100 responses are collected at each location during multiple 2-5 hr sessions over several days. During the survey sessions, acoustic data are collected via a stationary class 1 or class 2 \gls{slm} (as defined in IEC 61672-1:2013 \cit{IEC61672}) running throughout the survey period and through binaural recordings taken next to each respondent. These acoustic and response data are linked through an indexing system so that features of the acoustic environment can be correlated with individual responses or with the overall assessment of the soundscape, as required by researchers.

The second goal is achieved by making First-Order (or higher) Ambisonic recordings simultaneously with 360 video which can be reproduced in a virtual reality environment. It has been shown that head-tracked binaural and multi-speaker ambisonic reproduction of recorded acoustic environments recorded in this way have high ecological validity \cit{33}, particularly when paired with simultaneous head-tracked virtual reality video \cit{22, 34, 35}.

The on-site procedure to collect these data are separated into two stages, which will be outlined in detail in Section \ref{sec:proc}. The stage during which the audio-visual recordings are made for lab experiments is called the \textbf{Recording Stage}, while the stage during which questionnaires and environmental data are captured is called the \textbf{Questionnaire Stage}.

The procedure has been designed to include multiple levels of data and metadata redundancy, making it robust to on-site issues and human error. The most crucial aspect of the redundancy is ensuring the perceptual responses can be matched with the appropriate corresponding environmental and acoustic data even when some information is lost or forgotten.

\subsection{Labelling and Data Organisation}

In order to be able to identify all of the many data components of the Recording and Questionnaire Stages and to associate these with their various corresponding data, the following labelling system is suggested. This system is focussed on (1) relating all of the separate recordings and factors to specific questionnaire responses and (2) efficiency and consistency on site. A recent paper by \citep{Aumond2017Modeling} demonstrated the importance of addressing multiple levels of factors which influence perception, from individual-, to session-, to location-level. The successful pleasantness models built incorporating these information levels showed a marked improvement over the equivalent individual-level or location-level only models. The data organisation system proposed here was designed in order to maintain this important information, and the levels of information for the data collected on site are shown in Table \ref{tab:data-levels}.

At the top level is the \textbf{Location} information. This includes information about the location which does not change day-to-day, and generally characterises the architectural character of the space, or typical climate conditions for the area. As described in Section \ref{sec:loc-selection}, each 'environmental unit' should be considered a new location. Therefore, if researchers want to investigate the differences in soundscape assessment in the middle of a small urban park and along the road next to the same park, these would be considered different locations since they would (typically) have different environmental factors and should be given difference names. The name chosen should be concise, but it should be obvious what location is referred to.

