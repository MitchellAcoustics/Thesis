\chapter{Introduction}
\label{ch:intro}

\section{Research Summary}
 Urban noise pollution affects 80 million EU citizens with substantial impacts on public health which are not well addressed by conventional noise control methods. Traditional noise control methods have typically limited their focus to the reduction of unwanted noise, ignoring the potential benefits of increasing positive sounds and remaining restricted by practical limitations of noise reduction. Modern approaches to achieve improved health outcomes and public satisfaction aim to incorporate a person's perception of an acoustic environment, an approach known as 'Soundscape'.

 Soundscape studies strive to understand the perception of a sound environment, in context, including acoustic, (non-acoustic) environmental, contextual, and personal factors. These factors combine together to form a person's soundscape in complex interacting ways \citep{Berglund2006Tool}. In order to predict how people would perceive an acoustic environment, it is essential to identify the underlying acoustic and non-acoustic properties of soundscape.

 % From: Upgrade report
 When attempting to apply soundscape in practical applications in the built environment, it is immediately apparent that a predictive model of the users' perceptual response to the acoustic environment is necessary. Whether to determine the impact of a design change, or to integrate a large scale data at neighbourhood and city levels, a mathematical model of the interacting factors will form a vital component of the implementation of the soundscape approach. This work is intended to identify methods for incorporating contextual and objective information into a useable and interpretable predictive model of urban soundscapes. In order to achieve this, a protocol for collecting the multi-level, multi-factor perceptual assessment data has been developed and implemented, resulting in a large soundscape database. Several avenues of investigation are then drawn from the database. The primary research questions are:

 % TODO: Try to move this to the end of the introduction.
 \begin{enumerate}
   \item What are the primary acoustic features involved in soundscape formation and what are the driving interactions between acoustic features and soundscape assessment?
   \item How does the sound source composition in a complex sound environment mediate this interaction and how can this effect be simplified and modelled?
   \item How can the multiple levels of soundscape formation be simplified and integrated into a cohesive predictive model, and what interpretations about the cross-effects of these levels can be drawn from the model?
   \item How does the soundscape of a place vary over time, is this variation driven by environmental features or by context, and can this variation be predicted?
 \end{enumerate}

\section{The SSID Project}

 \subsection{Project collaborators}

 \subsection{Motivation for the SSID Project}

\section{Research Aims}

\section{Soundscape Indices and Metrics}

 \subsection{Standardisation}
   % From: Protocol paper
   The soundscape community is undergoing a period of increased methodological standardization in order to better coordinate and communicate the findings of the field. This process has resulted in many operational tools designed to assess and understand how sound environments are perceived and apply this to shape modern noise control engineering approaches. Important topics which have been identified throughout this process are soundscape 'descriptors', 'indicators', and 'indices'. \citet{Aletta2016Soundscape} defined soundscape descriptors as "measures of how people perceive the acoustic environment"; soundscape indicators as "measures used to predict the value of a soundscape descriptor; soundscape indices can then be defined as "single value scales derived from either descriptors or indicators that allow for comparison across soundscapes" \cite{Kang2019Towards}.

   This conception has recently been formalized and expanded upon with the adoption of the recent ISO 12813 set of standards \citep{ISO12913_1_2014IOS,ISO12913_2_2018IOS,ISO12913_3_2019IOS}. ISO 12913 Part 1 sets out the definition and conception of Soundscape, defining it as the "acoustic environment as perceived or experienced and/or understood by a person or people, in context". Here, the soundscape is separated from the idea of an acoustic environment, which encompasses all of the sound which is experienced by the receiver, including any acoustically modifying effects of the environment. In contrast, the soundscape considers the acoustic environment, but also considers the impact of non-acoustic elements, such as the listener's context and the visual setting, and how these interact with the acoustic environment to influence the listener's perception.

 \subsection{Soundscape Descriptors}
   In order to consistently discuss soundscape and the factors which influence it, it is important to understand what terms have been used to describe soundscapes and to construct a consistent framework within which to work. Both the traditional focus on the epidemiological impacts of noise and the development of the soundscape concept have used many different terms in order to describe the perception of a sound environment.

   \paragraph{Noise annoyance} is perhaps the best researched aspect of environmental sound perception.

   \paragraph{Pleasantness}

   \paragraph{Quietness / Tranquillity}

   \paragraph{Acoustic Comfort}

   \paragraph{Perceived sound level}

   \paragraph{Music-likeness}

   \paragraph{Restorativeness}

   \paragraph{Soundscape quality}

   \paragraph{Appropriateness}

   \paragraph{Perceived Affective Quality (PAQ)}

 \subsection{Soundscape Indicators}
   % From: Protocol paper
   Several studies prior to the formalization of the ISO standards on soundscape demonstrated the general, but inadequate, relationship between traditional acoustic metrics, such as $L_{Aeq}$, with~the subjective evaluation of the soundscape \citep{Berglund2006Tool,Yang2005Acoustic,Rychtarikova2013Soundscape,Aumond2017Modeling,AlsinaPages2021Perceptual}. These have typically aimed to address the existing gap between traditional environmental acoustics metrics and the experience of the sound environment. Yang and Kang (2005) showed that, when the sound level is 'lower than a certain value, say 70 dBA', there is no longer a significant change in the evaluation of acoustic comfort as the sound level changes. However, the perceived sound level does continue to change along with the measured sound level, showing that (1) measured sound level is not enough to predict soundscape descriptors such as 'acoustic comfort', and (2) there is a complex relationship between perceived sound level and soundscape descriptors which is mediated by other factors.

   Subsequent studies have shown that, even with large data sets and several possible acoustic indicators examined, models that are based on objective/measurable metrics under-perform in predicting soundscape assessment when compared to models based on perceptual responses. \citet{Ricciardi2015Sound}, with a methodology based on smart phone recordings, achieved $R^2 = 0.21$ with acoustic input factors $L_{50}$ and $L_{10} - L_{90}$, whereas the same dataset and model building method achieved $R^2 = 0.52$ with perceptual input factors overall loudness (OL), visual amenity (VA), traffic (T), voice (V), and birds (B). This indicates that merely examining the acoustic level is not sufficient for predicting the assessed soundscape quality, and that additional objective factors and a more holistic and involved method of characterizing the environment is required. These previous studies have generally been limited by one or many of the following factors:

   \begin{itemize}
     \item limited number or types of locations;
     \item limited responses sample size;
     \item no non-acoustic factors.
   \end{itemize}
   These factors generally limit the generalizability of their results beyond the investigated locations.

 \subsection{The need for predictive soundscape models}
   %TODO: Probably need to combine and rephrase with "Practical Applcations for Soundscape Mapping""
   % From: Protocol paper
   The ability to predict the likely soundscape assessment of a space is crucial to implementing the soundscape concept in practical design. Current methods of assessing soundscapes are generally limited to a post-hoc assessment of an existing environment, where users of the space in question are surveyed regarding their experience of the acoustic environment \citep{Engel2018Review, Zhang2018Effect}. While this approach has proved useful in identifying the impacts of an existing environment, designers require the ability to predict how a change or proposed design will impact the soundscape of the space. To this end, a model that is built upon measurable or estimate-able quantities of the environment would represent a leap forward in the ability to design soundscapes.

   Developing soundscape indices is a process that requires consideration of how people perceive, experience, and understand the surrounding sound environment. For the purpose of modelling and comparisons,

   % ? Replace with section from Protocol paper?
   Previous soundscape research has demonstrated that perception of the acoustic environment, while primarily driven by sound level, is mediated heavily by non-acoustic factors which interact with the sound level, spectral information, and temporal acoustic behaviour in complex ways. The soundscape is influenced by several levels of factors: the immediate and long-term acoustic environment, other environmental factors (e.g. temperature, air quality), the physical / visual characteristics of the space, the type of architectural space, and even cultural and country-level expectations. When approached in a predictive model context, the acoustic data must form the core components, but a coherent framework for describing how the influence of the acoustic factors is affected by the non-acoustic factors is required.

   Simpler analyses have taken a fragmented approach, for instance where separate acoustic-factor models are built independently for each type of architectural space considered in the data set and, separately, statistical models are built to investigate another non-acoustic factor, e.g. visual greenness vs lack of greenness. In order to properly extract the influences of all of these levels of factors as well as to build a generalisable model which can be used in practice, this fragmented approach should be combined into a single multi-level model.

   % From: Turing application proposal
   The first key step for this approach is the creation of a coherent, large-scale, multi-factor database of objective environmental measurements and subjective perceptual responses. My research makes use of in-person field questionnaires, long-term manned questionnaires, and multi-factor characterisation of the environment as part of the ERC-funded project Soundscape Indices (SSID) and in collaboration with The French Institute of Science and Technology for Transport, Development and Networks (IFSTTAR) to collect this database across a wide range of locations and soundscape types. This work has already been mostly completed and the database is now ready to be put to use in building the overall soundscape predictive model.

   This approach is unique in that it:
   \begin{enumerate}
     \item fundamentally incorporates all identified factors of soundscape perception in a coherent manner;
     \item is extensible and interpretable;
     \item considers how soundscape change over both multi-hour and multi-day timescales and incorporates this dynamic behaviour for increased accuracy.
   \end{enumerate}

   \draft{(In contrast to \citet{Lacey})} Where previous ground-breaking strategies toward practical urban soundscape design \citep{Lacey}, have been limited in their scope, providing methods of improving individual soundscapes or approaches which can be applied to bespoke projects, this work aims to move towards a generalised and widely applicable engineering-based approach. The goal is to promote a soundscape mindset as the 'standard', not just as an extra add-on for forward-thinking projects or as a localised sonic rupture which, while incredible effective (and affective) within its radius, is not suited to being applied on a city- or national-policy scale. For this purpose, we require a standardised and implementable index and direction of best practice which can be implemented by trained technicians, engineers, designers, and planners across all aspects of urban design, from the billion dollar museum to the inner-city public elementary school. A desire for good and restorative soundscapes should be the baseline standard in a city's design, upon which art which highlights the 'mythic, imaginative and poetic relationships within the affective environments' \citep{Lacey} can be implemented by the specialists. The goal of this work therefore, is not to critique or counter the creative approaches taken by those within sound art or acoustic ecology, but instead to move towards a new baseline, a new way of designing all environments of the city, from the lowest to the highest (but mostly at the lowest, where it is needed most).



\section{Practical Applications for Predictive Modelling}

 The soundscape approach faces several challenges in practical applications which are unaddressed by current assessment methods, but which may be solved through the development of a predictive modelling framework. The first of these challenges is predicting how a change in an existing sound environment will be reflected in the soundscape. While it is possible in this scenario to measure the existing soundscape via questionnaire surveys, if a change is then introduced to the acoustic environment, it is so far impossible to say what the resulting soundscape change would be. This question relates strongly to the idea of soundscape interventions; where a particular noise pollution challenge is addressed by introducing more pleasant sounds (e.g. a water feature), following the soundscape principle of treating sound as a resource. Predicting how much a particular intervention would improve the soundscape (or, indeed whether it would improve at all) is not yet possible with the retrospective methods available. This question is also addressed in Study III of this thesis \draft{cite Lockdown paper} which uses a predictive model to look at how the changes in the acoustic environment due to the COVID-19 lockdowns resulted in changes in the soundscapes of the spaces.

 Retrospective assessment methods also struggle to capture the dynamics of the soundscape in a space. Whether through the narrative interview method of \draft{section of ISO12913-2}, through soundwalks, or through in-situ questionnaires \citep{Mitchell2020Soundscape}, only the soundscape during the particular period which the researchers are actively investigating is captured. This makes it very difficult to determine diurnal, seasonal, or yearly patterns of the soundscape. These patterns may be driven by corresponding diurnal, seasonal, or yearly patterns in the acoustic or visual environment, or by variations in how people process and respond to the sound at different times of day/season/year. Currently the only way to investigate any of these patterns is through repeated surveys. Predictive modelling, on the other hand, could allow a trained soundscape model to be paired with longterm monitoring methods to track how a soundscape may change in response to changes in the acoustic environment. \draft{Should mention De Coensel's saliency summary as a solution here.}

 Admittedly, this method would not be able to answer the second part of the question - how do people's responses to a given acoustic and visual environment change throughout the various daily/seasonal/yearly periods? \draft{This part should maybe be moved to a discussion}One approach to answering this question which has not, as far as the author is aware, been employed is through an un-attended survey method. Such a method could involve creating and posting fliers asking users of a space to complete a soundscape survey (accessed through a QR code) and leaving these fliers installed for longer periods of time. It is unclear how successful such a general approach would be, in particular what response rate would be expected, but given the increasing familiarity with QR codes among the general public following their use for track-and-trace during COVID-19, it does appear promising. These un-attended surveys could also be paired with long-term acoustic and environmental monitoring via a WASN or powered SLM which could simultaneously track the acoustic environment. This would thus result in a time series of online soundscape questionnaires with a corresponding time series of acoustic and environmental information, allowing us to track the changes of each over long periods of time.

 \draft{Predicting soundscape of not-yet-existing spaces}

 \draft{Soundscape mapping}

 \draft{Where to put this in this section?}The existing methods for soundscape assessment and measurement, such as those given in the ISO 12913 series, have been focussed primarily at determining the \emph{status quo} of an environment. That is, they are able to determine how the space is \emph{currently} perceived, but offer little insight into hypothetical environments. As such, they are less relevant for design purposes, where a key goal is to determine how a space \emph{will be} perceived, not just how an existing space is perceived. Toward this, and following from the combination of perceptual and objective data collection encouraged in \citet{ISO12913_2_2018IOS}, the natural push from the design perspective is towards 'predictive modeling' In this context, predictive modeling involves predicting how physical acoustic environments would likely be perceived or assessed by the users of the space.




\section{General Aim}
