\chapter{Introduction}
\label{ch:intro}

\section{Research Summary}
Urban noise pollution affects 80 million EU citizens with substantial impacts on public health which are not well addressed by conventional noise control methods. Traditional noise control methods have typically limited their focus to the reduction of unwanted noise, ignoring the potential benefits of increasing positive sounds and remaining restricted by practical limitations of noise reduction. Modern approaches to achieve improved health outcomes and public satisfaction aim to incorporate a person's perception of an acoustic environment, an approach known as \textquote{Soundscape}.

Soundscape studies strive to understand the perception of a sound environment, in context, including acoustic, (non-acoustic) environmental, contextual, and personal factors. These factors combine together to form a person's soundscape in complex interacting ways \citep{Berglund2006Tool}. In order to predict how people would perceive an acoustic environment, it is essential to identify the underlying acoustic and non-acoustic properties of soundscape.

% From: Upgrade report
When attempting to apply soundscape in practical applications in the built environment, it is immediately apparent that a predictive model of the users' perceptual response to the acoustic environment is necessary. Whether to determine the impact of a design change, or to integrate a large scale data at neighbourhood and city levels, a mathematical model of the interacting factors will form a vital component of the implementation of the soundscape approach. This work is intended to identify methods for incorporating contextual and objective information into a useable and interpretable predictive model of urban soundscapes. In order to achieve this, a protocol for collecting the multi-level, multi-factor perceptual assessment data has been developed and implemented, resulting in a large soundscape database. Several avenues of investigation are then drawn from the database and addressed throughout this thesis. The primary research questions are:

% TODO: Try to move this to the end of the introduction.
\begin{enumerate}
  \item What are the primary acoustic features involved in soundscape formation and what are the driving interactions between acoustic features and soundscape assessment?
  \item How does the sound source composition in a complex sound environment mediate this interaction and how can this effect be simplified and modelled?
  \item How can the multiple levels of soundscape formation be simplified and integrated into a cohesive predictive model, and what interpretations about the cross-effects of these levels can be drawn from the model?
  \item In what ways and to what extent can predictive soundscape modelling be applied to address future urban design challenges? How can these methods best be integrated into policy, design, noise mapping, and engineering practice?
\end{enumerate}

Towards answering these questions, the results of five %TODO: check this number at the end.
peer-reviewed studies are presented. These studies represent a series of work to 

\begin{enumerate}
  \item Advance the conceptual development and practice of soundscape studies
  \item Develop a transparent and useful method of predicting soundscape assessments
  \item Investigate the various components which influence soundscape perception, including personal factors like psychological well-being, acoustical factors, and sound source specifics and to integrate these components into the predictive modelling methods.
\end{enumerate}

\section{The SSID Project}
The \gls{ssid} Project is a five-year, multi-disciplinary project funded by a Horizon 2020 European Research Council grant (no. 740696).

\subsection{Project collaborators}

\subsection{Motivation for the SSID Project}

\section{Research Aims}

\section{Soundscape Indices and Metrics}

\section{General Aim}

\section{Thesis Structure}

This thesis presents the results of several studies which develop the conceptual and statistical frameworks to enable the prediction and presentation of the soundscape analysis of urban spaces. The first study presents a protocol for conducting large-scale soundscape assessment describes the resulting publicly available database which includes 18 locations in 4 European cities. The second study reviews the current state of research on the relationships between soundscape features and psychophysiological health and presents an initial development of the multilevel modelling approach used throughout the thesis to investigate the influence of psychological wellbeing on soundscape perception.

Studies three and four developed two approaches to incorporating context into a predictive model. The first makes use of sound source information to inform the statistical relationships between psychoacoustic metrics and noise annoyance, demonstrating that loudness is a crucial factor, regardless of the sound type. The second model presents a multilevel model which incorporates contextual information about the location.
