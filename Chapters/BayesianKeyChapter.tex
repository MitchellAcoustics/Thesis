\chapter{A Bayesian Hierarchical Predictive Soundscape Model and a Proposed Soundscape Index}
\label{ch:bayes}

\section{Introduction}
 \subsection{Probabilistic Distribution Thinking}

   % !: from JASA-EL circumplex critique
   The instruments described in the ISO 12913 Part 2 \citet{ISO12913_2_2018IOS} were originally designed primarily for the context of individual or small group assessments. In these scenarios, the focus is on assessing the particular soundscape of the person in question. Recent advances in the soundscape approach since the development of the standards have shifted some focus from individual soundscapes to characterizing the overall soundscape of public spaces \citep{Mitchell2020Soundscape}. In this context, a consideration of the natural variation in people's perception and the variation over time of a soundscape must be a core feature of how the soundscape is discussed. Boiling a public space which may have between tens and tens of thousands of people moving through it in a single day down to the mean (or median, or any other single metric) soundscape assessment completely dismisses the reality of the space. Likewise, this overall soundscape of a public space cannot possibly be determined through a 10-person soundwalk, as there is no guarantee that the sample of people engaged in the soundwalk are representative of the users of the space (in fact it is very likely they would not be).

   This shift is part of a move towards a more holistic approach to urban noise and to integrating the soundscape approach into urban design and regulations.

\section{Methods}

\section{Results}

\section{Discussion}

\section{Communicating SSID on the Basis of Percentages}

\section{Conclusion}

