\chapter[Study III: Soundscapes of the COVID-19 lockdown]{Study III: Investigating Urban Soundscapes of the COVID-19 Lockdown: A predictive soundscape modeling approach}
%%%%%%%%%%%%%%%%%%%%%%%%%%

\section*{Abstract}
% Put your abstract here. Abstracts are limited to 200 words for
% regular articles and 100 words for Letters to the Editor. Please no
% personal pronouns, also please do not use the words ``new'' and/or
% ``novel'' in the abstract. An article usually includes an abstract, a
% concise summary of the work covered at length in the main body of the
% article.
This paper is part of a special issue on COVID-19 Pandemic Acoustic Effects.

The unprecedented lockdowns enforced around the world to fight COVID-19 in spring 2020 triggered changes in human activities in public spaces. A predictive modeling approach was developed to characterize the resulting change in the perception of the sound environment when people could not be surveyed. Building on a database of soundscape questionnaires ($N=1,318$) and binaural recordings ($N=693$) collected in 13 locations across London and Venice during 2019, new recordings ($N=608$) were made in the same locations during the lockdowns in 2020. Using these 30-second-long recordings, linear multi-level models were developed to predict the pleasantness and eventfulness of the soundscapes during the lockdown and compare changes for each location. An online listening study also investigated the change in sound sources within the spaces. Results indicate: 1) human sounds were less dominant and natural sounds more dominant across all locations; 2) contextual information is important for predicting pleasantness but not for eventfulness; 3) in general perception shifted towards less eventful soundscapes and to more pleasant soundscapes for traffic-dominated locations, but not for human- and natural-dominated locations. This study demonstrates the usefulness of predictive modeling and the importance of considering contextual information when discussing the impact of sound level reductions on the soundscape.

%% pacs numbers not used

%  End of title page for Preprint option --------------------------------- %

%% See preprint.tex/.pdf or reprint.tex/.pdf for many examples

\section{Introduction}
\label{sec:intro}
The global emergency caused by the COVID-19 pandemic in early 2020 required national lockdown measures across the world, primarily targeting human activity. In the United Kingdom, construction and transport were allowed to continue, but a decrease in activity was observed \citep{hadjidemetriou_impact_2020}. In other countries, such as Italy, the restrictions were more severe and even included limiting people's movement to a certain radius from their place of residence \citep{ren_pandemic_2020}. The explorations in environmental acoustics of lockdown conditions across the world have revealed various degrees of impact on the acoustic environment, both at a city-scale \citep{asensio_madrid_2020, bonet-sola_soundscape_2021, munoz_lockdown_2020, rumpler_noise_2021, hornberg_covid_2021} and at a more local, public space-scale \citep{aletta_assessing_2020, alsina-pages_changes_2021, bonet-sola_soundscape_2021, manzano_sound_2021}. In general, these studies have demonstrated a decrease in urban noise levels and indicated a difference in the amount the level decrease depending on the type of space investigated (e.g. parks, urban squares, etc.) and the type of human activity characteristic for the space.

 Those studies were mostly focused around the $L_{Aeq}$, as well as a standardization approach to reporting subsequent changes in soundscape proposed by \citet{asensio_taxonomy_2020}. They were not able to reveal the perceptual impact of such conditions in public spaces also because of: 1) the lack of subjective data for the exact or comparable locations in previous years; and 2) the lack of participants present in public spaces during the lockdown, hence the inability to collect soundscape data in situ. Attempts have been made to bridge this gap by using social networks to source subjective data but resulted in a focus on indoor conditions following the shift in the citizens’ behavior, i.e. spending more time indoors \citep{bartalucci_survey_2021, lee_attitudes_2021}. \citet{garrido_greenCOVID_2021} relied on an online survey deployed in England, Ireland and Spain to explore the perceived change in natural environments in particular. They observed a consistent increase in the perceived presence of natural sounds across all major cities and rural areas respectively in these three countries. A very similar trend was observed in Argentina, also based on an online questionnaire without a listening task \citep{maggi_perception_2021}. By combining field recordings and focus groups,  \citet{sakagami_covid_2020} and \citet{lenzi_soundscape_2021} observed changes in the sound source composition and the affective quality of soundscape in a residential area in Kobe, Japan and a public space in Getxa, Spain, respectively, during the different stages of the lockdown period. Following the easing of lockdown measures, a decrease in animal and traffic sounds was observed in Kobe, while an increase in eventfulness, loudness, and presence of human sound sources, followed by a decrease in pleasantness, was shown in Getxa.

\citet{aletta_assessing_2020} explored the impacts of the COVID-19 lockdowns on the acoustic environment in London in particular, through many short-term (30s) binaural recordings. This study revealed that average reductions in the various locations considered ranged from 10.7 dB ($L_{Aeq}$) to 1.2 dB, with an overall average reduction of 5.4 dB.
This metric-reporting focused approach left the following research questions unanswered: how would people have perceived these spaces as a result of this change in acoustic environment (RQ1), and would these sound level reductions result in improvements to the soundscape of the spaces (RQ2)? The \nth{1} research question (RQ1), addressing the perceptual effect of the change in urban soundscape induced by the lockdowns, can be further broken down into the following questions: how was the sound source composition influenced by the change; how would the affective response to the acoustic environment in lockdowns change; and could this demonstrate the effect of human activities on the perception of an acoustic environment in general?

These questions arise out of the soundscape approach, which is characterized by prioritizing the perceptual effect of an acoustic environment by taking into account the interaction of sound sources, context, and the person perceiving it \citep{international_organisation_for_standardization_iso_2014, truax_handbook_1999}, bringing together objective and subjective factors. The soundscape approach to noise mitigation and management is being recognized as a response to arising environmental requirements on noise pollution and sustainability, such as the regulation of quiet areas in Europe \citep{european_environment_agency_environmental_2020, kang_impact_2018, radicchi_sound_2021}. This has been further formalized in \citet{international_organisation_for_standardization_isots_2018} via the adoption of the circumplex model of soundscape \citep{axelsson_principal_2010}, in which the perception of a soundscape can be described in terms of its pleasantness and eventfulness, as one of the standard methods of soundscape assessment.

Soundscape research is therefore traditionally rooted in environmental acoustics and environmental psychology, typically dealing with outdoor spaces \citep{torresin_indoor_2020} and urban open spaces, where parks and squares are often used as case study sites \citep{kang_urban_2007}. A soundscape assessment typically requires people to be surveyed but the presence of people at a location influences assessment \citep{aletta_towards_2018} and ‘quiet places’ usually require low numbers of users to remain quiet, which limits the possibility of an assessment. Even in a crowded public space, soundscape surveys are demanding as they require significant resources to carry out at scale, limiting their widespread application \citep{mitchell_soundscape_2020}. Therefore, a need for a predictive model arises to overcome this limitation and improve the implementation of the soundscape approach into everyday planning and management practices.

According to a recent review of predictive soundscape models from \citet{lionello_systematic_2020}, the degree of employing auditory and non-auditory factors in soundscape prediction varies with some studies relying on contextual \citep{kajihara_imaginary_2017}, personal/demographic \citep{erfanian_psychological_2020, tarlao_investigating_2021} or social media \citep{aiello_chatty_2016} data entirely to predict and generate soundscape features. Some methods also incorporate perceptually-derived features, such as subjective sound level and visual pleasantness as predictors \citep{lionello_systematic_2020}, however this information must also be obtained from people via a survey and therefore are unsuitable for predictive modeling where surveys are not possible. This indicates the necessity for considering and accounting for the influence which contextual factors in a space have on the relationship between the sound environment itself and the listener's perception of it (i.e. the soundscape). 

Therefore, a third research question arises: what are the key features needed for a soundscape prediction model based on comprehensive acoustic on-site measurements to be used for assessing locations with low social presence or in situations where conducting surveys is impractical (RQ3)?

\section{Materials and methods}

This study was conducted via initial onsite data collection campaigns in Central London and Venice in 2019 before the outbreak of COVID-19 as part of the Soundscape Indices (SSID) project \citep{mitchell_soundscape_2020} and in 2020 during the strictest part of the lockdowns \citep{aletta_assessing_2020}, including objective acoustic data (2019 and 2020) and subjective responses (2019 only). Using both 2019 and 2020 binaural recordings, an online listening experiment was conducted to provide an understanding about the change in sound source composition. The 2019 onsite questionnaire data were used to define the dominant sound source at each location as a starting point for interpreting soundscape change. A predictive model was developed to reveal the change in the perceived pleasantness and eventfulness using objective acoustic data and location to predict subjective responses. Although the initial (2019) dataset contains additional locations (specifically, in Spain, the Netherlands, and China), due to the nature of this study as a reaction to the strict movement and activity restrictions, the sites which could be included in the lockdown (2020) measurement campaigns were limited to locations where staff and equipment had access and where recordings could be undertaken during the spring of 2020. 

The sites were selected to provide a mixture of sizes and uses, varying in typology ranging from paved squares to small and large parks to waterside spaces across both cities. Throughout the text they are indexed via a LocationID based on the location's name (e.g. CamdenTown, SanMarco), while a more in-depth overview of each is given in supplementary files. London is taken as an example of a large, typically noisy city while the Venice sample provides a unique look at spaces with typically very high human activity levels and no road traffic activity. In particular, the 2019 Venice surveys were taken to coincide with the yearly Carnevale festival in order to capture its distinct soundscape.

The ISO/TS 12913 \citep{international_organisation_for_standardization_isots_2018} series were consulted for reporting on soundscape data. A detailed description of the 2019 survey campaigns is featured throughout the paper and in the supplementary files. This study was approved by departmental UCL IEDE Ethics Committee on \nth{17} July 2018 for onsite data collection and on the \nth{2} of June 2020 for the on-line listening experiment and is conducted in adherence to the ethical requirements of the Declaration of Helsinki \citep{WMA_Helsinki_2013}.

\subsection{Onsite data: Questionnaires, binaural measurements, and recordings}
The initial onsite data collection featured both questionnaire data collected from the general public and acoustic measurements, conducted across thirteen urban locations (in London $N=11$, in Venice $N=2$) between the \nth{28} of February and the \nth{21} of June 2019, with additional sessions in July and October 2019. A total of 1,318 questionnaire responses were collected from the general population across the measurement points during 1 – 3 hour-long campaigns in both cities in 2019, accompanied by 693 approximately 30-second long 24-bit 44.1 kHz binaural recordings. Each of the 13 locations was characterized by between 14 to 80 recordings and between 32 to 155 questionnaire responses. Mean age of the participants was 33.9 (45\% male, 53.8\% female, 0.4\% non-conforming, 0.9\% prefer-not-to-say).

The subsequent measurement campaign in 2020 mimicked the binaural recording strategy applied in the initial campaign and was performed between the \nth{6} and the \nth{25} of April 2020 in both cities, this time excluding the questionnaire. An additional 608 binaural recordings were collected on-site in 2020. 

\subsubsection{Data collection}
The 2019 data collection was performed across all the locations using the protocol based on the Method A of the \citet{international_organisation_for_standardization_isots_2018}, as described in \citep{aletta_assessing_2020, mitchell_soundscape_2020}, collected either via handheld tablets or paper copies of the questionnaire. The full questionnaire and data collection procedure are given in \citet{mitchell_soundscape_2020}, however the key parts used for this study are those addressing sound source dominance and perceived affective quality (PAQ). 

Participants are first asked to rate the perceived dominance of several sound sources, as assessed via a 5-point Likert scale, coded from 1 (Not at all) to 5 (Dominates completely). The sound sources are split into four categories: Traffic noise, Other noise, Human sounds, and Natural sounds and each is rated separately. Next are the 8 PAQs which make up the circumplex model of soundscape \citep{axelsson_principal_2010}: pleasant, chaotic, vibrant, uneventful, calm, annoying, eventful, and monotonous. These are assessed on a 5-point Likert scale from 1 (Strongly disagree) to 5 (Strongly agree). In order to simplify the results and allow for modeling the responses as continuous values, the 8 PAQs undergo a trigonometric projection to reduce them onto the two primary dimensions of pleasant and eventful, according to the procedure outlined in Part 3 of the ISO 12913 series \citep{international_organisation_for_standardization_isots_2019}. In order to distinguish the projected values from the Likert-scale PAQ responses, the projected values will be referred to as ISOPleasant and ISOEventful and can be considered to form an x-y coordinate point (x = ISOPleasant, y = ISOEventful) as explained in detail in \citet{lionello_introducing_2021}.

The calibrated binaural device SQobold with BHS II by Head Acoustics was used in both campaigns at all the locations by various operators to capture acoustic data, as mentioned in the acknowledgments. Following the established onsite protocol \citep{mitchell_soundscape_2020}, when participants were stopped in a group and filled in their responses simultaneously, a single binaural recording was used to capture their experience as a group. The purpose behind this sampling strategy was to obtain data from the perspective of a typical user, corresponding to a range of individual experiences available within an urban open space. These recordings are indexed by a GroupID such that the recording for each group is matched up to each of the corresponding respondents and their individual survey responses.

\subsubsection{Data cleaning}

The cleaning of the samples was conducted using the ArtemiS SUITE 11. The researcher discarded or cropped whole recordings, or its parts affected by wind gusts or containing noises and speech generated by the recording operator by accident or for the purpose of explaining the questionnaire to a participant. This resulted in 1,291 binaural recordings then processed further, as described in the section B.2. Psychoacoustic analyses and shown in supplementary files.

In order to maintain data quality and exclude cases where respondents either clearly did not understand the PAQ adjectives or intentionally misrepresented their answers, surveys for which the same response was given for every PAQ (e.g. `Strongly agree' to all 8 attributes) were excluded prior to calculating the ISO projected values. This is justified as no reasonable respondent who understood the questions would answer that they `strongly agree' that a soundscape is pleasant and annoying, calm and chaotic, etc. Cases where respondents answered `Neutral' to all PAQs are not excluded in this way, as a neutral response to all attributes is not necessarily contradictory. In addition, surveys were discarded as incomplete if more than 50\% of the PAQ and sound source questions were not completed. 

The site characterization per \citet{international_organisation_for_standardization_isots_2018} is available in the supplementary files, featuring the address, overall psychoacoustic characteristics of the location, typical use of each location, and pictures taken during the survey sessions.

\subsubsection{Psychoacoustic analyses}

The binaural recordings were analyzed in ArtemiS SUITE 11 to calculate the following suite of 11 acoustic and psychoacoustic features to be used as initial predictors:
\begin{enumerate}
    \item Loudness ($N_5$, sones, per ISO 532-1:2017)
    \item Sharpness (acum, per ISO 532-1:2017)
    \item Roughness (asper)
    \item Impulsiveness (iu)
    \item Fluctuation Strength (vacil)
    \item Tonality (tuHMS)
    \item Zwicker Psychoacoustic Annoyance (per \citet{zwicker_psychoacoustics_2007})
    \item $L_{Aeq}$, 30s (dB)
    \item $L_{A10}-L_{A90}$ (dB)
    \item $L_{Ceq}-L_{Aeq}$ (dB)
    \item Relative Approach (per \citet{genuit_objective_1996})

\end{enumerate}

The (psycho)acoustic predictors investigated were selected in order to describe many aspects of the recorded sound – in particular, the goal was to move beyond a focus on sound level, which currently dominates the existing literature on the acoustic effects of lockdowns noted in Section \ref{sec:intro}. In all, they are expected to reflect the sound level ($L_{Aeq}$), perceived sound level ($N_5$), spectral content (Sharpness, $L_{Ceq}-L_{Aeq}$, Tonality), temporal character or predictability (Impulsiveness, Fluctuation Strength, Relative Approach), and overall annoyance (Psychoacoustic Annoyance). These metrics have been proposed as indicators to predict perceptual constructs of the soundscape \citep{Aletta2016Soundscape, Aletta2017Dimensions} and have shown promise when combined together to form a more comprehensive model applied to real-world sounds \citep{Orga2021Multilevel}. The maximum value from the left and right channels of the binaural recording are used, as suggested in \citet{international_organisation_for_standardization_isots_2019}. 

\begin{table}[ht]
\caption{Pearson correlation coefficients between candidate acoustic features and ISOPleasant and ISOEventful across all 13 locations. Only statistically significant ($p < 0.01$) coefficients are shown.}
\label{tab:corr}
\centering
\def\arraystretch{.5}
\begin{ruledtabular}
\begin{tabular}{@{}r|cccccccccccc@{}}
\textbf{Parameter} & \textbf{ISOPl} & \textbf{ISOEv} & \textbf{PA} & $N_5$ & \textbf{S} & \textbf{R} & \textbf{I} & \textbf{FS} & \textbf{T} & $L_{Aeq}$ & $L_{A10}-L_{A90}$ & $L_{Ceq}-L_{Aea}$ \\ 
 \midrule
 ISOPleasant &  &  &  &  &  &  &  &  &  &  &  &  \\ 
  ISOEventful & -0.24 &  &  &  &  &  &  &  &  &  &  &  \\ 
  PA & -0.28 &  0.24 &  &  &  &  &  &  &  &  &  &  \\ 
  N5 & -0.37 &  0.33 &  0.94 &  &  &  &  &  &  &  &  &  \\ 
  S &      &      &  0.71 &  0.56 &  &  &  &  &  &  &  &  \\ 
  R & -0.36 &  0.32 &  0.63 &  0.74 &  0.11  &  &  &  &  &  &  &  \\ 
  I &     &       & -0.10  &      & -0.37 &  0.24 &  &  &  &  &  &  \\ 
  FS & -0.11  &  0.14 &  0.37 &  0.43 &      &  0.46 &  0.55 &  &  &  &  &  \\ 
  T & -0.21 &  0.30 &  0.58 &  0.63 &  0.12 &  0.54 &  0.16 &  0.52 &  &  &  &  \\ 
  $L_{Aeq}$ & -0.34 &  0.37 &  0.84 &  0.93 &  0.56 &  0.72 & -0.09   &  0.37 &  0.57 &  &  &  \\ 
  $L_{A10}-L_{A90}$ & -0.18 &  0.15 &  0.21 &  0.33 & -0.20 &  0.31 &  0.36 &  0.44 &  0.40 &  0.23 &  &  \\ 
  $L_{Ceq}-L_{Aeq}$ &      & -0.20 & -0.49 & -0.49 & -0.54 & -0.31 &      & -0.27 & -0.28 & -0.61 & -0.22 &  \\ 
  RA & -0.34 &  0.31 &  0.60 &  0.74 &  0.18 &  0.71 &  0.31 &  0.63 &  0.58 &  0.73 &  0.23 & -0.14 \\ 
\end{tabular}
\end{ruledtabular}
\end{table}

Table 1 shows the Pearson correlation coefficient between each of the candidate acoustic features and the outcome pleasantness and eventfulness. For ISOPleasant (\textit{ISOPl}), we can perhaps see three tiers of correlations: the more highly correlated tier $(|r| > 0.28)$ consists of \textit{RA}, $L_{Aeq}$, \textit{R}, $N_5$, and \textit{PA}; the low correlation tier consists of $L_{A10}-L_{A90}$, \textit{T}, and \textit{I}; while $L_{Ceq}-L_{Aeq}$, \textit{I}, and \textit{S} show no correlation. For ISOEventful (\textit{ISOEv}), these tiers are: \textit{RA}, $L_{Aeq}$, \textit{T}, \textit{R}, and $N_5$ comprise the most correlated tier $(|r| > 0.30)$; $L_{Ceq}-L_{Aeq}$, $L_{A10}-L_{A90}$, \textit{FS}, and \textit{PA} show low correlations; \textit{I} and \textit{S} show no correlation.

Among the correlations for the psychoacoustic metrics considered for inclusion as input features, we can see several highly correlated features. As expected, PA, $L_{Aeq}$, and $N_5$ are highly correlated, meaning that careful consideration is paid to these features to ensure they do not contribute to multicollinearity in the final model.

\subsection{Modelling}

Two linear multi-level models (MLM) were computed to predict: 1) ISOPleasant, and 2) ISOEventful. The inherent grouped structure of the SSID database necessitates a modeling and analysis approach which considers the differing relationships between the objective acoustic features and the soundscape’s perceived affective quality ratings across the various locations and contexts. The individual-level of the models is made up of the acoustic features calculated from the binaural recordings made during each respondent’s survey period, while the group-level includes the categorical `LocationID` variable indicating the location in which the survey was taken, acting as a non-auditory contextual factor. 

A separate backwards-step feature selection was performed for each of the outcome models in order to identify the minimal feature set to be used for predicting each outcome. In this feature selection process, an initial model containing all of the candidate features was fit. Each feature was then removed from the model one at a time, then the best-performing model is selected and the procedure continues step-wise until no improvement is seen by removing more features. This process is carried out first on the location-level features (including the potential to remove all features including LocationID, resulting in a ‘flat’ or standard multivariate linear regression model), then on the individual-level features. The performance criterion used for this process was the Akaike Information Criterion (AIC) \citep{akaike_hirotugu_new_1974}. To check for multicollinearity among the selected features, the variance inflation factor (VIF) was calculated and a threshold of VIF \textless{ 5} was set. Any features which remained after the backwards stepwise selection and which exceeded this threshold were investigated and removed if they were highly collinear with the other features.

All of the input features are numeric values, in the units described above. Before conducting feature selection, the input features are z-scaled to enable proper comparison of their effect sizes. After the feature selection, the scaled coefficients are used in the text when reporting the final fitted models to facilitate discussion and comparison between the features. The unscaled model coefficients are reported in Appendix \ref{app:mod} to enable the models to be applied to new data. In order to properly assess the predictive performance of the model, an 80/20 train-test split with a balanced shuffle across LocationIDs was used. The z-scaling and feature selection was performed on the training set only, in order to prevent data leakage. To score the performance of the model on the training and testing sets, we use the mean absolute error (MAE), which is in the scale of the response feature - for ISOPleasant this means our response can range from -1 to +1. However, since the end-goal of the model is to predict the soundscape assessment of the location as a whole, rather than the individual responses, we also assess the performance of the model in predicting the average response in each location. To do this, the mean response value for each location is calculated, and the $R^2$ accuracy across LocationIDs is reported for both the training and testing sets.

The model fitting and feature selection was performed using the `step` function from `lmerTest` (v3.1.3) \citep{lmertest2017} in R statistical software (v4.0.3) \citep{Rstats2020}. The summaries and plots were created using the `sjPlot` package (v2.8.6) \citep{sjPlot2021} and `seaborn` (v0.11.1) \citep{Waskom2021}.

\subsection{Online survey}

An online listening test was conducted using the Gorilla Experiment Builder (\url{www.gorilla.sc}) \citep{anwyl2020gorilla}. The participants were exposed to a random selection of 78 binaural recordings (39 from 2019 and 39 from 2020, 6 recordings per each location). Each participant had the option to evaluate either 1 or 2 sets of 6 recordings randomly assigned between 13 stimuli sets. Mp3 files, converted at 256 kBps were used due to the requirements of the Gorilla platform.

No visual stimuli were used in the experiment. The experiment consisted of: 1) an initial exercise to enhance chances of participants complying to the instructions and wearing headphones; 2) a training set using two randomly chosen binaural recordings (then not used in the main task) from the dataset; 3) a soundscape characterization questionnaire starting with an open-ended question about perceived sound sources and featuring the same questions as the one used in situ, looking into the perceived sound source dominance of the following four types: traffic noise, other noise, human sounds and natural sounds; 4) a questionnaire on the basic demographic factors. The questionnaire used in the Part 3 of the online experiment is reported in Appendix A.

Having in mind the remote nature of the study and to ensure a minimum level of robustness for reliable sound source recognition, an initial exercise was performed consisting of a headphone screening test \citep{woods_headphone_2017} and a headphone reproduction level adjustment test \citep{gontier_estimation_2019}. The level adjustment was performed using an eleven-second-long pink noise sample matched to the lowest and the highest $L_{A90}$ values from the experimental set. Participants were asked to adjust their listening level to clearly hear the quieter sample while keeping the level low enough, so they don’t find the louder sample disturbing. The headphones screening test followed, featuring a stereo signal of one-second-long 100 Hz sine tone, generated with Izotope RX 6 application, played at a 3 dB difference where one of the equally loud pairs had its phase inverted. A 100 Hz sine was used because the pilot tests revealed the 200 Hz sine tone proposed by \citet{woods_headphone_2017} created a higher uncertainty varying across different laptop models and would likely contribute to the chances of a participant fooling the test. It was expected that participants using speakers would not be able to either hear the sine wave or would be fooled by the inverted phase effect and therefore not able to pass the trials, unless they were indeed using headphones. The participant needed to recognize the quietest of the 3 samples in a trial of 6 attempts. Only participants correctly answering 5 or more out of 6 trials were allowed to proceed with the experiment. Participants were asked not to change their audio output settings during the rest of the experiment. (This was introduced to ensure that a participant is using a headphones playback system which allows a listener to clearly recognize a 3 dB difference at 100 Hz as a proxy for sufficient audio quality playback.)

Online questionnaire data was collected between the \nth{9} of June and the \nth{9} of August 2020. Within the Gorilla Experiment Builder, a total of 250 attempts to complete the experiment were recorded, where 165 participants were excluded either on the basis of not passing the headphones screening ($N=79$) or for not completing the experiment, usually before engaging into the screening ($N=83$). Out of a total of 88 participants who completed the test, 2 participants were excluded as outliers as they provided uniform answers across all the questions and commented on not being able to properly hear the stimuli, despite their successful completion of the training tests. The participants of the online experiment were of mean age 32.42, 45.1\% male, 54.9\% female.

Figure \ref{fig:framework} illustrates and summarizes the framework and sections described above. 

\begin{figure}[ht]
    \centering
    \includegraphics[width=\textwidth]{Figure1.pdf}
    \caption{The study flowchart indicating the data collection, analysis, modeling, and discussion throughout the study. The subsections in the text to which each box refers are indicated in italics.}
    \label{fig:framework}
\end{figure}

\section{Results}

The results of the onsite surveys, online experiment, and the model development are reported here. They are reported following the structure of the ISO/TS 12913 series, revealing the perceived sound source dominance, key perceptual attributes (ISOPleasant and ISOEventful) and the lockdown-related changes.

\subsection{Perceived sound source dominance}

\subsubsection{2019 sound source composition per location}

Questionnaire data was collected in English, Italian, and Spanish in both cities. The respective questionnaires can be found in the supplementary files and \citet{mitchell_soundscape_2020}. Data presented here was aggregated per LocationID.

According to the highest scored mean value of the dominant sound source type, as shown in Figure \ref{fig:barchart}, the locations can be grouped into: natural sounds dominated (RegentsParkJapan, RegentsParkFields, RussellSq), human sounds dominated (SanMarco, TateModern, StPaulsRow, StPaulsCross, MonumentoGaribaldi), noise (traffic and other noise) sounds dominated (CamdenTown, EustonTap, TorringtonSq, PancrasLock).

\begin{figure}[h]
    \centering
    \includegraphics[width=.75\textwidth]{Figure2.jpg}
    \caption{(Color online) Mean values per Location ID for the perceived dominance of the sound source types, for the 2019 on-site campaign.}
    \label{fig:barchart}
\end{figure}

\subsubsection{Overall change in the perceived sound source dominance during lockdown}

1803 words describing the sound sources present in the 2019 recordings and 1395 words related to the 2020 recordings were input by participants in response to the open-ended question Q1 (see Appendix A). The frequency of occurrence, generated using the WordClouds web app, is shown in the Figure \ref{fig:wordclo}, for the 2019 and the 2020 sets respectively. The most frequent words from both 2019 and 2020 groups are: noise, car/traffic, bird/birds, talk/voice and (foot)steps.

\begin{figure}
    \centering
    \figline{\fig{Figure3a.jpg}{8cm}{(2019)}}
    \figline{\fig{Figure3b.jpg}{8cm}{(2020)}}
    \caption{A graphic illustrating the frequency of occurrence of the sound sources reported by the participants of the online study across all locations, shown for recordings from the 2019 (above) and 2020 (below).}
    \label{fig:wordclo}
\end{figure}

The results from the listening tests deployed online were analyzed using the SPSS Statistics v. 25. Levene’s test for equality of variances resulted in highly statistically significant values for all 4 sound sources investigated (less than 0.001). Therefore, a Mann-Whitney U-test test was used as a non-parametric equivalent to the T-test to investigate the change in the perceived dominance of the four sound source types \citep{weiner_mann-whitney_2010}. The results for human sounds indicated that the perceived dominance was greater for the 2019 sample (M=3.82), than for the 2020 sample (M=2.62), U=41,656, p\textless{0.001}. The results for natural sounds indicated the perceived dominance increased from 2019 (M=2.00) to 2020 (M=2.54), U=63,797, p\textless{0.001}. However, the differences for the noise sources (traffic and other) were not statistically significant.

\begin{table}[]
\centering
\caption{Mean values and standard deviation for the perceived dominance of sound sources (rated from 1 - 5), assessed via an online survey.}
\label{tab:source}
\def\arraystretch{.5}
\begin{ruledtabular}
\begin{tabular}{@{}lccccc@{}}
\textbf{\begin{tabular}[c]{@{}l@{}}Sound source \\ type\end{tabular}} &
  \textbf{Campaign} &
  N &
  Mean &
  Std. Dev. &
  \begin{tabular}[c]{@{}c@{}}Std. Error \\ Mean\end{tabular} \\ \midrule
\textbf{Traffic} & 2019 & 422 & 2.51 & 1.369 & .067 \\
\textbf{}        & 2020 & 383 & 2.56 & 1.525 & .078 \\
\textbf{Other}   & 2019 & 422 & 2.00 & 1.182 & .058 \\
\textbf{}        & 2020 & 382 & 2.23 & 1.333 & .068 \\
\textbf{Human}   & 2019 & 423 & 3.82 & 1.143 & .056 \\
\textbf{}        & 2020 & 382 & 2.62 & 1.346 & .069 \\
\textbf{Natural} & 2019 & 424 & 2.00 & 1.307 & .063 \\
\textbf{}        & 2020 & 380 & 2.54 & 1.441 & .074 \\
\end{tabular}
\end{ruledtabular}
\end{table}

\subsection{Model selection, performance, and application}
\begin{table}[ht]
\centering
\caption{Scaled linear regression models of ISOPleasant and ISOEventful for 13 locations in London and Venice. The ISOPleasant model is a multi-level regression model with one level for individual effects and a second level for LocationID effects, while the ISOEventful model is a 'flat' multi-variate linear regression with no location effects.}
\label{tab:model}
% \includegraphics[width=\textwidth]{Table3.jpg} % included since manuscript double-spacing messes up the table formatting

\def\arraystretch{.5}
\begin{tabular}{@{}l|lccccc@{}}
\toprule
\multicolumn{1}{l|}{} &
  \multicolumn{3}{c}{\textbf{ISOPleasant}} &
  \multicolumn{3}{c}{\textbf{ISOEventful}} \\
\textit{Predictors} &
  \multicolumn{1}{c}{\textit{Estimates}} &
  \textit{CI} &
  \textit{p} &
  \textit{Estimates} &
  \textit{CI} &
  \textit{p} \\ \midrule
(Intercept) &
  \multicolumn{1}{c}{0.24} &
  0.15 - 0.33 &
  \textbf{\textless{}0.001} &
  0.14 &
  0.12 - 0.16 &
  \textbf{\textless{}0.001} \\
$N_5$             & \multicolumn{1}{c}{-0.06} & -0.10 - -0.02 & \textbf{\textless{}0.001} &       &               &                           \\
S                 & \multicolumn{1}{c}{}      &               &                           & -0.08 & -0.11 - -0.06 & \textbf{\textless{}0.001} \\
FS                & \multicolumn{1}{c}{}      &               &                           & -0.02 & -0.05 - -0.00 & \textbf{0.033}            \\
T                 & \multicolumn{1}{c}{}      &               &                           & 0.04  & 0.01 - 0.07   & \textbf{0.002}            \\
$L_{Aeq}$         & \multicolumn{1}{c}{}      &               &                           & 0.14  & 0.11 - 0.17   & \textbf{\textless{}0.001} \\
$L_{Ceq}-L_{Aeq}$ & \multicolumn{1}{c}{}      &               &                           & -0.03 & -0.05 - 0.00  & 0.052                     \\
\\
\textbf{Random Effects} &
  &
  \multicolumn{1}{l}{} &
  \multicolumn{1}{l}{} &
  \multicolumn{1}{l}{} &
  \multicolumn{1}{l}{} &
  \multicolumn{1}{l}{} \\
$\sigma^2$ &
  0.11 &
  \multicolumn{1}{l}{} &
  \multicolumn{1}{l}{} &
  \multicolumn{1}{l}{} &
  \multicolumn{1}{l}{} &
  \multicolumn{1}{l}{} \\
$\tau_{00}$       & \multicolumn{6}{l}{$0.03_{LocationID}$}                                                                                   \\
$\tau_{11}$       & \multicolumn{6}{l}{$0.02_{LocationID.L_{Aeq}}$}                                                                           \\
                  & \multicolumn{6}{l}{$0.00_{LocationID.L_{A10}-L_{A90}}$}                                                                   \\
                  & \multicolumn{6}{l}{$0.00_{LocationID.L_{Ceq}-L_{Aeq}}$}                                                                   \\
ICC &
  0.90 &
  \multicolumn{1}{l}{} &
  \multicolumn{1}{l}{} &
  \multicolumn{1}{l}{} &
  \multicolumn{1}{l}{} &
  \multicolumn{1}{l}{} \\ \midrule
N                 & \multicolumn{6}{l}{$13_{LocationID}$}                                                                                     \\
Observations &
  914 &
  \multicolumn{1}{l}{} &
  \multicolumn{1}{l}{} &
  \multicolumn{1}{l}{914} &
  \multicolumn{1}{l}{} &
  \multicolumn{1}{l}{} \\ 
  
MAE Test, Train &
0.259 &
0.259 &
 &
\multicolumn{1}{l}{0.233} &
0.231 \\
  
  \bottomrule
\end{tabular}
\end{table}

\subsubsection{ISOPleasant model selected}
Following the feature selection, the ISOPleasant model (given in Table \ref{tab:model}) has $N_5$ as the fixed effect with a scaled coefficient of -0.06, and $L_{Aeq}$, $L_{A10}-L_{A90}$, and $L_{Ceq}-L_{Aeq}$ as coefficients which vary depending on the LocationID.  The training and testing MAE are very similar, indicating that the model is neither over- nor under-fitting to the training data ($MAE_{train} = 0.259$; $MAE_{test} = 0.259$). The model performs very well at predicting the average soundscape assessment of the locations ($R^2_{train} = 0.998$; $R^2_{test} = 0.85$).

The high intraclass correlation ($ICC = 0.90$) demonstrates that the location-level effects are highly important in predicting the pleasantness dimension. Within this random-intercept random-slope model structure, these effects include both the specific context of the location (i.e. the LocationID factor), but also the $L_{Aeq}$, $L_{A10}-L_{A90}$, and $L_{Ceq}-L_{Aeq}$ features whose effects vary across locations. These slopes are given in Figure \ref{fig:random}. This point highlights the need to consider how the context of a location will influence the relationship between the acoustic features and the perceived pleasantness. 

\begin{figure}
    \centering
    \includegraphics[width=\textwidth]{Figure4.jpg}
    \caption{(Color online) Location-level scaled coefficients for the ISOPleasant model.}
    \label{fig:random}
\end{figure}

\subsubsection{ISOEventful model selected}

Through the group-level feature selection, all of the group-level coefficients were removed, including the LocationID factor itself. Therefore the final ISOEventful model is a ‘flat’ multi-variate linear regression model, rather than a multi-level model. The ISOEventful model is a linear combination of S, FS, T, $L_{Aeq}$, and $L_{Ceq}-L_{Aeq}$. The training and testing MAE are very similar, indicating that the model is not over-fit to the training data ($MAE_{train} = 0.233$; $MAE_{test} = 0.231$). The model performs slightly worse than the ISOPleasant at predicting the mean location responses, but still performs well ($R^2_{train} = 0.873$; $R^2_{test} = 0.715$). 

\subsubsection{Application to lockdown data}

\begin{figure}
    \centering
    \includegraphics[width=.65\textwidth]{Figure5.jpg}
    \caption{(Color online) Soundscape circumplex coordinates for (a) the mean ISOPleasant and ISOEventful responses for each location; and (b) the mean predicted responses based on recordings made during the lockdown and the location's movement in the circumplex.}
    \label{fig:locations}
\end{figure}

Once the two models were built and assessed, they were then applied to the lockdown recording data in order to predict the new soundscape ISO coordinates. Figure \ref{fig:locations}(a) shows the pre--lockdown ISO coordinates for each location and Figure \ref{fig:locations}(b) shows how the soundscapes are predicted to have been assessed during the lockdown period. As in the model assessment process, the predicted responses are calculated for each recording individually, then the mean for each location is calculated and plotted on the circumplex.

In 2019 the majority of locations in the dataset fall within the ‘vibrant’ quadrant of the circumplex, particularly those which are primarily dominated by human activity (e.g. San Marco, Tate Modern). Camden Town and Euston Tap, which are both in general visually and acoustically dominated by traffic, are the only two to be rated as ‘chaotic’, while no locations are overall considered to be ‘monotonous’. During the 2020 lockdown, there is general positive move along the ‘pleasant’ dimension and general negative move along the ‘eventful' dimension, but several different patterns of movement can be noted. These are investigated further in the Discussion section below.

\section{Discussion}

\subsection{Interpretation of the results}

To interpret the results addressing the RQ1 and RQ2, it is necessary to separately look into the overall change in sound source composition, and the change in the affective quality of soundscapes per location. 

\subsubsection{Change in the sound source composition}

The open-ended question about sound sources in the online survey did not reveal a change in sound source types but rather confirmed that all types were still present in both conditions. The sound source composition question taken from the Method A of the \citet{international_organisation_for_standardization_isots_2018} revealed a statistically significant reduction in human sound sources and a significant increase in the perceived dominance of natural sound sources. %(This has implications for the ISO 12913 series.)

The most frequent sound sources detected from the open-ended question correspond to the main four sound source types investigated, which indicated that all types remained present in the lockdown condition (at all the locations). While traffic intensity might have gone down, where the results of the Mann-Whitney U-test were inconclusive, but supported by the psychoacoustic measurements \citep{aletta_assessing_2020}, traffic-related sound sources were still clearly present.

The sound source composition of an outdoor acoustic environment is extremely complex. Removing one component, such as human sounds, has implications on the whole \citep{gordo_rapid_2021}. Testing the effects of this in-situ is not straightforward and interpreting this study in line with 'what is the impact of human sounds' must be taken within the broader context of the range of conditions which changed within the acoustic environment. However, looking at the overarching picture, the lockdown condition was a useful and unique case study to understand the impact which human activities -- and the human sound source type in particular -- can have on soundscape perception of urban open spaces.

\subsubsection{Movement of soundscapes}

\begin{figure}
    \centering
    \includegraphics[width=0.75\textwidth]{Figure6.jpg}
    \caption{(Color online) The relative movement of soundscape perception in the circumplex due to the COVID-19 lockdowns, represented as vectors centered on the origin. *The lawn-works dominated session is shown separately as MonumentoGaribaldi* with a gray arrow to indicate that this is distinct from the effects of the lockdown changes.}
    \label{fig:vectors}
\end{figure}

In order to interpret how the change of the acoustic environment at the locations examined would have been perceived, and to answer RQ2, movement vectors within the circumplex space are shown in Figure \ref{fig:vectors}. This clearly shows a few different patterns of movement due to the effects of the 2020 lockdown. These can be further looked into depending on 1) the magnitude of change; 2) the direction of change; 3) shift between the quadrants shown in Figure \ref{fig:locations}; 4) sound source composition. 

The largest change is seen in Piazza San Marco, with a predicted increase in pleasantness of 0.24 and a decrease in eventfulness of 0.44, enough to move the soundscape out of the ‘vibrant’ quadrant and into ‘calm’. This extreme change (relative to the rest of the locations) is exactly what would be expected given the unique context of the measurements taken in 2019 – the measurement campaign corresponded with Carnevale, a yearly festival which centers around the square. By contrast, due to the particularly strict measures imposed in Italy, during the lockdown measurement period, the square was almost entirely devoid of people. What is promising is that, without any of this contextual information about the presence or absence of people, our model is able to capture and reflect what may be considered a reasonable and expected direction and scale of movement within the soundscape circumplex.

The next locations of interest are those which, in the 2019 survey data, were rated as being dominated by traffic noise: Euston Tap, Camden Town, Torrington Square, and Pancras Lock. These are the only locations (besides San Marco) which show a predicted increase in pleasantness. Of these traffic-dominated spaces, the two which were most heavily dominated by traffic noise (Camden Town and Euston Tap) showed the most increase in pleasantness, with Torrington Square having slightly less of an increase. Pancras Lock, which was also rated as having high levels of both Human and Natural sounds shows only a modest improvement in pleasantness.

Among the locations which are predicted to experience a negative effect on pleasantness we see a mix of spaces which were assessed as being dominated by Human (St Pauls Cross and Tate Modern) and Natural (Regents Park Japan, Regents Park Fields, Russell Square) sounds before the lockdown. It is hard to discern a pattern of difference between these two groups, although it appears that the Human-dominated spaces saw a greater reduction in eventfulness, compared to the Natural-dominated spaces. 

In general, we note that most of the spaces experience some degree of reduction in  eventfulness. This pattern is particularly consistent with what would be expected from a reduction in human presence in these spaces \citep{aletta_towards_2018}, as reflected by the observation that, in general, those spaces which had the most human sounds prior to the lockdown showed the greatest reduction in eventfulness during the lockdown. 

An unexpected result is that Euston Tap is predicted to experience an increase in eventfulness and it is unclear whether this accurately reflects the real experience people would have had in the space. Normally, Euston Tap is a mostly-outdoor drinking venue located at the entrance to the Euston Train station and situated directly along a very busy central London road. During the 2020 survey, the researchers noted that the music and chatter of people from the pub was noticeably missing, but that the perceived reduction in road traffic was minimal. Based on the theory of vibrancy which would suggest it is driven by human presence and sounds \citep{aletta_towards_2018}, we would not therefore expect a shift in the vibrant direction as indicated here. This discrepancy may reveal a weakness in the context-independent ISOEventful model, or it may in fact be indicating that, at certain thresholds of traffic noise, a reduction in level -- and therefore a reduction in energetic masking -- will allow other aspects of the sound to influence the perception. 

Finally, special attention should be paid to the results shown for Monumento Garibaldi, which in 2019 was perceived as a pleasant and slightly calm green space featuring a gravel walkway. During the first measurement session during the lockdown in 2020, the researcher noted that the soundscape was dominated by landscaping works, in particular noise from strimmers (or weed whackers). In order to gain a sample which was more representative of the impact of the lockdowns, the researcher returned another day to repeat the measurements without interference from the works.

To examine the impact of these two scenarios separately, the prediction model was fitted to the data from the two sessions independently and the session which was impacted by the landscaping works is shown in Figure \ref{fig:vectors} in gray and labeled MonumentoGaribaldi*, while the unaffected session is shown in red. In the latter case, the predicted change in soundscape as a result of the lockdown fits neatly into what would be expected and closely matches the predicted behavior of similar locations in London (i.e. Marchmont Garden and Russell Square). On the other hand, the session which was dominated by noise from the strimmers is predicted to have become much more chaotic, with a decrease in pleasantness of 0.16 and an increase in eventfulness of 0.27. This indicates that, although the model has no contextual information about the type of sound and in fact the training data never included sounds from similar equipment, just based on the psychoacoustic features of the sound it is able to reasonably predict the expected change in soundscape. 

As a whole, the primary impact of the 2020 lockdowns on the soundscapes in London and Venice was an overall decrease in eventfulness. With the exception of Euston Tap, all of the sessions show some degree of reduction in eventfulness, reflecting the general decrease in sound levels and human sound sources across the locations. The impact of the lockdowns on pleasantness is more mixed and seems to be driven by the previous dominance of traffic noise in the space. However, it could also be noted that, while all locations experienced a reduction in sound level, those which are predicted to become more pleasant had an average $L_{Aeq}$ above 60 dB in 2019. By contrast, the locations which were predicted to experience a decrease in pleasantness generally had sound levels below 60 dBA in 2019. This may indicate that reductions in sound level can improve pleasantness when the sound level exceeds some threshold of around 60 - 65 dBA but are ineffective when sound levels are below this threshold. Similarly, \citet{Yang2005Acoustic} showed that, when the sound level is 'lower than a certain value, say 70 dB' there is no longer a significant improvement in the evaluation of acoustic comfort as the sound level reduces. It is unclear at this point where this threshold would lie for pleasantness/annoyance, how strict it may be, or how it is impacted by the sound source composition of the acoustic environment, therefore further research is needed in this area.

\subsubsection{Model selection results}

The most immediately interesting result of the model-building and feature selection process, answering to the RQ3, is the apparent irrelevance of location context to the ISOEventful dimension. The multilevel model structure was chosen since the starting assumption was that soundscape perception is heavily influenced by contextual factors, such as expectations of the space and visual context (references). For this modeling, these factors can be considered as location-level latent variables at least partially accounted for by the inclusion of the LocationID as the second-level factor. While this assumption certainly held true for ISOPleasant, our results indicate that these types of contextual factors are not significant for ISOEventful, and do not affect the relationship between the acoustic features of the sound and the perception.

In particular this result may herald a shift in modeling approach for soundscapes – where previous methods, in both the soundscape and noise paradigms, have mostly focused on deriving acoustic models of annoyance (in other words have focused on the ISOPleasant dimension) perhaps they should instead consider the acoustic models as primarily describing the eventfulness dimension when considered in situ. In addition this study takes the approach of modeling responses at an individual level in order to derive the soundscape assessment of the location. Rather than either attempting to represent the predicted response of an individual person -- which is less useful in this sort of practical application -- or to base the model on average metrics of the location, the goal is instead to characterize the location itself, through the aggregated predicted responses of individuals. The authors believe this modeling approach better addresses the practical goal of predictive soundscape modeling and reflects the structure of the data collection.

\subsection{Limitations of the study}
The onsite sampling method was initially not intended as the ultimate characterization of a location's soundscape but rather as a tool for model development. Therefore, the change observed does not necessarily represent the ground truth about the site's soundscape, if such a thing exists. Further, the online listening tests took a relatively small but random sample from the available database and did not include any contextual information. This proved to be sufficient for the purpose of detecting a change in sound source composition, however the relatively small sample of recordings included in the online study does limit how representative they are of the location's sound environment as a whole.

The surveys and recordings taken represent only a snapshot of the soundscape or sound environment for a short period in time. This is a flaw in most soundscape sampling methods presented both in the literature and in ISO/TS 12913-2. To truly be said to characterize the soundscape of a space, long-term monitoring and survey methods will need to be developed in order to capture the changing environmental and contextual conditions in the space. Models of the sort presented here, which are based on measurable quantities, could prove to be useful in this sort of longterm monitoring as they could take continuous inputs from sensors and generate the likely soundscape assessment over time.

Further, the lockdown condition is likely to cause distortions of the circumplex soundscape perception model. Therefore, it is important to acknowledge that all the predictions were made for the people with no experience of the pandemic and its psychological effects. Conceptually, this model captured the perceptual mapping (i.e. the relationship between the acoustic indicator inputs and the soundscape descriptor outputs) of people in 2019, but this perceptual mapping is likely to have been affected by the psychological and contextual impacts of the lockdown itself, independent of its changes on the sound environment. Future research might look into potential perception changes in the post-pandemic world.

% Finally, although a wide array of psychoacoustic features were included in the initial model selection, it's unclear whether the 'top-level' features used are the most ideal. 

\section{Conclusion}

This study demonstrates an application of predictive modeling to the field of soundscape studies. The model building results reveal that, within this dataset, an approach based on psychoacoustics can achieve $R^2 = 0.85$ for predicting the pleasantness of locations and $R^2 = 0.715$ for predicting the eventfulness. A modeling--focused method of this sort is a key component to the potential scalability of the soundscape approach to applications such as smart city sensing, urban planning, and cost-effective, sustainable design. To demonstrate the usefulness and feasibility of such an approach, we apply our predictive model to a unique case study in which traditional soundscape survey methods were impossible. 

By applying this predictive model to recordings collected during the 2020 lockdown, the change in perception of the urban soundscapes is revealed. In general, soundscapes became less eventful, and those locations which were previously dominated by traffic noise became more pleasant. By contrast, previously human- and natural-dominated locations are in fact predicted to become less pleasant despite the decrease in sound levels. Although these results are limited in that they represent one snapshot of the soundscape of the spaces, the success of the model in responding to new and disturbing sound events demonstrates its potential usefulness in long-term monitoring of urban soundscapes.

%% before appendix (optional) and bibliography:
\begin{acknowledgments}
This project has received funding from the European Research Council (ERC) under the European Union’s Horizon 2020 research and innovation program (Grant Agreement No. 740696, project title: Soundscape Indices – SSID). More information and related publications can be found at the CORDIS webpage of the project: \url{https://cordis.europa.eu/project/rcn/211802/factsheet/en}.

The authors would like to thank Zhongzhe Li for conducting binaural recordings at Euston Square Gardens and Torrington Square in 2019. The authors would like to thank Meihui Ba, Nicolas Assiotis, Veronica Rugeles Allan, Yu Wang and Hua Su for their help in conducting on-site surveys during the spring and the autumn of 2019.

On-site study data were collected and managed using REDCap electronic data capture tools hosted at University College London (UCL).

\end{acknowledgments}

% -------------------------------------------------------------------------------------------------------------------
%   Appendix  (optional)

\appendix
\section{Online Questionnaire}

\begin{table}[ht!]
\centering
\caption{Questionnaire deployed via the Gorilla Experiment Builder}
\label{tab:gorilla}
\def\arraystretch{.5}
\resizebox{.7\textwidth}{!}{%
\begin{tabular}{@{}ll@{}}
\toprule
\textbf{Q1} & \textbf{\begin{tabular}[c]{@{}l@{}}While listening, please note any sound sources \\ you can identify in this sound environment:\end{tabular}} \\ \midrule
\textbf{Q2} & \textbf{To what extent have you heard the following four types of sounds?}      \\ \cmidrule(l){2-2} 
            & \textbf{Traffic noise (e.g. cars, buses, trains, airplanes)}                    \\
            & Not at all / A little / Moderately / A lot / Dominates completely               \\
            & \textbf{Other noise (e.g. sirens, construction, industry, loading of goods)}    \\
            & Not at all / A little / Moderately / A lot / Dominates completely               \\
            & \textbf{\begin{tabular}[c]{@{}l@{}}Sounds from human beings \\ (e.g. conversation, laughter, children at play, footsteps)\end{tabular}}        \\
            & Not at all / A little / Moderately / A lot / Dominates completely               \\
            & \textbf{Natural sounds (e.g. singing birds, flowing water, wind in vegetation)} \\
            & Not at all / A little / Moderately / A lot / Dominates completely               \\ \bottomrule
\end{tabular}%
}
\end{table}

\section{Model Results}
\label{app:mod}
Table \ref{tab:unscl} presents the unscaled coefficients for the ISOPleasant and ISOEventful predictive models. The scaled coefficients are presented in the body of the text to facilitate comparisons between the various factors. However, we feel it is important to present unscaled coefficients such that these models could be implemented and compared for future work. 


\begin{table*}[ht!]
\centering
\caption{Unscaled linear regression models of ISOPleasant and ISOEventful for 13 locations in London and Venice.}
\label{tab:unscl}
% \includegraphics[width=\textwidth]{Table5.jpg}
\def\arraystretch{.5}
\begin{tabular}{@{}l|lccccc@{}}
\toprule
\multicolumn{1}{l|}{} &
  \multicolumn{3}{c}{\textbf{ISOPleasant}} &
  \multicolumn{3}{c}{\textbf{ISOEventful}} \\
\textit{Predictors} &
  \multicolumn{1}{c}{\textit{Estimates}} &
  \textit{CI} &
  \textit{p} &
  \textit{Estimates} &
  \textit{CI} &
  \textit{p} \\ \midrule
(Intercept) &
  \multicolumn{1}{c}{0.39} &
  0.28 - 0.50 &
  \textbf{\textless{}0.001} &
  -0.77 &
  -1.05 - -0.48 &
  \textbf{\textless{}0.001} \\
$N_5$ &
  \multicolumn{1}{c}{-0.01} &
  -0.01 - -0.00 &
  \textbf{\textless{}0.001} &
  &
  &
  \\
S &
  \multicolumn{1}{c}{} &
  &
  &
  -0.17 &
  -0.23 - -0.12 &
  \textbf{\textless{}0.001} \\
FS &
  \multicolumn{1}{c}{} &
  &
  &
  -1.36 &
  -2.61 - -0.11 &
  \textbf{0.033} \\
T &
  \multicolumn{1}{c}{} &
  &
  &
  0.24 &
  0.08 - 0.39 &
  \textbf{0.002} \\
$L_{Aeq}$ &
  \multicolumn{1}{c}{} &
  &
  &
  0.02 &
  0.02 - 0.02 &
  \textbf{\textless{}0.001} \\
$L_{Ceq}-L_{Aeq}$ &
  \multicolumn{1}{c}{} &
  &
  &
  -0.01 &
  -0.02 - 0.00 &
  0.052 \\
  \\
\textbf{Random Effects} &     & \multicolumn{1}{l}{} & \multicolumn{1}{l}{} & \multicolumn{1}{l}{}    & \multicolumn{1}{l}{} & \multicolumn{1}{l}{} \\
$\sigma^2$ &
  0.11 &
  \multicolumn{1}{l}{} &
  \multicolumn{1}{l}{} &
  \multicolumn{1}{l}{} &
  \multicolumn{1}{l}{} &
  \multicolumn{1}{l}{} \\
$\tau_{00}$ &
  \multicolumn{6}{l}{$1.01_{LocationID}$} \\
$\tau_{11}$ &
  \multicolumn{6}{l}{$0.00_{LocationID.L_{Aeq}}$} \\
 &
  \multicolumn{6}{l}{$0.00_{LocationID.L_{A10}-L_{A90}}$} \\
 &
  \multicolumn{6}{l}{$0.00_{LocationID.L_{Ceq}-L_{Aeq}}$} \\
ICC &
  0.90 &
  \multicolumn{1}{l}{} &
  \multicolumn{1}{l}{} &
  \multicolumn{1}{l}{} &
  \multicolumn{1}{l}{} &
  \multicolumn{1}{l}{} \\ \midrule
N &
  \multicolumn{6}{l}{$13_{LocationID}$} \\
Observations            & 914 & \multicolumn{1}{l}{} & \multicolumn{1}{l}{} & \multicolumn{1}{l}{914} & \multicolumn{1}{l}{} & \multicolumn{1}{l}{} \\ \bottomrule
\end{tabular}
\end{table*}

%If only one appendix, please use
%\appendix*
%\section{Appendix title}


%=======================================================
%IMPORTANT

%Use \bibliography{<name of your .bib file>}+
%to make your bibliography with BibTeX. 
% \bibliography{JASA_Lockdown.bib}

\begin{thebibliography}{52}
\def\enquote#1{``#1,''}
\def\plainquote#1{``#1''}
\expandafter\ifx\csname natexlab\endcsname\relax\def\natexlab#1{#1}\fi
\providecommand{\dourl}[1]{\href{http://#1}{\nolinkurl{#1}}}
\providecommand{\bibinfo}[2]{#2}
\providecommand{\noopsort}[1]{}
\providecommand{\switchargs}[2]{#2#1}
  \def\eatspace #1{#1}
\providecommand{\dodoi}[1]{doi: \href{http://dx.doi.org/#1}{\nolinkurl{#1}}}

\bibitem[{Aiello \emph{et~al.}(2016)Aiello, Schifanella, Quercia, and
  Aletta}]{aiello_chatty_2016}
\bibinfo{author}{Aiello, L.~M.}, \bibinfo{author}{Schifanella, R.},
  \bibinfo{author}{Quercia, D.},  and \bibinfo{author}{Aletta, F.}
  (\textbf{\bibinfo{year}{2016}}). \enquote{\bibinfo{title}{Chatty maps:
  constructing sound maps of urban areas from social media data}}
  \bibinfo{journal}{Royal Society Open Science} \textbf{3}(3),
  \bibinfo{pages}{150690},
  \dourl{https://royalsocietypublishing.org/doi/10.1098/rsos.150690},
  \dodoi{10.1098/rsos.150690} \bibinfo{note}{number: 3}.

\bibitem[{{Akaike, Hirotugu}(1974)}]{akaike_hirotugu_new_1974}
\bibinfo{author}{{Akaike, Hirotugu}} (\textbf{\bibinfo{year}{1974}}).
  \enquote{\bibinfo{title}{A {New} {Look} at the {Statistical} {Model}
  {Identification}}} \bibinfo{journal}{IEEE Transactions on Automatic Control}
  \textbf{19}(6), \bibinfo{pages}{716--723} \bibinfo{note}{number: 6}.

\bibitem[{Aletta \emph{et~al.}(2017)Aletta, Axelsson, and
  Kang}]{Aletta2017Dimensions}
\bibinfo{author}{Aletta, F.}, \bibinfo{author}{Axelsson, {\"{O}}.},  and
  \bibinfo{author}{Kang, J.} (\textbf{\bibinfo{year}{2017}}).
  \enquote{\bibinfo{title}{{Dimensions underlying the perceived similarity of
  acoustic environments}}} \bibinfo{journal}{Frontiers in Psychology}
  \textbf{8}(JUL), \bibinfo{pages}{1--11}, \dodoi{10.3389/fpsyg.2017.01162}.

\bibitem[{Aletta and Kang(2018)}]{aletta_towards_2018}
\bibinfo{author}{Aletta, F.},  and \bibinfo{author}{Kang, J.}
  (\textbf{\bibinfo{year}{2018}}). \enquote{\bibinfo{title}{Towards an {Urban}
  {Vibrancy} {Model}: {A} {Soundscape} {Approach}}}
  \bibinfo{journal}{International Journal of Environmental Research and Public
  Health} \textbf{15}(8), \bibinfo{pages}{1712},
  \dourl{http://www.mdpi.com/1660-4601/15/8/1712},
  \dodoi{10.3390/ijerph15081712} \bibinfo{note}{number: 8}.

\bibitem[{Aletta \emph{et~al.}(2016)Aletta, Kang, and
  Axelsson}]{Aletta2016Soundscape}
\bibinfo{author}{Aletta, F.}, \bibinfo{author}{Kang, J.},  and
  \bibinfo{author}{Axelsson, {\"O}.} (\textbf{\bibinfo{year}{2016}}).
  \enquote{\bibinfo{title}{{Soundscape descriptors and a conceptual framework
  for developing predictive soundscape models}}} \bibinfo{journal}{Landscape
  and Urban Planning} \textbf{149}, \bibinfo{pages}{65--74},
  \dodoi{10.1016/j.landurbplan.2016.02.001}.

\bibitem[{Aletta \emph{et~al.}(2020)Aletta, Oberman, Mitchell, Tong, and
  Kang}]{aletta_assessing_2020}
\bibinfo{author}{Aletta, F.}, \bibinfo{author}{Oberman, T.},
  \bibinfo{author}{Mitchell, A.}, \bibinfo{author}{Tong, H.},  and
  \bibinfo{author}{Kang, J.} (\textbf{\bibinfo{year}{2020}}).
  \enquote{\bibinfo{title}{Assessing the changing urban sound environment
  during the {COVID}-19 lockdown period using short-term acoustic
  measurements}} \bibinfo{journal}{Noise Mapping} \textbf{7}(1),
  \bibinfo{pages}{123--134},
  \dourl{https://www.degruyter.com/view/journals/noise/7/1/article-p123.xml},
  \dodoi{10.1515/noise-2020-0011} \bibinfo{note}{number: 1}.

\bibitem[{Alsina-Pag\`{e}s \emph{et~al.}(2021)Alsina-Pag\`{e}s, Bergad\`{a},
  and Mart\'{\i}nez-Suqu\'{\i}a}]{alsina-pages_changes_2021}
\bibinfo{author}{Alsina-Pag\`{e}s, R.~M.}, \bibinfo{author}{Bergad\`{a}, P.},
  and \bibinfo{author}{Mart\'{\i}nez-Suqu\'{\i}a, C.}
  (\textbf{\bibinfo{year}{2021}}). \enquote{\bibinfo{title}{Changes in the
  soundscape of {Girona} during the {COVID} lockdown}} \bibinfo{journal}{The
  Journal of the Acoustical Society of America} \textbf{149}(5),
  \bibinfo{pages}{3416--3423},
  \dourl{https://asa.scitation.org/doi/10.1121/10.0004986},
  \dodoi{10.1121/10.0004986} \bibinfo{note}{number: 5}.

\bibitem[{Anwyl-Irvine \emph{et~al.}(2020)Anwyl-Irvine, Massonni{\'e}, Flitton,
  Kirkham, and Evershed}]{anwyl2020gorilla}
\bibinfo{author}{Anwyl-Irvine, A.~L.}, \bibinfo{author}{Massonni{\'e}, J.},
  \bibinfo{author}{Flitton, A.}, \bibinfo{author}{Kirkham, N.},  and
  \bibinfo{author}{Evershed, J.~K.} (\textbf{\bibinfo{year}{2020}}).
  \enquote{\bibinfo{title}{Gorilla in our midst: An online behavioral
  experiment builder}} \bibinfo{journal}{Behavior research methods}
  \textbf{52}(1), \bibinfo{pages}{388--407}.

\bibitem[{Asensio \emph{et~al.}(2020{\natexlab{a}})Asensio, Aumond, Can,
  Gasc\'{o}, Lercher, Wunderli, Lavandier, de~Arcas, Ribeiro, Mu\~{n}oz, and
  Licitra}]{asensio_taxonomy_2020}
\bibinfo{author}{Asensio, C.}, \bibinfo{author}{Aumond, P.},
  \bibinfo{author}{Can, A.}, \bibinfo{author}{Gasc\'{o}, L.},
  \bibinfo{author}{Lercher, P.}, \bibinfo{author}{Wunderli, J.-M.},
  \bibinfo{author}{Lavandier, C.}, \bibinfo{author}{de~Arcas, G.},
  \bibinfo{author}{Ribeiro, C.}, \bibinfo{author}{Mu\~{n}oz, P.},  and
  \bibinfo{author}{Licitra, G.} (\textbf{\bibinfo{year}{2020}}{\natexlab{a}}).
  \enquote{\bibinfo{title}{A {Taxonomy} {Proposal} for the {Assessment} of the
  {Changes} in {Soundscape} {Resulting} from the {COVID}-19 {Lockdown}}}
  \bibinfo{journal}{International Journal of Environmental Research and Public
  Health} \textbf{17}(12), \bibinfo{pages}{4205},
  \dourl{https://www.mdpi.com/1660-4601/17/12/4205},
  \dodoi{10.3390/ijerph17124205} \bibinfo{note}{number: 12}.

\bibitem[{Asensio \emph{et~al.}(2020{\natexlab{b}})Asensio, Pavón, and
  de~Arcas}]{asensio_madrid_2020}
\bibinfo{author}{Asensio, C.}, \bibinfo{author}{Pavón, I.},  and
  \bibinfo{author}{de~Arcas, G.} (\textbf{\bibinfo{year}{2020}}{\natexlab{b}}).
  \enquote{\bibinfo{title}{Changes in noise levels in the city of madrid during
  covid-19 lockdown in 202}} \bibinfo{journal}{The Journal of the Acoustical
  Society of America} \textbf{148}, \bibinfo{pages}{1748--1755},
  \dourl{https://asa.scitation.org/doi/10.1121/10.0002008},
  \dodoi{10.1121/10.0002008}.

\bibitem[{Axelsson \emph{et~al.}(2010)Axelsson, Nilsson, and
  Berglund}]{axelsson_principal_2010}
\bibinfo{author}{Axelsson, O.}, \bibinfo{author}{Nilsson, M.~E.},  and
  \bibinfo{author}{Berglund, B.} (\textbf{\bibinfo{year}{2010}}).
  \enquote{\bibinfo{title}{A principal components model of soundscape
  perception}} \bibinfo{journal}{The Journal of the Acoustical Society of
  America} \textbf{128}(5), \bibinfo{pages}{2836--2846},
  \dourl{http://asa.scitation.org/doi/10.1121/1.3493436},
  \dodoi{10.1121/1.3493436} \bibinfo{note}{number: 5}.

\bibitem[{Bartalucci \emph{et~al.}(2021)Bartalucci, Bellomini, Luzzi, Pulella,
  and Torelli}]{bartalucci_survey_2021}
\bibinfo{author}{Bartalucci, C.}, \bibinfo{author}{Bellomini, R.},
  \bibinfo{author}{Luzzi, S.}, \bibinfo{author}{Pulella, P.},  and
  \bibinfo{author}{Torelli, G.} (\textbf{\bibinfo{year}{2021}}).
  \enquote{\bibinfo{title}{A survey on the soundscape perception before and
  during the {COVID}-19 pandemic in {Italy}}} \bibinfo{journal}{Noise Mapping}
  \textbf{8}(1), \bibinfo{pages}{65--88},
  \dourl{https://www.degruyter.com/document/doi/10.1515/noise-2021-0005/html},
  \dodoi{10.1515/noise-2021-0005} \bibinfo{note}{number: 1}.

\bibitem[{Bonet-Sol\`{a} \emph{et~al.}(2021)Bonet-Sol\`{a},
  Mart\'{\i}nez-Suqu\'{\i}a, Alsina-Pag\`{e}s, and
  Bergad\`{a}}]{bonet-sola_soundscape_2021}
\bibinfo{author}{Bonet-Sol\`{a}, D.},
  \bibinfo{author}{Mart\'{\i}nez-Suqu\'{\i}a, C.},
  \bibinfo{author}{Alsina-Pag\`{e}s, R.~M.},  and \bibinfo{author}{Bergad\`{a},
  P.} (\textbf{\bibinfo{year}{2021}}). \enquote{\bibinfo{title}{The
  {Soundscape} of the {COVID}-19 {Lockdown}: {Barcelona} {Noise} {Monitoring}
  {Network} {Case} {Study}}} \bibinfo{journal}{International Journal of
  Environmental Research and Public Health} \textbf{18}(11),
  \bibinfo{pages}{5799}, \dourl{https://www.mdpi.com/1660-4601/18/11/5799},
  \dodoi{10.3390/ijerph18115799} \bibinfo{note}{number: 11}.

\bibitem[{Erfanian \emph{et~al.}(2020)Erfanian, Mitchell, Aletta, and
  Kang}]{erfanian_psychological_2020}
\bibinfo{author}{Erfanian, M.}, \bibinfo{author}{Mitchell, A.},
  \bibinfo{author}{Aletta, F.},  and \bibinfo{author}{Kang, J.}
  (\textbf{\bibinfo{year}{2020}}). \enquote{\bibinfo{title}{Psychological
  {Well}-being, {Age} and {Gender} can {Mediate} {Soundscapes} {Pleasantness}
  and {Eventfulness}: {A} large sample study}} \bibinfo{type}{preprint} ,
  \dourl{http://biorxiv.org/lookup/doi/10.1101/2020.10.16.341834},
  \dodoi{10.1101/2020.10.16.341834}.

\bibitem[{{European Environment
  Agency}(2020)}]{european_environment_agency_environmental_2020}
\bibinfo{author}{{European Environment Agency}}
  (\textbf{\bibinfo{year}{2020}}). \emph{\bibinfo{title}{Environmental noise in
  {Europe}, 2020.}} (\bibinfo{publisher}{EEA, Publications Office},
  \bibinfo{address}{LU}), \dourl{https://data.europa.eu/doi/10.2800/686249}.

\bibitem[{Garrido-Cumbrera \emph{et~al.}(2021)Garrido-Cumbrera, Foley, Braçe,
  Correa-Fernández, López-Lara, Guzman, Gonzalez~Marín, and
  Hewlett}]{garrido_greenCOVID_2021}
\bibinfo{author}{Garrido-Cumbrera, M.}, \bibinfo{author}{Foley, R.},
  \bibinfo{author}{Braçe, O.}, \bibinfo{author}{Correa-Fernández, J.},
  \bibinfo{author}{López-Lara, E.}, \bibinfo{author}{Guzman, V.},
  \bibinfo{author}{Gonzalez~Marín, A.},  and \bibinfo{author}{Hewlett, D.}
  (\textbf{\bibinfo{year}{2021}}). \enquote{\bibinfo{title}{Perceptions of
  change in the natural environment produced by the first wave of the covid-19
  pandemic across three european countries. results from the greencovid study}}
  \bibinfo{journal}{Urban Forestry \& Urban Greening}
  \dourl{https://linkinghub.elsevier.com/retrieve/pii/S1618866721002879},
  \dodoi{10.1016/j.ufug.2021.127260}.

\bibitem[{Genuit(1996)}]{genuit_objective_1996}
\bibinfo{author}{Genuit, K.} (\textbf{\bibinfo{year}{1996}}).
  \enquote{\bibinfo{title}{Objective evaluation of acoustic quality based on a
  relative approach}} in \emph{\bibinfo{booktitle}{Proceedings-{Institute} of
  {Acoustics}}}, Vol. 18, pp. \bibinfo{pages}{3233--3238}, \bibinfo{note}{iSBN:
  0309-8117}.

\bibitem[{Gontier \emph{et~al.}(2019)Gontier, Lavandier, Aumond, Lagrange, and
  Petiot}]{gontier_estimation_2019}
\bibinfo{author}{Gontier, F.}, \bibinfo{author}{Lavandier, C.},
  \bibinfo{author}{Aumond, P.}, \bibinfo{author}{Lagrange, M.},  and
  \bibinfo{author}{Petiot, J.-F.} (\textbf{\bibinfo{year}{2019}}).
  \enquote{\bibinfo{title}{Estimation of the {Perceived} {Time} of {Presence}
  of {Sources} in {Urban} {Acoustic} {Environments} {Using} {Deep} {Learning}
  {Techniques}}} \bibinfo{journal}{Acta Acustica united with Acustica}
  \textbf{105}(6), \bibinfo{pages}{1053--1066},
  \dourl{https://www.ingentaconnect.com/content/10.3813/AAA.919384},
  \dodoi{10.3813/AAA.919384} \bibinfo{note}{number: 6}.

\bibitem[{Gordo \emph{et~al.}(2021)Gordo, Brotons, Herrando, and
  Gargallo}]{gordo_rapid_2021}
\bibinfo{author}{Gordo, O.}, \bibinfo{author}{Brotons, L.},
  \bibinfo{author}{Herrando, S.},  and \bibinfo{author}{Gargallo, G.}
  (\textbf{\bibinfo{year}{2021}}). \enquote{\bibinfo{title}{Rapid behavioural
  response of urban birds to {COVID}-19 lockdown}}
  \bibinfo{journal}{Proceedings of the Royal Society B: Biological Sciences}
  \textbf{288}(1946), \bibinfo{pages}{20202513},
  \dourl{https://royalsocietypublishing.org/doi/10.1098/rspb.2020.2513},
  \dodoi{10.1098/rspb.2020.2513} \bibinfo{note}{number: 1946}.

\bibitem[{Hadjidemetriou \emph{et~al.}(2020)Hadjidemetriou, Sasidharan,
  Kouyialis, and Parlikad}]{hadjidemetriou_impact_2020}
\bibinfo{author}{Hadjidemetriou, G.~M.}, \bibinfo{author}{Sasidharan, M.},
  \bibinfo{author}{Kouyialis, G.},  and \bibinfo{author}{Parlikad, A.~K.}
  (\textbf{\bibinfo{year}{2020}}). \enquote{\bibinfo{title}{The impact of
  government measures and human mobility trend on {COVID}-19 related deaths in
  the {UK}}} \bibinfo{journal}{Transportation Research Interdisciplinary
  Perspectives} \textbf{6}, \bibinfo{pages}{100167},
  \dourl{https://linkinghub.elsevier.com/retrieve/pii/S2590198220300786},
  \dodoi{10.1016/j.trip.2020.100167}.

\bibitem[{Hornberg \emph{et~al.}(2021)Hornberg, Haselhoff, Lawrence, Fischer,
  Ahmed, Gruehn, and Moebus}]{hornberg_covid_2021}
\bibinfo{author}{Hornberg, J.}, \bibinfo{author}{Haselhoff, T.},
  \bibinfo{author}{Lawrence, B.~T.}, \bibinfo{author}{Fischer, J.~L.},
  \bibinfo{author}{Ahmed, S.}, \bibinfo{author}{Gruehn, D.},  and
  \bibinfo{author}{Moebus, S.} (\textbf{\bibinfo{year}{2021}}).
  \enquote{\bibinfo{title}{Impact of the covid-19 lockdown measures on noise
  levels in urban areas-a pre/during comparison of long - term sound pressure
  measurements in the ruhr area, germany}} \bibinfo{journal}{International
  Journal of Environmental Research and Public Health} \textbf{18},
  \bibinfo{pages}{4653}, \dourl{https://www.mdpi.com/1660-4601/18/9/4653},
  \dodoi{10.3390/ijerph18094653}.

\bibitem[{{ISO
  12913-1:2014}(2014)}]{international_organisation_for_standardization_iso_2014}
\bibinfo{author}{{ISO 12913-1:2014}} (\textbf{\bibinfo{year}{2014}}).
  \plainquote{\bibinfo{title}{{Acoustics} -- {Soundscape} -- {Part} 1:
  {Definition} and conceptual framework}} \bibinfo{note}{{International
  Organization for Standardization, Geneva, Switzerland, 2014}}.

\bibitem[{{ISO/TS
  12913-2:2018}(2018)}]{international_organisation_for_standardization_isots_2018}
\bibinfo{author}{{ISO/TS 12913-2:2018}} (\textbf{\bibinfo{year}{2018}}).
  \plainquote{\bibinfo{title}{{Acoustics} -- {Soundscape} -- {Part} 2: {Data}
  collection and reporting requirements}} \bibinfo{note}{{International
  Organization for Standardization, Geneva, Switzerland, 2018}}.

\bibitem[{{ISO/TS
  12913-3:2019}(2019)}]{international_organisation_for_standardization_isots_2019}
\bibinfo{author}{{ISO/TS 12913-3:2019}} (\textbf{\bibinfo{year}{2019}}).
  \plainquote{\bibinfo{title}{{Acoustics} -- {Soundscape} -- {Part} 3: {Data}
  analysis}} \bibinfo{note}{{International Organization for Standardization,
  Geneva, Switzerland, 2019}}.

\bibitem[{Kajihara \emph{et~al.}(2017)Kajihara, Dozono, and
  Tokui}]{kajihara_imaginary_2017}
\bibinfo{author}{Kajihara, Y.}, \bibinfo{author}{Dozono, S.},  and
  \bibinfo{author}{Tokui, N.} (\textbf{\bibinfo{year}{2017}}).
  \enquote{\bibinfo{title}{Imaginary {Soundscape}: {Cross}-{Modal} {Approach}
  to {Generate} {Pseudo} {Sound} {Environments}}} in
  \emph{\bibinfo{booktitle}{31st {Conference} on {Neural} {Information}
  {Processing} {Systems}}}, \bibinfo{address}{Long Beach, CA, USA},
  \dourl{https://nips2017creativity.github.io/doc/Imaginary\%5FSoundscape.pdf}.

\bibitem[{Kang(2007)}]{kang_urban_2007}
\bibinfo{author}{Kang, J.} (\textbf{\bibinfo{year}{2007}}).
  \emph{\bibinfo{title}{Urban sound environment}} (\bibinfo{publisher}{Taylor
  \& Francis}, \bibinfo{address}{London ; New York}), \bibinfo{note}{oCLC:
  ocm65301511}.

\bibitem[{Kang and Aletta(2018)}]{kang_impact_2018}
\bibinfo{author}{Kang, J.},  and \bibinfo{author}{Aletta, F.}
  (\textbf{\bibinfo{year}{2018}}). \enquote{\bibinfo{title}{The {Impact} and
  {Outreach} of {Soundscape} {Research}}} \bibinfo{journal}{Environments}
  \textbf{5}(5), \bibinfo{pages}{58},
  \dourl{http://www.mdpi.com/2076-3298/5/5/58},
  \dodoi{10.3390/environments5050058}.

\bibitem[{Kuznetsova \emph{et~al.}(2017)Kuznetsova, Brockhoff, and
  Christensen}]{lmertest2017}
\bibinfo{author}{Kuznetsova, A.}, \bibinfo{author}{Brockhoff, P.~B.},  and
  \bibinfo{author}{Christensen, R. H.~B.} (\textbf{\bibinfo{year}{2017}}).
  \enquote{\bibinfo{title}{{lmerTest} package: Tests in linear mixed effects
  models}} \bibinfo{journal}{Journal of Statistical Software} \textbf{82}(13),
  \bibinfo{pages}{1--26}, \dodoi{10.18637/jss.v082.i13}.

\bibitem[{Lee and Jeong(2021)}]{lee_attitudes_2021}
\bibinfo{author}{Lee, P.~J.},  and \bibinfo{author}{Jeong, J.~H.}
  (\textbf{\bibinfo{year}{2021}}). \enquote{\bibinfo{title}{Attitudes towards
  outdoor and neighbour noise during the {COVID}-19 lockdown: {A} case study in
  {London}}} \bibinfo{journal}{Sustainable Cities and Society} \textbf{67},
  \bibinfo{pages}{102768},
  \dourl{https://linkinghub.elsevier.com/retrieve/pii/S2210670721000603},
  \dodoi{10.1016/j.scs.2021.102768}.

\bibitem[{Lenzi \emph{et~al.}(2021)Lenzi, S\'{a}daba, and
  Lindborg}]{lenzi_soundscape_2021}
\bibinfo{author}{Lenzi, S.}, \bibinfo{author}{S\'{a}daba, J.},  and
  \bibinfo{author}{Lindborg, P.} (\textbf{\bibinfo{year}{2021}}).
  \enquote{\bibinfo{title}{Soundscape in {Times} of {Change}: {Case} {Study} of
  a {City} {Neighbourhood} {During} the {COVID}-19 {Lockdown}}}
  \bibinfo{journal}{Frontiers in Psychology} \textbf{12},
  \bibinfo{pages}{570741},
  \dourl{https://www.frontiersin.org/articles/10.3389/fpsyg.2021.570741/full},
  \dodoi{10.3389/fpsyg.2021.570741}.

\bibitem[{Lionello \emph{et~al.}(2020)Lionello, Aletta, and
  Kang}]{lionello_systematic_2020}
\bibinfo{author}{Lionello, M.}, \bibinfo{author}{Aletta, F.},  and
  \bibinfo{author}{Kang, J.} (\textbf{\bibinfo{year}{2020}}).
  \enquote{\bibinfo{title}{A systematic review of prediction models for the
  experience of urban soundscapes}} \bibinfo{journal}{Applied Acoustics}
  \textbf{170}, \bibinfo{pages}{107479},
  \dourl{https://linkinghub.elsevier.com/retrieve/pii/S0003682X20305831},
  \dodoi{10.1016/j.apacoust.2020.107479}.

\bibitem[{Lionello \emph{et~al.}(2021)Lionello, Aletta, Mitchell, and
  Kang}]{lionello_introducing_2021}
\bibinfo{author}{Lionello, M.}, \bibinfo{author}{Aletta, F.},
  \bibinfo{author}{Mitchell, A.},  and \bibinfo{author}{Kang, J.}
  (\textbf{\bibinfo{year}{2021}}). \enquote{\bibinfo{title}{Introducing a
  {Method} for {Intervals} {Correction} on {Multiple} {Likert} {Scales}: {A}
  {Case} {Study} on an {Urban} {Soundscape} {Data} {Collection} {Instrument}}}
  \bibinfo{journal}{Frontiers in Psychology} \textbf{11},
  \bibinfo{pages}{602831},
  \dourl{https://www.frontiersin.org/articles/10.3389/fpsyg.2020.602831/full},
  \dodoi{10.3389/fpsyg.2020.602831}.

\bibitem[{Lüdecke(2021)}]{sjPlot2021}
\bibinfo{author}{Lüdecke, D.} (\textbf{\bibinfo{year}{2021}}).
  \emph{\bibinfo{title}{sjPlot: Data Visualization for Statistics in Social
  Science}}, \dourl{https://CRAN.R-project.org/package=sjPlot},
  \bibinfo{note}{r package version 2.8.9}.

\bibitem[{Maggi \emph{et~al.}(2021)Maggi, Muratore, Gaet\'{a}n, Zalazar-Jaime,
  Evin, P\'{e}rez~Villalobo, and Hinalaf}]{maggi_perception_2021}
\bibinfo{author}{Maggi, A.~L.}, \bibinfo{author}{Muratore, J.},
  \bibinfo{author}{Gaet\'{a}n, S.}, \bibinfo{author}{Zalazar-Jaime, M.~F.},
  \bibinfo{author}{Evin, D.}, \bibinfo{author}{P\'{e}rez~Villalobo, J.},  and
  \bibinfo{author}{Hinalaf, M.} (\textbf{\bibinfo{year}{2021}}).
  \enquote{\bibinfo{title}{Perception of the acoustic environment during
  {COVID}-19 lockdown in {Argentina}}} \bibinfo{journal}{The Journal of the
  Acoustical Society of America} \textbf{149}(6), \bibinfo{pages}{3902--3909},
  \dourl{https://asa.scitation.org/doi/10.1121/10.0005131},
  \dodoi{10.1121/10.0005131} \bibinfo{note}{number: 6}.

\bibitem[{{McKnight} and Najab(2010)}]{weiner_mann-whitney_2010}
\bibinfo{author}{{McKnight}, P.~E.},  and \bibinfo{author}{Najab, J.}
  (\textbf{\bibinfo{year}{2010}}). \enquote{\bibinfo{title}{Mann-whitney u
  test}} in \emph{\bibinfo{booktitle}{The Corsini Encyclopedia of Psychology}},
  edited by \bibinfo{editor}{I.~B. Weiner} and \bibinfo{editor}{W.~E.
  Craighead} (\bibinfo{publisher}{John Wiley \& Sons, Inc.}), p.
  \bibinfo{pages}{corpsy0524},
  \dourl{http://doi.wiley.com/10.1002/9780470479216.corpsy0524},
  \dodoi{10.1002/9780470479216.corpsy0524}.

\bibitem[{Mitchell \emph{et~al.}(2020)Mitchell, Oberman, Aletta, Erfanian,
  Kachlicka, Lionello, and Kang}]{mitchell_soundscape_2020}
\bibinfo{author}{Mitchell, A.}, \bibinfo{author}{Oberman, T.},
  \bibinfo{author}{Aletta, F.}, \bibinfo{author}{Erfanian, M.},
  \bibinfo{author}{Kachlicka, M.}, \bibinfo{author}{Lionello, M.},  and
  \bibinfo{author}{Kang, J.} (\textbf{\bibinfo{year}{2020}}).
  \enquote{\bibinfo{title}{The {Soundscape} {Indices} ({SSID}) {Protocol}: {A}
  {Method} for {Urban} {Soundscape} {Surveys}--{Questionnaires} with
  {Acoustical} and {Contextual} {Information}}} \bibinfo{journal}{Applied
  Sciences} \textbf{10}(7), \bibinfo{pages}{2397},
  \dourl{https://www.mdpi.com/2076-3417/10/7/2397}, \dodoi{10.3390/app10072397}
  \bibinfo{note}{number: 7}.

\bibitem[{Munoz \emph{et~al.}(2020)Munoz, Vincent, Domergue, Gissinger,
  Guillot, Halbwachs, and Janillon}]{munoz_lockdown_2020}
\bibinfo{author}{Munoz, P.}, \bibinfo{author}{Vincent, B.},
  \bibinfo{author}{Domergue, C.}, \bibinfo{author}{Gissinger, V.},
  \bibinfo{author}{Guillot, S.}, \bibinfo{author}{Halbwachs, Y.},  and
  \bibinfo{author}{Janillon, V.} (\textbf{\bibinfo{year}{2020}}).
  \enquote{\bibinfo{title}{Lockdown during {COVID}-19 pandemic: impact on road
  traffic noise and on the perception of sound environment in {France}}}
  \bibinfo{journal}{Noise Mapping} \textbf{7}(1), \bibinfo{pages}{287--302},
  \dourl{https://www.degruyter.com/document/doi/10.1515/noise-2020-0024/html},
  \dodoi{10.1515/noise-2020-0024} \bibinfo{note}{number: 1}.

\bibitem[{Orga \emph{et~al.}(2021)Orga, Mitchell, Freixes, Aletta,
  Alsina-Pag\`{e}s, and Foraster}]{Orga2021Multilevel}
\bibinfo{author}{Orga, F.}, \bibinfo{author}{Mitchell, A.},
  \bibinfo{author}{Freixes, M.}, \bibinfo{author}{Aletta, F.},
  \bibinfo{author}{Alsina-Pag\`{e}s, R.~M.},  and \bibinfo{author}{Foraster,
  M.} (\textbf{\bibinfo{year}{2021}}). \enquote{\bibinfo{title}{Multilevel
  annoyance modelling of short environmental sound recordings}}
  \bibinfo{journal}{Sustainability} \textbf{13}(11),
  \dourl{https://www.mdpi.com/2071-1050/13/11/5779},
  \dodoi{10.3390/su13115779}.

\bibitem[{{R Core Team}(2020)}]{Rstats2020}
\bibinfo{author}{{R Core Team}} (\textbf{\bibinfo{year}{2020}}).
  \emph{\bibinfo{title}{R: A Language and Environment for Statistical
  Computing}}, \bibinfo{organization}{R Foundation for Statistical Computing},
  \bibinfo{address}{Vienna, Austria}, \dourl{https://www.R-project.org/}.

\bibitem[{Radicchi \emph{et~al.}(2021)Radicchi, Cevikayak~Yelmi, Chung, Jordan,
  Stewart, Tsaligopoulos, McCunn, and Grant}]{radicchi_sound_2021}
\bibinfo{author}{Radicchi, A.}, \bibinfo{author}{Cevikayak~Yelmi, P.},
  \bibinfo{author}{Chung, A.}, \bibinfo{author}{Jordan, P.},
  \bibinfo{author}{Stewart, S.}, \bibinfo{author}{Tsaligopoulos, A.},
  \bibinfo{author}{McCunn, L.},  and \bibinfo{author}{Grant, M.}
  (\textbf{\bibinfo{year}{2021}}). \enquote{\bibinfo{title}{Sound and the
  healthy city}} \bibinfo{journal}{Cities \& Health} \textbf{5}(1-2),
  \bibinfo{pages}{1--13},
  \dourl{https://www.tandfonline.com/doi/full/10.1080/23748834.2020.1821980},
  \dodoi{10.1080/23748834.2020.1821980}.

\bibitem[{Ren(2020)}]{ren_pandemic_2020}
\bibinfo{author}{Ren, X.} (\textbf{\bibinfo{year}{2020}}).
  \enquote{\bibinfo{title}{Pandemic and lockdown: a territorial approach to
  {COVID}-19 in {China}, {Italy} and the {United} {States}}}
  \bibinfo{journal}{Eurasian Geography and Economics} \textbf{61}(4-5),
  \bibinfo{pages}{423--434},
  \dourl{https://www.tandfonline.com/doi/full/10.1080/15387216.2020.1762103},
  \dodoi{10.1080/15387216.2020.1762103} \bibinfo{note}{number: 4-5}.

\bibitem[{Rumpler \emph{et~al.}(2021)Rumpler, Venkataraman, and
  G\"{o}ransson}]{rumpler_noise_2021}
\bibinfo{author}{Rumpler, R.}, \bibinfo{author}{Venkataraman, S.},  and
  \bibinfo{author}{G\"{o}ransson, P.} (\textbf{\bibinfo{year}{2021}}).
  \enquote{\bibinfo{title}{Noise measurements as a proxy to evaluating the
  response to recommendations in times of crisis: {An} update analysis of the
  transition to the second wave of the {CoViD}-19 pandemic in {Central}
  {Stockholm}, {Sweden}}} \bibinfo{journal}{The Journal of the Acoustical
  Society of America} \textbf{149}(3), \bibinfo{pages}{1838--1842},
  \dourl{https://asa.scitation.org/doi/10.1121/10.0003778},
  \dodoi{10.1121/10.0003778} \bibinfo{note}{number: 3}.

\bibitem[{Sakagami(2020)}]{sakagami_covid_2020}
\bibinfo{author}{Sakagami, K.} (\textbf{\bibinfo{year}{2020}}).
  \enquote{\bibinfo{title}{How did the 'state of emergency' declaration in
  japan due to the covid-19 pandemic affect the acoustic environment in a
  rather quiet residential area?}} \bibinfo{journal}{UCL Open Environment}
  \textbf{2},
  \dourl{https://ucl.scienceopen.com/document?vid=0dccbf42-eac3-4bda-b3c7-acf6b5dc62ed},
  \dodoi{10.14324/111.444/ucloe.000009}.

\bibitem[{Tarlao \emph{et~al.}(2021)Tarlao, Steffens, and
  Guastavino}]{tarlao_investigating_2021}
\bibinfo{author}{Tarlao, C.}, \bibinfo{author}{Steffens, J.},  and
  \bibinfo{author}{Guastavino, C.} (\textbf{\bibinfo{year}{2021}}).
  \enquote{\bibinfo{title}{Investigating contextual influences on urban
  soundscape evaluations with structural equation modeling}}
  \bibinfo{journal}{Building and Environment} \textbf{188},
  \bibinfo{pages}{107490},
  \dourl{https://linkinghub.elsevier.com/retrieve/pii/S036013232030857X},
  \dodoi{10.1016/j.buildenv.2020.107490}.

\bibitem[{Torresin \emph{et~al.}(2020)Torresin, Albatici, Aletta, Babich,
  Oberman, Siboni, and Kang}]{torresin_indoor_2020}
\bibinfo{author}{Torresin, S.}, \bibinfo{author}{Albatici, R.},
  \bibinfo{author}{Aletta, F.}, \bibinfo{author}{Babich, F.},
  \bibinfo{author}{Oberman, T.}, \bibinfo{author}{Siboni, S.},  and
  \bibinfo{author}{Kang, J.} (\textbf{\bibinfo{year}{2020}}).
  \enquote{\bibinfo{title}{Indoor soundscape assessment: {A} principal
  components model of acoustic perception in residential buildings}}
  \bibinfo{journal}{Building and Environment} \textbf{182},
  \bibinfo{pages}{107152},
  \dourl{https://linkinghub.elsevier.com/retrieve/pii/S0360132320305266},
  \dodoi{10.1016/j.buildenv.2020.107152}.

\bibitem[{Truax(1999)}]{truax_handbook_1999}
\bibinfo{author}{Truax, B.} (\textbf{\bibinfo{year}{1999}}).
  \emph{\bibinfo{title}{Handbook for {Acoustic} {Ecology}}}
  (\bibinfo{publisher}{Cambridge Street Publishing},
  \bibinfo{address}{Cambridge, MA}),
  \dourl{https://www.sfu.ca/sonic-studio-webdav/handbook/index.html}.

\bibitem[{Vida~Manzano \emph{et~al.}(2021)Vida~Manzano, Pastor, Quesada,
  Aletta, Oberman, Mitchell, and Kang}]{manzano_sound_2021}
\bibinfo{author}{Vida~Manzano, J.}, \bibinfo{author}{Pastor, J. A.~A.},
  \bibinfo{author}{Quesada, R.~G.}, \bibinfo{author}{Aletta, F.},
  \bibinfo{author}{Oberman, T.}, \bibinfo{author}{Mitchell, A.},  and
  \bibinfo{author}{Kang, J.} (\textbf{\bibinfo{year}{2021}}).
  \enquote{\bibinfo{title}{The ``sound of silence'' in {Granada} during the
  {COVID}-19 lockdown}} \bibinfo{journal}{Noise Mapping} \textbf{8}(1),
  \bibinfo{pages}{16--31},
  \dourl{https://www.degruyter.com/document/doi/10.1515/noise-2021-0002/html},
  \dodoi{10.1515/noise-2021-0002} \bibinfo{note}{number: 1}.

\bibitem[{Waskom(2021)}]{Waskom2021}
\bibinfo{author}{Waskom, M.~L.} (\textbf{\bibinfo{year}{2021}}).
  \enquote{\bibinfo{title}{seaborn: statistical data visualization}}
  \bibinfo{journal}{Journal of Open Source Software} \textbf{6}(60),
  \bibinfo{pages}{3021}, \dourl{https://doi.org/10.21105/joss.03021},
  \dodoi{10.21105/joss.03021}.

\bibitem[{Woods \emph{et~al.}(2017)Woods, Siegel, Traer, and
  McDermott}]{woods_headphone_2017}
\bibinfo{author}{Woods, K. J.~P.}, \bibinfo{author}{Siegel, M.~H.},
  \bibinfo{author}{Traer, J.},  and \bibinfo{author}{McDermott, J.~H.}
  (\textbf{\bibinfo{year}{2017}}). \enquote{\bibinfo{title}{Headphone screening
  to facilitate web-based auditory experiments}} \bibinfo{journal}{Attention,
  Perception, \& Psychophysics} \textbf{79}(7), \bibinfo{pages}{2064--2072},
  \dourl{http://link.springer.com/10.3758/s13414-017-1361-2},
  \dodoi{10.3758/s13414-017-1361-2} \bibinfo{note}{number: 7}.

\bibitem[{{World Medical Association}(2013)}]{WMA_Helsinki_2013}
\bibinfo{author}{{World Medical Association}} (\textbf{\bibinfo{year}{2013}}).
  \enquote{\bibinfo{title}{{World Medical Association Declaration of Helsinki:
  Ethical Principles for Medical Research Involving Human Subjects}}}
  \bibinfo{journal}{JAMA} \textbf{310}(20), \bibinfo{pages}{2191--2194},
  \dodoi{10.1001/jama.2013.281053}.

\bibitem[{Yang and Kang(2005)}]{Yang2005Acoustic}
\bibinfo{author}{Yang, W.},  and \bibinfo{author}{Kang, J.}
  (\textbf{\bibinfo{year}{2005}}). \enquote{\bibinfo{title}{{Acoustic comfort
  evaluation in urban open public spaces}}} \bibinfo{journal}{Applied
  Acoustics} \textbf{66}(2), \bibinfo{pages}{211--229},
  \dodoi{10.1016/j.apacoust.2004.07.011}.

\bibitem[{Zwicker and Fastl(2007)}]{zwicker_psychoacoustics_2007}
\bibinfo{author}{Zwicker, E.},  and \bibinfo{author}{Fastl, H.}
  (\textbf{\bibinfo{year}{2007}}). \emph{\bibinfo{title}{Psychoacoustics: facts
  and models}}, \bibinfo{edition}{third ed.} ed.
  (\bibinfo{publisher}{Springer}, \bibinfo{address}{Berlin ; New York}),
  \dourl{https://link.springer.com/book/10.1007/978-3-540-68888-4}.

\end{thebibliography}


%Once you have used BibTeX you
%should open the resulting .bbl file and cut and paste the entire contents 
%into the end of your article.
%=======================================================
