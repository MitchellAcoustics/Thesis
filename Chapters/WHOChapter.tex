\chapter{Psychological Well-being and Demographic Factors can Mediate Soundscape Pleasantness and Eventfulness}

\draft{This study will not be copied in wholesale. Instead, the goal is report the results in the context of working out the multi-level regression method and the influence of the non-acoustic factors. Would like to extend it by re-running the analysis with the full database, and controlling for binaural sound level.}

\section*{Context}

 Soundscape studies aim to consider the holistic human perception of a sound environment, including both the physical phenomena and how these are mediated by internal factors.

\section{Introduction}
\draft{Wholesale from Lit review paper}
\draft{From a physiological point of view, a sound signal is a detectable change in our external environment that can elicit a series of unconscious responses, unbalancing our homeostastis (the dynamic state of equilibrium). These responses are similar among populations in a similar situation, are automatic, and regulated by the \gls{sns}. The \gls{psns}, on the contrary, is constantly active to maintain homeostastis. It has long been established that both the \gls{sns} and \gls{psns} are part of the \gls{ans}, which is \emph{per se} a division of the \gls{pns} \cit{8}. Humans and soundscapes have a dynamic bidirectional relationship -- while humans and their behaviour directly influence their soundscape, humans and their behaviour are in turn influenced by their soundscape. Several scientific communities in the area of neuroscience and psychology, therefore, have begun to pay close attention to our day-to-day exposure to particular sounds and their impact on the mental and physical health of individuals. Researchers in the areas of acoustics, environmental psychology, and auditory neuroscience outline the adverse impact of noise or negative sounds on well-being in an attempt to improve modern living standards \cit{20-24}. In this regard, ,evidence indicates that positively perceived sounds (e.g. natural sounds) are tied with a high quality of life and enhanced physiological and physical health \cit{5, 25-27}. Subsequently, \gls{art} argues the impact of nature (e.g. being exposed to natural sounds such as waterfalls) on humans improved cognitive performance and stress recovery \cit{28-31}. Not only has spending time in nature been demonstrated to have positive effects on humans' nervous system but it has also been shown that humans innately tend to seek connections with nature, a hypothesis known as Biophilia \cit{32}. These theories and effects demonstrate the effect external environments have on the health and well-being of individuals, but this relationship is bidirectional within the mind as well. The prior mental and psychological state of the individual will also influence how that individual experiences the environment and can change their perception of it.}

\draft{The physiological and psychological approaches are two sides of the same coin in the realm of soundscape research, strongly interconnected and equally important. The psychological approach, in soundscape research, strives to depict the acoustic environment through the human behavioural pattern by borrowing a more deductive approach. It translates the underlying mechanisms into explicit behavioural manifestations, resulting from the perception of the acoustic environment. The physiological approach investigates the impact of the acoustic environment through the investigation of fundamental mechanisms of \gls{cns} and \gls{pns} by adopting a more inductive inferential approach. The physiological approach delineates the causation of the particular behaviour evoked by the environmental sounds.}

\draft{The soundscape is composed of three main components -- human interaction, acoustic environment, and perception -- so it potentially draws attention across several life-science disciplines such as environmental psychology and public health, psychophysiology, and auditory neuroscience. To the best of the author's knowledge, no work previous to this review has highlighted the explicit psychophysiological underpinnings of the soundscape. The review in this chapter is the first work that reflects the fundamental mechanisms of the soundscape rather than its behavioural expressions.}

\subsection{Psychophysiological studies}




\section{Discussion}
 \subsection{Incorporating personal factors into a predictive soundscape model}
Although, as \citet{Droumeva2021sound} points out, each individual brings their own cultural and subjective aspects of listening to the stage of urban sound, when attempting to characterise the soundscape of a space, it is not a particular individual's aspects we should be concerned with. That individual forms a part of the collective perception of the space. Their cultural and subjective (i.e. personal) aspects mitigate their perception, but this perception then forms only a single component of the collective perception. How then should we consider these personal factors? Surely there is no suggestion to disregard their influence and importance within the soundscape approach? In my view, there are two approaches:

\begin{enumerate}
  \item Incorporate these personal factors as demographic statistics of a location; or
  \item An agent-based approach where each individual likely to use the space is simulated and modelled with their personal factors to then be included in the collective perception.
\end{enumerate}

Let's look at how these two approaches would be implemented in a multilevel acoustics-based predictive model, such as those presented in \crefrange{ch:mlmann}{ch:lockdown}.

\subsubsection{Approach 1}
In the first, the demographic breakdown of the space under investigation is estimated, either through a census or by the designers' desired use case. This demographic breakdown can then be compared to the results presented above \citep{Erfanian2021Psychological} to derive weighting factors which adjust the predicted soundscape assessment. For instance, the results suggest that retired persons perceive the soundscape as \draft{XX\% or amount [need to check with results]} less pleasant than others. If the particular space under investigation has a large proportion of retired persons, say 65\% we could then apply an adjustment to the initial personal-factors-agnostic prediction to reflect this tendency. In this example, an initial location-level \gls{isopl} prediction of 0.36, with a 65\% retired population would be corrected by \draft{-XX [0.65 x result]} for a final demographics-corrected \gls{isopl} prediction of \draft{XX}.

\subsubsection{Approach 2}

\subsubsection{Benefits and downsides of each approach}