\chapter{Psychological Well-being and Demographic Factors can Mediate Soundscape Pleasantness and Eventfulness}
\label{ch:whostudy}

\draft{This study will not be copied in wholesale. Instead, the goal is report the results in the context of working out the multi-level regression method and the influence of the non-acoustic factors. Would like to extend it by re-running the analysis with the full database, and controlling for binaural sound level.}

\section*{Context}

 Soundscape studies aim to consider the holistic human perception of a sound environment, including both the physical phenomena and how these are mediated by internal factors.

\section{Introduction}
\copied{From a physiological point of view, a sound signal is a detectable change in our external environment that can elicit a series of unconscious responses, unbalancing our homeostastis (the dynamic state of equilibrium). These responses are similar among populations in a similar situation, are automatic, and regulated by the \gls{sns}. The \gls{psns}, on the contrary, is constantly active to maintain homeostastis. It has long been established that both the \gls{sns} and \gls{psns} are part of the \gls{ans}, which is \emph{per se} a division of the \gls{pns} \cit{8}. Humans and soundscapes have a dynamic bidirectional relationship -- while humans and their behaviour directly influence their soundscape, humans and their behaviour are in turn influenced by their soundscape. Several scientific communities in the area of neuroscience and psychology, therefore, have begun to pay close attention to our day-to-day exposure to particular sounds and their impact on the mental and physical health of individuals. Researchers in the areas of acoustics, environmental psychology, and auditory neuroscience outline the adverse impact of noise or negative sounds on well-being in an attempt to improve modern living standards \cit{20-24}. In this regard, ,evidence indicates that positively perceived sounds (e.g. natural sounds) are tied with a high quality of life and enhanced physiological and physical health \cit{5, 25-27}. Subsequently, \gls{art} argues the impact of nature (e.g. being exposed to natural sounds such as waterfalls) on humans improved cognitive performance and stress recovery \cit{28-31}. Not only has spending time in nature been demonstrated to have positive effects on humans' nervous system but it has also been shown that humans innately tend to seek connections with nature, a hypothesis known as Biophilia \cit{32}. These theories and effects demonstrate the effect external environments have on the health and well-being of individuals, but this relationship is bidirectional within the mind as well. The prior mental and psychological state of the individual will also influence how that individual experiences the environment and can change their perception of it.}

\copied{The physiological and psychological approaches are two sides of the same coin in the realm of soundscape research, strongly interconnected and equally important. The psychological approach, in soundscape research, strives to depict the acoustic environment through the human behavioural pattern by borrowing a more deductive approach. It translates the underlying mechanisms into explicit behavioural manifestations, resulting from the perception of the acoustic environment. The physiological approach investigates the impact of the acoustic environment through the investigation of fundamental mechanisms of \gls{cns} and \gls{pns} by adopting a more inductive inferential approach. The physiological approach delineates the causation of the particular behaviour evoked by the environmental sounds.}

\copied{The soundscape is composed of three main components -- human interaction, acoustic environment, and perception -- so it potentially draws attention across several life-science disciplines such as environmental psychology and public health, psychophysiology, and auditory neuroscience. To the best of the author's knowledge, no work previous to this review has highlighted the explicit psychophysiological underpinnings of the soundscape. The review in this chapter is the first work that reflects the fundamental mechanisms of the soundscape rather than its behavioural expressions.}

\subsection{Psychophysiological studies}

\section[Methods]{Methods\footnote{This section closely resembles the original Methods section in \citep{Erfanian2021Psychological} of which I was the second author. I contributed significantly to the drafting of the original paper and the content presented here directly informs the analysis presented later.}}

The study was approved by the local ethics committee of University College London (UCL), BSEER, Institute for Environmental Design \& Engineering (IEDE) (Dated 11-10-2019).

\subsection{Data Collection}
This study made use of a subset of data from the \gls{isd}. This study was conducted and published during the first round of \gls{ssid} data collection, prior to the first publication of the ISD. It includes 11 locations in London, with data collected from general members of the public. This study made use of the same \gls{ssid} questionnaire presented in full in \cref{app:questionnaire}, which is an adapted version of \citet{ISO12913Part2}\footnote{The ISO/TS 12913-2:2018 specifies requirements and provides supporting information on data collection and reporting for soundscape studies, investigations, and applications.} Method `A' (urban soundwalk method) and the WHO-5 Well-being Index \citep{Hall2011Examining}, as well as demographic information. As this study focusses on the items related to psychological well-being, demographics, and personal factors, we used a subset of the variables available in the full ISD. Only the sections of the questionnaire which were examined within this study are reported in this chapter. \cref{tab:whoDemo} reports the demographic characteristics of the sample used.


\begin{table}[!ht]
  %TODO: Check the discrepancy in percentages between Employed and White - White has higher number but lower percentage?
\centering
\caption{The sample demographic characteristics \label{tab:whoDemo}}
% \resizebox{\textwidth}{!}{%
\begin{tabular}{@{}ll@{}}
\toprule
Demographic characteristics                 & N(\%)                            \\ \midrule
Total Samples                               & N = 1134                         \\
\textbf{Gender}                             &                                  \\
\quad Female                                & 610 (53.79)                      \\
\quad Male                                  & 524 (46.2)                       \\
\textbf{Age}                                &                                  \\
\quad Mean                                  & 34.67 years $\pm$ 15.11          \\
\quad 18-30                                 & 627 (55.29)                      \\
\quad 31-40                                 & 195 (17.19)                      \\
\quad 41-50                                 & 112 (9.87)                       \\
\quad 51-60                                 & 97  (8.55)                       \\
\quad 61-70                                 & 72  (6.34)                       \\
\quad 71+                                   & 31  (2.73)                       \\
\textbf{Educational Level}                  &                                  \\
\quad Some high school                      & 22  (1.2)                        \\
\quad High school graduate                  & 315 (17.3)                       \\
\quad Trade/technical/vocational training   & 51  (2.8)                        \\
\quad University (undergraduate/bachelor)   & 422 (32.1)                       \\
\quad Postgraduate degree (master)          & 324 (17.8)                       \\
\textbf{Occupation Status}                  &                                  \\
\quad Employed                              & 613 (54.05)                      \\ 
\quad Unemployed                            & 25  (2.2)                        \\
\quad Retired                               & 84  (7.4)                        \\
\quad Student                               & 348 (30.6)                       \\
\quad Employed-Student                      & 5   (0.4)                        \\
\quad Other                                 & 44  (3.8)                        \\
\quad Rather not say                        & 15  (1.3)                        \\
\textbf{Ethnicity}                          &                                  \\
\quad White                                 & 806 (71.08)                      \\
\quad Mixed/Multiple ethnic groups          & 63  (3.5)                        \\
\quad Asian/Asian British                   & 156 (8.6)                        \\
\quad Black/African/Caribbean/Black British & 31  (1.7)                        \\
\quad Middle Eastern                        & 23  (1.3)                        \\
\quad Rather not say                        & 55  (3)                          \\
\bottomrule
\end{tabular}%
% }
\end{table}

\subsubsection*{Perceived affective quality / Perceptual attributes}

\copied{The \glsfirst{paq} of the sound environment as adopted in Method `A', described in \citet{ISO12913Part2}, consists of category scales containing five response categories, based on the \gls{ssqp} \cite{Axelsson2012Swedish}. It includes a question `to what extent they agree/disagree that the present surrounding sound environment is \ldots'. The participants judged the quality of the acoustic environment by 8 adjectives: pleasant, chaotic, vibrant, uneventful, calm, annoying, eventful, or monotonous. The answers were presented in a 5-point Likert scale ranging from `strongly disagree = 1' to `strongly agree = 5'. The perceptual attributes measure as a unidimensional measuring tool for the perception of the acoustic environment has not been validated to this date. The \glspl{paq} were utilised as aggregated values to construct the principal components of the soundscape (\gls{isopl} and \gls{isoev}) (see \cref{sec:whoOutcomeVar}).}

In order to maintain data quality and exclude cases where respondents either clearly did not understand the \gls{paq} adjectives or intentionally misrepresented their answers, surveys for which the same response was given for every \gls{paq} (e.g. `Strongly agree' to all 8 attributes) were excluded. This is justified as no reasonable respondent who understood the questions would answer that they `strongly agree' that a soundscape is pleasant and annoying, calm and chaotic, etc. Cases where respondents answered `Neutral' to all \glspl{paq} are not excluded in this way, as a neutral response to all attributes is not necessarily contradictory. In addition, surveys were discarded as incomplete if more than 50\% of the \gls{paq} and sound source questions were not completed. 

\subsubsection*{Psychological well-being/WHO-5 well-being index}

\copied{The \glsfirst{who5} asks how individuals have been feeling over the last two weeks such as `I have felt cheerful and in good spirits'. The \gls{who5} has been designed for multiple research and clinical purposes, covering a wide range of mental health domains, namely perinatal mental health, geriatrics mental health, endocrinology, clinical psychometrics, and psychiatric disorders screening.}

\copied{The \gls{who5} is known to be one of the most valid generic scales for quantification of general well-being. In terms of the construct validity of the scale, \gls{who5} shoed to have properties that are a coherent measure of well-being \citep{Topp2015WHO}. With regards to relevant literature, \gls{who5} confirmed that all items constitute an integrated scale in which items add up related information about the level of general pscyhological well-being among both youngsters and elderlies \cit{Blom2012, Lucas-Carrasco2012}. For the purpose of analysis, a composite \gls{who5} score is calculated by summing the responses to each of the 5 questions (coded from 0 (for `at no time') to 5 (for `all of the time')), then multiplying by 4 to get a single score which ranges from 0 (the lowest level of well-being) to 100 ( the highest level of well-being) \citep{Topp2015WHO}.}

\subsubsection*{Demographic characteristics}

\copied{Demographic characteristics were presented such as age, gender (male, female), education level (some high school, high school, trade/technical/vocational training, university, and postgraduate), occupational status (employed, unemployed, retired, student, employed-student, other, and rather not say), and ethnicity (Asian, Black/Caribbean, Middle Eastern, White, and Mixed). Some blank spaces were provided if the participant wished to add further information. At the end of the survey, participants had the opportunity to write down any additional questions or remarks and were thanked for their participation.}

\subsubsection*{Outcome variables (\gls{isopl} and \gls{isoev})}

The soundscape data were analysed according to the procedure laid out in Part 3 of the ISO 12913 \footnote{The ISO/TS 12913-3:2019 provides requirements and supporting information on analysis of data collected \emph{in-situ}.} standard series. In order to ease data analysis and modelling, the standard suggests a method to collapse the \gls{paq} responses for each of the 8 \glspl{pa} down to a 2-dimensional coordinate scatter plot with continuous values for `Pleasantness' on the X-axis (referred to as \gls{isopl}) and `Eventfulness' on the Y-axis (referred to as \gls{isoev}). These coordinates are then normalized to between -1 and 1 (per the recommendation of \citet{ISO12913Part3}).

\subsubsection*{Survey procedure}

The data was collected according to the \gls{ssid} protocol outlined in \cref{chap:protocol}. The goal of the researchers on site was to collect a minimum of one hundred questionnaires from each selected site/location, which was typically achieved over a period of 2-3 days each consisting of approximately a 4 hour session. In some cases, either due to extenuating circumstances, time constraints, or excluded surveys, the full one hundred surveys were not achieved. The data for this study were collected from \nth{28} February 2019 to \nth{18} October 2019 between 11 a.m. and 3 p.m.

In line with the \gls{ssid} protocol, during the survey period, acoustic and environmental metrics were simultaneously collected through binaural recordings, a calibrated \gls{slm}, and an environmental meter which recorded temperature, lighting level, and humidity data. The \gls{slm} was set up in the space in which the questionnaires were conducted (i.e. the \gls{environmental-unit}) and left running for the full duration of the survey in order to characterise the acoustic environment. The environmental metrics were not reported in this study since they were not in the scope of this paper but are included in \cref{app:location-data} in order to provide context for the interested readers. 

\subsection{Data analysis strategy}

\subsubsection*{Missing data, checking for outliers, and data scaling}

Prior to the data analysis, I imputed missing data and the imputed data was used across all analyses. Missing education values were imputed with the mode value (University). Missing values for age were imputed with the median age value (29). \gls{who5} (psychological well-being) missing values were imputed with the median value (64). I excluded those who responded `non-conforming' (N=4) or `decline' (N=21) for gender, due to the very small sample size and to simplify the effects of gender in the model. The initial data sample size was N=1467; the data included in the analysis N=1134.

I took a lenient approach to outliers. Due to the nature of survey data, it was typically inappropriate to remove data solely because it represented a deviation from the typical response. However, I wanted to catch data which was incorrect, intentionally wrong, or a typo and then remove them. For the most part, this was handled with our data quality method implemented in REDCap, to ensure the \gls{ssqp}/perceptual attribute values (N=8) were filled in such that they complied with the circumplex theory to a minimum degree. We were therefore, only looking for values which were extreme outliers or impossible.

\subsubsection*{Correlation between predictors and output variables}
To establish the linearity between all pairs of variables including the predictors and outcome variables, the Pearson correlation coefficient, \gls{anova}, and Chi-square were performed between psychological well-being, age, gender, ethnicity, education level, occupation status, and the circumplex coordinate values (\gls{isopl} and \gls{isoev}). These results are given in \cref{tab:whoCorr}.

\subsection{Model specification (linear mixed-effects modelling)}

\glsfirst{lmer} with random intercept and fixed slope, using backward stepwise feature selection was utilised to (a) identify the association of our features of interest (FOIs) including psychological well-being, age, gender, education level, ethnicity, occupation status, and their interaction terms with \gls{isopl} and \gls{isoev} and (b) accommodate associations within participants among locations. In order to account for latent differences in the pleasantness and eventfulness ratings of various locations, the intercepts of each model are allowed to vary as a function of the location. Therefore, the model is constructed with two levels -- the individual level (the random effects) and the location level (the fixed effects). Separate models were constructed for each \gls{isopl} and \gls{isoev} and take the form:
%
\begin{equation}
  \label{eqn:whoPl}
  ISOPleasant_{ij} = \beta_{0j} + \beta_1 x_{1ij} + \beta_2 x_{2ij} + \ldots + \beta_n x_{nij} + \epsilon_{ij}
\end{equation}
%
\begin{equation}
  \label{eqn:whoEv}
  ISOEventful_{ij} = \beta_{0j} + \beta_1 x_{1ij} + \beta_2 x_{2ij} + \ldots + \beta_n x_{nij} + \epsilon_{ij}
\end{equation}
%
where $ISOPleasant_{ij}$ or $ISOEventful_{ij}$ are the dependent variable value for individual $i$ in Location $j$; $\beta_{0j}$ is the intercept for Location $j$; $\beta_1$ through $\beta_n$ are the slopes relating the independent variables $x_1$ through $x_n$ to the dependent variable; $x_{1ij}$ through $x_{nij}$ are the dependent variables for individual $i$ in Location $j$; $\epsilon_{ij}$ is the random error for individual $i$ in Location $j$. In turn, $\beta_{0j}$ can be expressed as:
%
\begin{equation}
  \beta_{0j} = \gamma_{00} + U_{0j}
\end{equation}
%
where $\gamma_{00}$ is the mean intercept across Locations; and $U_{0j}$ is the unique effect of Location $j$ on the intercept. In a random intercept model, the slope coefficients ($B_n$) are considered fixed across the locations (hence, labelled as the fixed effects) indicating that the relationship between the dependent variable (e.g. age, gender, etc.) and the independent variable (\gls{isopl} or \gls{isoev}) is the same for all locations, while the general \gls{isopl} of the location is accounted for by the varying intercept.

In order to identify the significant FOIs within the multi-level structure, we employed a stepwise feature selection on the fixed effects portion of the mixed-effects model, with an inclusion threshold of $p < 0.05$. Since this model includes only the LocationID at the random effects level, only the fixed effects are reduced in the feature selection process. To check for multicollinearity among the selected features, the \glsfirst{vif} was calculated and a threshold of $VIF < 5$ was set. Any features which remained after the backwards stepwise selection which exceeded this threshold were investigated and removed if they were highly collinear with the other features. Once the feature selection process is completed, the final model with only significant FOIs included is fit and the table of the model coefficients is printed along with plots of the random effects and z-scaled and non-standardised estimates terms. 

The model fitting and feature selection was performed using `lme4' (version 1.1) and the `step' function from `lmerTest' (version 3.1.3) \cit{Kuznetsova, Brokhoff, Christensen, 2017} in R statistical software (version 4.0.3) \cit{R core team, 2013}. The summaries and plots were created using the `sjPlot' package (version 2.8.6) \cit{Ludecke, 2018}. 

\section{Results}

The setup and procedures %TODO: Return here

\section{Discussion}
 \subsection{Incorporating personal factors into a predictive soundscape model}
Although, as \citet{Droumeva2021sound} points out, each individual brings their own cultural and subjective aspects of listening to the stage of urban sound, when attempting to characterise the soundscape of a space, it is not a particular individual's aspects we should be concerned with. That individual forms a part of the collective perception of the space. Their cultural and subjective (i.e. personal) aspects mitigate their perception, but this perception then forms only a single component of the collective perception. How then should we consider these personal factors? Surely there is no suggestion to disregard their influence and importance within the soundscape approach? In my view, there are two approaches:

\begin{enumerate}
  \item Incorporate these personal factors as demographic statistics of a location; or
  \item An agent-based approach where each individual likely to use the space is simulated and modelled with their personal factors to then be included in the collective perception.
\end{enumerate}

Let's look at how these two approaches would be implemented in a multilevel acoustics-based predictive model, such as those presented in \crefrange{ch:mlmann}{ch:lockdown}.

\subsubsection{Approach 1}
In the first, the demographic breakdown of the space under investigation is estimated, either through a census or by the designers' desired use case. This demographic breakdown can then be compared to the results presented above \citep{Erfanian2021Psychological} to derive weighting factors which adjust the predicted soundscape assessment. For instance, the results suggest that retired persons perceive the soundscape as \draft{XX\% or amount [need to check with results]} less pleasant than others. If the particular space under investigation has a large proportion of retired persons, say 65\% we could then apply an adjustment to the initial personal-factors-agnostic prediction to reflect this tendency. In this example, an initial location-level \gls{isopl} prediction of 0.36, with a 65\% retired population would be corrected by \draft{-XX [0.65 x result]} for a final demographics-corrected \gls{isopl} prediction of \draft{XX}.

\subsubsection{Approach 2}

\subsubsection{Benefits and downsides of each approach}