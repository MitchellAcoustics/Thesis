\chapter[Study I: Personal Factors]{Study I: Psychological Well-being and Demographic Factors can Mediate Soundscape Pleasantness and Eventfulness: A large sample study}

\draft{This study will not be copied in wholesale. Instead, the goal is report the results in the context of working out the multi-level regression method and the influence of the non-acoustic factors. Would like to extend it by re-running the analysis with the full database, and controlling for binaural sound level.}

\section*{Abstract}

 Soundscape studies aim to consider the holistic human perception of a sound environment, including both the physical phenomena and how these are mediated by internal factors.


\section{Discussion}
 \subsection{Incorporating personal factors into a predictive soundscape model}
Although, as \citet{Droumeva2021sound} points out, each individual brings their own cultural and subjective aspects of listening to the stage of urban sound, when attempting to characterise the soundscape of a space, it is not a particular individual's aspects we should be concerned with. That individual forms a part of the collective perception of the space. Their cultural and subjective (i.e. personal) aspects mitigate their perception, but this perception then forms only a single component of the collective perception. How then should we consider these personal factors? Surely there is no suggestion to disregard their influence and importance within the soundscape approach? In my view, there are two approaches:

\begin{enumerate}
  \item Incorporate these personal factors as demographic statistics of a location; or
  \item An agent-based approach where each individual likely to use the space is simulated and modelled with their personal factors to then be included in the collective perception.
\end{enumerate}

Let's look at how these two approaches would be implemented in a multilevel acoustics-based predictive model, such as those presented in \crefrange{ch:mlmann}{ch:lockdown}.

\subsubsection{Approach 1}
In the first, the demographic breakdown of the space under investigation is estimated, either through a census or by the designers' desired use case. This demographic breakdown can then be compared to the results presented above \citep{Erfanian2021Psychological} to derive weighting factors which adjust the predicted soundscape assessment. For instance, the results suggest that retired persons perceive the soundscape as \draft{XX\% or amount [need to check with results]} less pleasant than others. If the particular space under investigation has a large proportion of retired persons, say 65\% we could then apply an adjustment to the initial personal-factors-agnostic prediction to reflect this tendency. In this example, an initial location-level \gls{isopl} prediction of 0.36, with a 65\% retired population would be corrected by \draft{-XX [0.65 x result]} for a final demographics-corrected \gls{isopl} prediction of \draft{XX}.

\subsubsection{Approach 2}

\subsubsection{Benefits and downsides of each approach}