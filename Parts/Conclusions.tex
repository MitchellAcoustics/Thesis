\chapter{Conclusions}
\label{ch:conc}

\section{General Discussion}
\begin{itemize}
  \item Soundscape studies have been focussed for too long on the retrospective post-hoc evaluation of a space.
  
\end{itemize}

Retrospective assessment methods also struggle to capture the dynamics of the soundscape in a space. Whether through the narrative interview method of \draft{section of ISO12913-2}, through soundwalks, or through in-situ questionnaires \citep{Mitchell2020Soundscape}, only the soundscape during the particular period which the researchers are actively investigating is captured. This makes it very difficult to determine diurnal, seasonal, or yearly patterns of the soundscape. These patterns may be driven by corresponding diurnal, seasonal, or yearly patterns in the acoustic or visual environment, or by variations in how people process and respond to the sound at different times of day/season/year. Currently the only way to investigate any of these patterns is through repeated surveys. Predictive modelling, on the other hand, could allow a trained soundscape model to be paired with longterm monitoring methods to track how a soundscape may change in response to changes in the acoustic environment.

Admittedly, this method would not be able to answer the second part of the question - how do people's responses to a given acoustic and visual environment change throughout the various daily/seasonal/yearly periods? \draft{This part should maybe be moved to a discussion}One approach to answering this question which has not, as far as the author is aware, been employed is through an un-attended survey method. Such a method could involve creating and posting fliers asking users of a space to complete a soundscape survey (accessed through a QR code) and leaving these fliers installed for longer periods of time. It is unclear how successful such a general approach would be, in particular what response rate would be expected, but given the increasing familiarity with QR codes among the general public following their use for track-and-trace during COVID-19, it does appear promising. These un-attended surveys could also be paired with long-term acoustic and environmental monitoring via a WASN or powered SLM which could simultaneously track the acoustic environment. This would thus result in a time series of online soundscape questionnaires with a corresponding time series of acoustic and environmental information, allowing us to track the changes of each over long periods of time.

\begin{itemize}
  \item Soundscape has also been focussed on the local / individual scale, whereas assessment and legislation need data at the city-scale.
  \item Society ( and engineers) are interested in possibilities, in designing and improving future spaces
  \item Because of this limited view, the methods available in soundscape studies are unsuitable for these challenges.
  \item If noise control wantss to progress beyond
  \item Psychoacoustics alone is not enough to model soundscape perception
  \item To be useful, Predictive models can't include perceptual inputs, this would be recursive and self-defeating.
\end{itemize}

Predictive soundscape modelling thus provides a possibility for a more holistic approach to large scale urban sound investigations. Studies from outside of soundscape have demonstrated that a user's perception of a space is a much better predictor of how the use it -- and of the benefits they derive from it -- than the strict physical characteristics of the space \citep{Kruize2019Exploring}. It thus stands that a soundscape approach focussed on perception which can be generalised across a city-scale -- rather than in isolated spaces -- could provide more reliable metrics with which to investigate the health, social, and psychological effects of sound.

The empirical and modelling work in this thesis represents a key step towards realising this application to soundscape mapping. When the predictive modelling approach is paired with data from, e.g. a large-scale acoustic sensor network, it could be used to produce a dynamic map of the likely perception of spaces across a city. Alternatively, 
%\cit{https://reader.elsevier.com/reader/sd/pii/S0003682X17311283?token=75D2B5B9AB562DC1501F4FD072E35205CA61B84786D2ABD6C0FA79E7D02E508B980E264AB272553D5863178E2DB9B9AF&originRegion=eu-west-1&originCreation=20220523164448}


\cref{ch:lockdown} demonstrated the human-level impact of a drastic change in urban transport. As a result of the \gls{covid19} lockdowns, an ideal implementation of noise reduction efforts was achieved through drastic reductions in traffic flows and substantial reductions in transport and delivery activity. Our question was then whether these changes in the common sources of urban noise actually resulted in the desired noise reduction and whether these noise reductions would have achieved an improved in the perceived soundscape of urban spaces. In the first case, how effective these traffic reductions were at reducing sound levels was heavily dependent on the type of space, although a general reduction was seen. However, the predictive modelling demonstrated that even large reductions in traffic noise levels at sites like Camden Town and Euston Tap were not enough to make those sites truly `pleasant', when considered from a holistic soundscape perspective. In addition, the transport reductions seen under \gls{covid19} resulted in negative impacts to other highly pleasant soundscapes, where the reduced traffic and human sounds resulted in less pleasant soundscapes. 

If noise control engineering and urban design want to progress beyond a singular focus on reducing sound levels, it needs tools which can 
Predictive soundscape modelling can 

\section{Contribution to Knowledge}
* Properly laying out a framework 

\section{Implications}

\section{Limitations}
\paragraph*{Qualitative / Community approach}
\draft{An approach rooted in the qualitative and sociological relationships between people and their soundscapes. Focus on Sarah Payne and Edda Bild's work. }

Several criticisms of the sorts of questionnaire-based approaches highlighted in \citet{ISO12913Part2} and used throughout this thesis have been raised. \citet{Bild2018Public} notes

\begin{quote}
  [\dots] the questionnaires used as tools to gain insight on users’ soundscape evaluations mostly employ categorical-based assessments and rarely include openended questions [\ldots] thus representing a limited understanding of users’ soundscape evaluations. Finally, these methods minimize or do not adequately account for the role of moderating factors, like activity, in influencing how people evaluate what they hear, despite increasing evidence on activity as a moderating activity for users' soundscapes.
\end{quote}

In contrast, \citet{Bild2018Public} employs a mixed-methods approach which includes both `reported' (i.e. questionnaire-based) and `enacted' soundscape evaluations. Enacted evaluations are assessed by observing how people actually use the space under investigation.  



\section{Recommendations for Future Research}

\section{Concluding Remarks}

Where previous ground-breaking strategies toward practical urban soundscape design \citep{Lacey2019Noise}, have been limited in their scope, providing methods of improving individual soundscapes or approaches which can be applied to bespoke projects, this work aims to move towards a generalised and widely applicable engineering-based approach. The goal is to promote a soundscape mindset as the 'standard', not just as an extra add-on for forward-thinking projects or as a localised sonic rupture which, while incredibly effective (and affective) within its radius, is not suited to being applied on a city- or national-policy scale. For this purpose, we require a standardised and implementable index and direction of best practice which can be implemented by trained technicians, engineers, designers, and planners across all aspects of urban design, from the billion dollar museum to the inner-city public elementary school. A desire for good and restorative soundscapes should be the baseline standard in a city's design, upon which art which highlights the 'mythic, imaginative and poetic relationships within the affective environments' \citep{Lacey2019Noise} can be implemented by the specialists. The goal of this work therefore, is not to critique or counter the creative approaches taken by those within sound art or acoustic ecology, but instead to move towards a new baseline, a new way of designing all environments of the city, from the lowest to the highest (but mostly at the lowest, where it is needed most).
