\chapter{Conclusions}
\label{ch:conc}

\section{General Discussion}
\begin{itemize}
  \item Soundscape studies have been focussed for too long on the retrospective post-hoc evaluation of a space.
  \item Soundscape has also been focussed on the local / individual scale, whereas assessment and legislation need data at the city-scale.
  \item Society ( and engineers) are interested in possibilities, in designing and improving future spaces
  \item Because of this limited view, the methods available in soundscape studies are unsuitable for these challenges.
  \item If noise control wantss to progress beyond
  \item Psychoacoustics alone is not enough to model soundscape perception
  \item To be useful, Predictive models can't include perceptual inputs, this would be recursive and self-defeating.
\end{itemize}


\section{Contribution to Knowledge}
* Properly laying out a framework 

\section{Implications}

\section{Limitations}

\section{Recommendations for Future Research}

\section{Concluding Remarks}