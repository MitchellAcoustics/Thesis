\chapter{Conclusions}
\label{ch:conc}

Urban soundscape studies have progressed a great deal over the last two decades. The soundscape approach has begun to make impacts in policy, such as with the Welsh \citet{Welsh2018Noise}, and has achieved increased international recognition as a key tool in creating positive and restorative urban environments, as in the UN Environment Programme's \emph{Frontiers 2022: Noise, Blazes and Mismatches} report \citep{Aletta2022Frontiers}. Their relevance for the planning and design of urban spaces is now generally acknowledged by both the academic and practitioners' communities. However, this increased attention has highlighted some shortcomings with the tools currently available in soundscape. 

Soundscape studies have been focussed for too long on retrospective post-hoc evaluations and on the individual or small group scale. The true scale of the impact of noise on the health of populations has been highlighted following the creation of urban noise maps and the proliferation of improved monitoring technology. If soundscape is to be effectively brought into assessment and legislation, data will be needed at the city scale.  Predictive soundscape modelling thus provides a possibility for a more holistic approach to large scale urban sound investigations. Studies from outside of soundscape have demonstrated that a user's perception of a space is a much better predictor of how they use the space -- and of the benefits they derive from it -- than the strict physical characteristics of the space \citep{Kruize2019Exploring}. It thus stands that a soundscape approach focussed on perception which can be generalised across a city-scale -- rather than in isolated spaces -- could provide more reliable metrics with which to investigate the health, social, and psychological effects of sound.

Society, designers, and engineers are interested in possibilities, in designing and improving future spaces. To bring soundscape into these fields, an approach which enables designers to test iterative design possibilities, score the effectiveness of their design, and identify new solutions is necessary. Likewise, if large-scale noise assessment is to move beyond a myopic focus on sound level, soundscape must provide methods which can be used in similar scenarios and applications as the tools available in noise modelling. As put by Francesco Aletta:

\begin{samepage}
\begin{quote}
  [\ldots]assessing an acoustic environment solely as loud or quiet is like judging a soup only by its temperature. Of course, if it's too hot, you need to know, but if you want to think about spices, flavor, you need a different approach.
  \begin{flushright}
    \citet{OSullivan2021Pandemic}
  \end{flushright}
\end{quote}
\end{samepage}

By integrating the impact of other sonic features like tonality and impulsiveness in terms of decibel adjustments, it is as if we are discussing how salty or spicy a soup is in terms of \textdegree C penalties. Likewise, even psychoacoustic metrics provide a one-sided view of sound perception. For the most part, these metrics were designed to characterise various negative qualities of the sound. Certainly, they therefore have a negative correlation with positive assessments of the sound, but the simple fact is that they were conceived of and implemented in an attempt to quantify some sonic characteristic that was found to contribute to a negative perception. Hence why in Zwicker's empirical formula for Psychoacoustic Annoyance (\cref{eqn:pa1}), all of the constituent parts have positive coefficients. While this would not theoretically hinder a formula for describing positive aspects of the sound, it creates a sort of conceptual barrier. If all of these metrics are designed to capture negative aspects of the sound, then it is insufficient to use them create a formula to describe a positive sound, since that formula would only represent the `absence of negativity', not necessarily positivity.

\section{Key findings and contribution to knowledge}
\cref{chap:protocol} has provided a new protocol for soundscape assessments and reports on the lessons learned through its application. Paired with the new visualisation and analysis method presented in \cref{ch:ProbabilisticPOC}, I hope to bring a more appropriate method of characterising the collective perception of urban spaces. From this, we can improve how the soundscapes of locations are discussed and designed. An important aspect of both of these pieces of work is their open access publication -- large scale soundscape assessments have been performed before, especially for the purpose of training machine learning models (e.g. \citep{Yu2009Modeling}), but without open access publication they are unavailable to future researchers to build upon. At each stage and for each new grant, efforts are restarted and repeated. The goal of the \gls{isd} is to provide a starting point which can be adapted, reanalysed, and to which new data can be added. Throughout the process of writing \cref{ch:ProbabilisticPOC} and developing \texttt{soundscapy}, one of my goals (in addition to the stated goals) was to make attractive and compelling visualisations which can communicate the multidimensional and varying nature of soundscape perception such that it would be useful for practitioners outside of the soundscape field. By making this code open source, my goal is to make it as accessible, flexible, and reproducible as possible and to become a key component of the soundscape designer's toolbox of the future. 

\cref{ch:lockdown} demonstrated the human-level impact of a drastic change in urban transport. As a result of the \gls{covid19} lockdowns, an ideal implementation of noise reduction efforts was achieved through drastic reductions in traffic flows and substantial reductions in transport and delivery activity. Our question was then whether these changes in the common sources of urban noise actually resulted in the desired noise reduction and whether these noise reductions would have achieved an improved in the perceived soundscape of urban spaces. In the first case, how effective these traffic reductions were at reducing sound levels was heavily dependent on the type of space, although a general reduction was seen across the city as a whole. However, the predictive modelling demonstrated that even large reductions in traffic noise levels at sites like Camden Town and Euston Tap were not enough to make those sites truly `pleasant', when considered from a holistic soundscape perspective. In addition, the transport reductions seen under \gls{covid19} resulted in negative impacts to other highly pleasant soundscapes, where the reduced traffic and human sounds resulted in less pleasant soundscapes. The predictive soundscape model was shown to be crucial to answering these questions and to investigating the lockdown impacts beyond the point of tracking to what extent sound levels were reduced.

Beyond the research questions regarding the lockdown changes, the model building procedure highlighted a fundamental difference between the two dimensions of the circumplex. While the contextual information provided by the LocationID were, unsurprisingly, found to be crucial to predicting the soundscape pleasantness, eventfulness was found to be predicted by the psychoacoustic features alone. This suggests a more direct connection between sonic characteristics and the perceived eventfulness, while perceived pleasantness is much more contextual. As far as I know, this difference in the importance of contextual information in pleasantness compared to eventfulness has not been commented on before. Even in soundscape studies, by primarily focussing on the pleasant/annoying dimension, we have missed the unique aspects of eventfulness perception. It also highlights the problems with previous attempts to create context-independent predictive models of pleasantness and annoyance. Since both \cref{ch:lockdown,ch:mlmann} demonstrate that most of the psychoacoustic features included have slopes which vary according to either their location or their sound source, assuming a fixed relationship across all contexts appears incomplete.

The goals, constraints, and proposals established in \cref{ch:bayes} lay out a proper framework for the future development of predictive soundscape models. By requiring that models are based on only measurable quantities, we set up a case where these models can be employed in the engineering use cases discussed throughout. \cref{ch:mlmann} provides a basis, both methodologically and empirically, for incorporating sound source information into predictive soundscape models. I demonstrate that the inclusion of sound source labels contributes significantly to the prediction of annoyance and, along with the DYNAMAP team, we are able to apply this model to data derived from a \gls{wasn}. An interesting finding from this analysis was that sharpness, as the fixed effects feature, is the only one of the included psychoacoustic features to have a relationship with annoyance which is independent of the sound source. This analysis sets the stage for future perception prediction models to be combined with automated source recognition to provide a more detailed model. Based on the work presented in \cref{ch:whostudy}, we were able to determine that personal factors do play a significant role in the formation of one's soundscape perception. While similar studies have attempted to examine psychological wellbeing as an outcome of positive or negative soundscapes \citep{Tarlao2020Investigating,Aletta2018Associations} we took the approach of looking at how one's pre-existing psychological state may influence their perception. This highlights another aspect of humans' bidirectional relationship with their soundscapes: our perception of a soundscape can improve or reduce our psychology wellbeing, but at the same time, our state of mind can enhance or diminish our perception of the soundscape. That said, in the context of developing a practical predictive model, the relatively low amount of variance explained by these factors and the difficulties in obtaining the necessary input data in a design application, indicates that personal factors should likely be considered optional factors.

\section{Limitations and future work}

This thesis represents the development of a novel approach to predictive soundscape models. As such, each chapter represents one facet of the development and therefore the specific limitations of that aspect are discussed in detail throughout. In addition, extensive proposals for future work tailored towards the creation of a future general and probabilistic model are given in each of the chapters. For this reason, this limitation and future work section are focussed on critiques and improvements of the overarching approach taken in the thesis.

\subsection{Limitations of the circumplex and quantitative analysis}

This thesis has relied heavily on the soundscape circumplex and on the methods included in \cite{ISO12913Part2}. It is prudent, then, to consider some of the critiques of the circumplex and of a quantitative approach to soundscape studies more broadly.

There has been some discussion regarding the interpretation of the Likert scales, the interdependence of the PAs, and the strict validity of the 90\textdegree and 45\textdegree relationships between the attributes \citep{Lionello2021Thesis}. In \citet{Lionello2021Introducing}, a study examining these aspects using a subset of the \gls{isd} to which I contributed as a co-author, we analysed the internal correlations of the Likert responses of the 8 perceptual attributes. In particular, the goal was to determine to what degree antipoles on the circumplex accurately reflect opposite understanding of the terms and whether the levels of the Likert scale are equidistant from each other; i.e. is annoying actually the opposite of pleasant, chaotic the opposite of calm, etc. and is each step along these dimensions equally spaced. In his previous work, \citet{Lionello2019dimension} found strong dependencies between the different categories of Likert levels (agree, disagree, neutral). The subsequent analysis found that, in general, the average participant tends to assess a given soundscape as more pleasant than it is not annoying. This indicates that, to some extent, `participants used the scales differently from what would be expected based on the soundscape assessment theoretical framework' and the intervals are not necessarily interpreted to range equidistant spaces between Likert scale categories \citep{Lionello2021Introducing}. Hence there is some question as to how accurately the circumplex projection method reflects the true intended expression of the participant.

Further work has indicated that the scaling between the attributes may vary, but the underlying relationships and their reflection of the emotional affect circumplex from \citet{Russell1980circumplex} hold. It is for this reason that I have taken the coordinate projection as the starting point of the critique of the ISO given in \cref{ch:ProbabilisticPOC}. It should also be noted that the particular PA descriptors used in ISO 12913 are intended for outdoor environments and should not be directly applied to indoor spaces. However, a proposed set of descriptors for some indoor environments has been derived which further confirms the validity of the circumplex relationships \citep{Torresin2020Indoor}. The methods proposed here should be directly applicable to indoor spaces by using the comfort/content descriptors as well as to any other translations of soundscape descriptors into other languages \citep{Aletta2020Soundscape} as long as the dimensional relationships of the circumplex are maintained.

\paragraph*{Qualitative analysis}

The summary and visualisation method presented in \cref{ch:ProbabilisticPOC} is a solution for representing the soundscape of a space, which requires considering the perception of many people, but it is important to note that this is only one goal of the soundscape approach. Psychological and sociological investigations of people's relationship to their sound environment and the interactions between social contexts and individual perception are a crucial aspect of the field for which this approach would likely not be sufficient \citep{Bild2018Public}. Open-response questions, structured interviews, and mixed-methods studies can provide additional insight into how people experience their environment and should be considered alongside or preceding this focus on how a space is likely to be perceived on a larger scale. Several criticisms of the sorts of questionnaire-based approaches highlighted in \citet{ISO12913Part2} and used throughout this thesis have been raised. \citet{Bild2018Public} notes:

\begin{quote}
  [\dots] the questionnaires used as tools to gain insight on users' soundscape evaluations mostly employ categorical-based assessments and rarely include open-ended questions [\ldots] thus representing a limited understanding of users' soundscape evaluations. Finally, these methods minimize or do not adequately account for the role of moderating factors, like activity, in influencing how people evaluate what they hear, despite increasing evidence on activity as a moderating activity for users' soundscapes.
\end{quote}

In contrast, \citet{Bild2018Public} employs a mixed-methods approach which includes both `reported' (i.e. questionnaire-based) and `enacted' soundscape evaluations. Enacted evaluations are assessed by observing how people actually use the space under investigation. These other approaches are not in opposition to the methods proposed here, but instead further expand our view. The circumplex is a limited view of soundscape perception (this is made obvious by the fact that it excludes the third component, \emph{familiarity}, identified in \citet{Axelsson2010principal}) but it is an exceptionally rich tool for dealing with the two primary aspects of soundscape perception which can readily expand the much more limited view provided by existing noise and annoyance assessment tools. Aspects of the psychological and sociological emphasis can also be integrated into a circumplex-focused approach, as demonstrated in \cref{ch:whostudy}, where personal factors such as age, gender, and psychological well-being were analysed in terms of how they mediated the ISOPleasant and ISOEventful outcomes.

\subsection{Long-term soundscape analysis and mapping}
One issue with retrospective assessment methods in \cref{sec:NeedForPredModels} is their struggle to capture the dynamics of a soundscape, how it may change throughout the day, week, or season. A second aspect of this dynamic is related to how people's own perceptual mapping may fluctuate throughout these periods. The predictive modelling framework presented here would not be able to address this as, once the data is collected and the model trained, it has essentially learned the perceptual mapping for period and scenario in which it was collected. One approach to answering this question which, as far as I am aware, has not been employed is through an un-attended survey method. Such a method could involve creating and posting fliers asking users of a space to complete a soundscape survey (accessed through a QR code) and leaving these fliers installed for longer periods of time. It is unclear how successful such a general approach would be, in particular what response rate would be expected, but given the increasing familiarity with QR codes among the general public following their use for track-and-trace during COVID-19, it does appear promising. These un-attended surveys could also be paired with long-term acoustic and environmental monitoring via a WASN or powered SLM which could simultaneously track the acoustic environment. This would thus result in a time series of online soundscape questionnaires with a corresponding time series of acoustic and environmental information, allowing us to track the changes of each over long periods of time.

\subsection{Personal factors}
One aspect of the examination of personal factors that was not considered in \cref{ch:whostudy} was the degree to which the expected variance of responses across groups may be influenced by these secondary factors. If, as proposed in \cref{sec:probPred}, future models are also used to predict the standard deviation of the expected response to a given input, then we should consider how different groups may have more or less agreement in terms of their perception. We could imagine that between two age groups, say above and below 65, when exposed to the same soundscape, one group may have more consensus on the soundscape perception while the other has more variation. In other words, the standard deviation of responses is correlated with personal factors. This would then result in a different \emph{soundscape shape} in the circumplex, which may be important for a designer to consider. This is a distinct concept from the questions in \cref{ch:whostudy}. There, we examined how the average pleasantness and eventfulness scores were mediated by these personal factors. In other words, how the \emph{location of the soundscape} in the circumplex would be changed. At this stage, our results cannot indicate to what extent the variation in responses to the same stimulus may be mediated by these personal factors. This would make an interesting avenue of future experimental studies.

\section{Concluding remarks}

Previous strategies toward practical urban soundscape design have been ground-breaking in their approach, bringing soundscape measurements and calculations into the design process \citep{SchulteFortkamp2008Using,SchulteFortkamp2016When} and creating interactive soundscape interventions \citep{Steele2021Bringing} or drawing on philosophies of sound art to reshape a community's relationship with their soundscape \citep{Lacey2016,Lacey2019Noise}. However, these have been limited in their scope, providing methods of improving individual soundscapes or approaches which can be applied to bespoke projects but are not scalable. This work aims to move towards a generalised and widely applicable engineering-based approach. The goal is to promote a soundscape mindset as the `standard', not just as an extra add-on for forward-thinking projects or as a localised sonic rupture which, while incredibly effective (and affective) within its radius, is not suited to being applied on a city- or national-policy scale. For this purpose, we require a standardised and implementable index and direction of best practice which can be implemented by trained technicians, engineers, designers, and planners across all aspects of urban design, from the billion dollar museum to the inner-city public elementary school. A desire for good and restorative soundscapes should be the baseline standard in a city's design. The goal of this work therefore, is not to critique or counter the creative approaches taken by those within sound art or acoustic ecology, but instead to move towards a new baseline, a new way of designing all environments of the city.

The soundscape approach offers the opportunity and frameworks to speak about soundscapes in terms of how they more directly impact people, how people experience and react to the environment. While predictive models will never fully capture this nuance and should form only one tool in the toolbox of user-centred design, the perceptually-derived metrics provide a more comprehensive insight into urban sound impacts. If suitable predictive models and subsequent soundscape indices are developed, they will provide a crucial intermediary through which to investigate the connection between urban environments and health and wellbeing. 

While the final goal of a general soundscape model which can be used for these aims has yet to be realised, in this thesis I hope I have provided four key steps towards it: the protocol and initial data collection to create a large and lasting database of soundscape assessments; an initial demonstration of how predictive soundscape models can enable new modes of research not addressed by existing soundscape methods; a framework for future developments of predictive models and a novel way of thinking about the collective soundscape of urban spaces; and strong empirical evidence for the importance of considering sound sources and demographic features in soundscape models.