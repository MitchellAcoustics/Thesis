\documentclass[twoside,fontsize=12pt,titlepage]{scrbook}

% * Packages
\usepackage[utf8]{inputenc}
% \usepackage{cite}
\usepackage{graphicx}
\usepackage{eso-pic}
\usepackage[hidelinks]{hyperref}
\usepackage{natbib}
\bibliographystyle{apalike}
\usepackage{color, soul}
\usepackage{xcolor}
\usepackage{pdfpages}

\usepackage{gensymb}
\usepackage{makecell}
\usepackage{booktabs}
% \usepackage{tabularx}

\newcommand{\remark}[3]{%
    {\colorbox{#2}{\sffamily\scriptsize\bfseries\textcolor{white}{#1}}}
    {\sffamily\small\itshape\textcolor{#2}{#3}}
}
\newcommand{\misc}[1]{\remark{misc}{green}{#1}}
\newcommand{\draft}[1]{\remark{draft}{blue}{#1}}
\newcommand{\error}[1]{\remark{err}{red}{#1}}
\newcommand{\proof}[1]{\remark{proof}{orange}{#1}}
\newcommand{\qt}[1]{\remark{quote}{brown}{#1}}

% * Formatting
% Margins at the binding edge must not be less than 40mm (1.5 inches) and other margins not less than 20mm (0.75 inches).
% Double or one-and-a-half spacing should be used in typescripts except for indented quotations or footnotes where single spacing may be used.
\usepackage[a4paper,margin=20mm, bindingoffset=20mm]{geometry}
\usepackage[onehalfspacing]{setspace}
\AddToShipoutPicture*
{%
\put(0,730)% the specific point of the page with coordinates (x=100,y=100)
{\includegraphics[scale=1]{Figures/UCL header.png}}%
}
\usepackage{ebgaramond}

%%%%%%%%%%%%%%%%%%%%%%%%%%%%%%%%%%%%%%%%%%%%%%%%%%%%%%%%%%%%%%%%%%%%%%%%%%%%%%%%%%%%%%%%%%%%%%%%%%

\begin{document}

\newgeometry{margin=20mm, bindingoffset=0mm}
\begin{titlepage}
      % The title page must bear the following:
      % - the officially-approved title of the thesis
      % - the candidates full name as registered
      % - the institution name 'UCL'
      % - the degree for which the thesis is submitted
      \AddToShipoutPicture*{}
      \begin{center}
            \vspace*{3cm}

            \Huge
            \textbf{Machine Learning and Regression Modelling of Complex Urban Soundscapes}

            % \vspace{0.5cm}
            \LARGE
            % Thesis Subtitle

            \vspace{1.5cm}

            \textbf{Andrew James Mitchell}

            \vfill
            A thesis presented for the degree of\\
            Doctor of Philosophy\\
            \rule[-.5cm]{0.5\textwidth}{1pt}

            \vspace{1.5cm}

            \Large
            Institute for Environmental Design \& Engineering\\
            University College London (UCL)\\
            \today

            \vspace{1cm}

            Principal Supervisor: Prof. Jian Kang\\
            Co-Supervisor: Dr. Phil Symonds

      \end{center}
\end{titlepage}


\restoregeometry

%%%%%%%%%%%%%%%%%%%%%%%%%%%%%%%%%%%%%%%%%%%%%%%%%%%%%%%%%%%%%%%%%%%%%%%%%%%%%%%%%%%%%%%%%%%%%%%%%%

% \chapter*{Declaration}
% % The title page should be followed by a signed declaration that the work presented in the thesis is the candidate's own:
% I, Andrew Mitchell, confirm that the work presented in this thesis is my own. Where information has been derived from other sources, I confirm that this has been indicated in the thesis.

\chapter*{Abstract}
% The signed declaration should be followed by an abstract consisting of no more than 300 words


% \chapter*{Acknowledgements}
% This research was funded by the European Research Council.

% People to thank:
% \begin{itemize}
%     \item Dr Francesco Aletta
%     \item Prof Jian Kang
%     \item Dr Tin Oberman
%     \item Dr Phil Symonds
%     \item Ms Mercede Erfanian
%     \item Ms Magdalena Kachlicka
%     \item Mr Matteo Lionello
%     \item Friends: Nicole Watson, Nick Wilson, Valentina
% \end{itemize}

\chapter*{List of Studies}

This doctoral thesis is based on the following studies:

\paragraph*{Commentary: }
Kang, J., Aletta, F., Oberman, T., \textbf{Mitchell, A}., Erfanian, M., Tong, H. (2021). Supportive Soundscapes are Crucial for Sustainable Cities and Communities. \emph{Nature Sustainability}.

\paragraph*{Protocol:  }
\textbf{Mitchell, A.}, Oberman, T., Aletta, F., Erfanian, M., Kachlicka, M., Lionello, M., \& Kang, J. (2020) The Soundscape Indices (SSID) Protocol: A Method for Urban Soundscape Surveys -- Questionnaires with Acoustical and Contextual Information. \emph{Applied Sciences, 10} (7), 2397. https://doi.org/10.3390/app10072397

% \paragraph*{Study II:   }
% Erfanian, M., Mitchell, A. J., Kang, J., \& Aletta, F. (2019). The Psychophysiological Implications of Soundscape: A Systematic Review of Empirical Literature and a Research Agenda. \emph{International Journal of Environmental Research and Public Health, 16(19)}, 3533. https://doi.org/10.3390/ijerph16193533

\paragraph*{Study I:    }
Erfanian, M., \textbf{Mitchell, A.}, Aletta, F., \& Kang, J. (2020). Psychological Well-being, Age and Gender can Mediate Soundscapes Pleasantness and Eventfulness: A large sample study. \emph{Environmental Psychology}.

\paragraph*{Study II:   }
Orga, F., \textbf{Mitchell, A.}, Freixes, M., Aletta, F., Alsina-Pagès, R. M., \& Foraster, M. (2021). Multilevel Annoyance Modelling of Short Environmental Sound Recordings. \emph{Sustainability, 13}(11), Article 11. https://doi.org/10.3390/su13115779

\paragraph*{Study III:  }
\textbf{Mitchell, A.}, Oberman, T., Kachlicka, M., Aletta, F., Lionello, M., Erfanian, M., \& Kang, J. (2021). Applied Predictive Soundscape Modelling: A Case Study Investigating Changes from the COVID-19 Lockdown. \emph{JASA}.

\paragraph*{Study IV:   }
\textbf{Mitchell, A.}, Soelitsyo, C., Erfanian, M., Xue, J-H., Oberman, T., Kang, J., \& Aletta, F. (2021). A Temporal Convolutional Neural Network for Multi-label Environmental Sound Recognition and Annoyance Detection. \emph{IEEE}.

\paragraph*{Commentary: }
\textbf{Mitchell, A.}, Aletta, F., Chalabi, Z., \& Kang, J. (2021). From Deterministic to Probabilistic Soundscapes: A critical tour around the soundscape circumplex. \emph{JASA-EL}.

\newpage
The following studies are related works which influenced this thesis and were completed as part of the same work but have not been included as key components:

\paragraph*{}Lionello, M., Aletta, F., \textbf{Mitchell, A.}, \& Kang, J. (2020). Introducing a Method for Intervals Correction on Multiple Likert Scales: A Case Study on an Urban Soundscape Data Collection Instrument. \emph{frontiers in Psychology}.

\paragraph*{}Aletta, F., Oberman, T., \textbf{Mitchell, A.}, Tong, H., \& Kang, J. (2020). Assessing the changing urban sound environment during the COVID-19 lockdown period using short-term acoustic measurements. \emph{Noise Mapping}.

\paragraph*{}Tong, H., Aletta, F., \textbf{Mitchell, A.}, Oberman, T., \& Kang, J. (2021). Increases in noise complaints during the COVID-19 lockdown in Spring 2020: A case study in Greater London, UK. \emph{Science of the Total Environment}.



\chapter*{Impact Statement}
% The abstract should be followed by an impact statement consisting of no more than 500 words.
The statement should describe, in no more than 500 words, how the expertise, knowledge, analysis, discovery or insight presented in your thesis could be put to a beneficial use. Consider benefits from \textbf{inside} and \textbf{outside} academia and the ways in which these benefits could be brought about.

\chapter*{COVID-19 Statement}

In March of 2020, 18 months into the development of this thesis, the COVID-19 pandemic hit the UK, forcing it into lockdowns which would continue for over a year. Solely by good fortune and a tendency to speed ahead with too-little thought, the primary data collection had fortunately been completed prior to the first lockdown. However, this work was impacted in three ways:

\begin{enumerate}
      \item Further in-situ data collection could not be completed, reducing the range of soundscape types we could include;
      \item The unprecedented and stressful world of the pandemic had a significant mental health and social impact, the effects of which cannot be quantified, nor overstated;
      \item In response to the unique scientific opportunity of a world-wide transportation and social lockdown, new, unplanned studies were carried out.
\end{enumerate}

In particular, this final point has had an impact on the structure and content of this thesis. Certain aspects of the research, in particular the model development and building, were accelerated and put into practice to investigate the impacts of the COVID lockdowns, before being returned to and further developed. The initial research plan would have followed a more logical path of nailing down the model development first, then moving on to a first implementation. In addition, new work was added to this thesis which may appear incongruous or unrelated, but represents a great deal of necessary work which further informed the key strains of the thesis.



%%%%%%%%%%%%%%%%%%%%%%%%%%%%%%%%%%%%%%%%%%%%%%%%%%%%%%%%%%%%%%%%%%%%%%%%%%%%%%%%%%%%%%%%%%%%%%%%%%

\tableofcontents
% In each copy of the thesis the abstract should be followed by a full table of contents (including any material not bound in) and a list of tables, photographs and any other materials.
% \listoffigures
% \listoftables

%%%%%%%%%%%%%%%%%%%%%%%%%%%%%%%%%%%%%%%%%%%%%%%%%%%%%%%%%%%%%%%%%%%%%%%%%%%%%%%%%%%%%%%%%%%%%%%%%%

% Thesis by publication add-ons

\chapter{Introduction}

\section{Impacts of Urban Noise on Health and Wellbeing}

\section{Current Methods of Assessing and Addressing Urban Noise}
 The approach to a practical predictive soundscape model arrived at within this thesis is heavily based on past environmental acoustics approaches, I will therefore begin with a brief summary of these past approaches.

\section{Soundscape - Theory and Application}
 \subsection{Soundscape Descriptors and Indices}
       \citep{Aletta2016Soundscape}

 \subsection{The ISO 12913 Standard Series}
\section{Environmental Acoustics and Psychoacoustics Analyses}


 \chapter{An Open International Urban Soundscape Database}


 %%%%%%%%%%%%%%%%%%%%%%%%%%%%%%%%%%%%%%%%%%%%%%%%%%%%%%%%%%%%%%%%%%%%%%%%%%%%%%%%%%%%%%%%%%%%%%%%%%%%

 \chapter{Studies \& Summaries}

 The following studies form the core of this thesis, with each contributing a key component of the soundscape modelling process. They are presented in a logical, rather than chronological order and are each preceded by a brief summary.

 \newpage
\section[Commentary]{Commentary: Supportive Soundscapes are Crucial for Sustainable Cities and Communities}

 Presented as a \textit{Comment} paper in Nature Sustainability, this paper provides an opininated view towards the placement of soundscape in future sustainability research and development. Heavily shaped and primarily drafted by me, it puts forth the argument that cities and communities cannot be sustainably designed without a consideration of 1) how noise impacts the community, and 2) how the community impacts the existing soundscape. \misc{add a bit about the agenda presented}

 Although published towards the end of my PhD, it is placed at the beginning as it offers our strongest argument for the necessity of soundscape in future sustainable design. It situates noise as a key environmental concern and provides a justification for a soundscape approach, which requires and incorporates predictive tools to be applied in engineering, research, and design contexts.

 \newpage
\section[Protocol]{Protocol: The Soundscape Indices (SSID) Protocol: A Method for Urban Soundscape Surveys -- Questionnaires with Acoustical and Contextual Information}

 Conducting urban soundscape studies on a scale large enough to form a machine learning dataset presents a unique challenge. The standardised methods of conducting soundscape surveys \citep{ISO12913_2_2018IOS} are labour-intensive, time-consuming, and provide limited information about the acoustical and environmental context.

 \includepdf[pages={1-16}]{./Papers/Mitchell2020Soundscape.pdf}

\section[Study I]{Study I: Psychological Well-being, Age, and Gender can Mediate Soundscape Pleasantness and Eventfulness: A large sample study}

 \includepdf[pages={1-32}]{./Papers/PsychoPreprint.pdf}

\section[Study II]{Study II: Multilevel Annoyance Modelling of Short Environmental Sound Recordings}

 \includepdf[pages=-]{./Papers/Orga2021Multilevel.pdf}

\section[Study III]{Study III: Applied Predictive Soundscape Modelling: A Case Study Investigating Changes from the COVID-19 Lockdown}

 \newpage
\section[Study IV]{Study IV: A Temporal Convolutional Neural Network for Multi-label Sound Recognition and Annoyance Detection of Complex Soundscapes}

 \includepdf[pages=-]{./Papers/IEEE_DeLTA_Preprint.pdf}

\section[Commentary]{Commentary: From Deterministic to Probabilistic Soundscapes: A critical tour around the soundscape circumplex}

 \includepdf[pages=-]{./Papers/J13_Circumplex_transform_Preprint.pdf}


 %%%%%%%%%%%%%%%%%%%%%%%%%%%%%%%%%%%%%%%%%%%%%%%%%%%%%%%%%%%%%%%%%%%%%%%%%%%%%%%%%%%%%%%%%%%%%%%%%%

 % \chapter{Introduction}
\label{ch:intro}

\section{Research Summary}
Urban noise pollution affects 80 million EU citizens with substantial impacts on public health which are not well addressed by conventional noise control methods. Traditional noise control methods have typically limited their focus to the reduction of unwanted noise, ignoring the potential benefits of increasing positive sounds and remaining restricted by practical limitations of noise reduction. Modern approaches to achieve improved health outcomes and public satisfaction aim to incorporate a person's perception of an acoustic environment, an approach known as 'Soundscape'.

Soundscape studies strive to understand the perception of a sound environment, in context, including acoustic, (non-acoustic) environmental, contextual, and personal factors. These factors combine together to form a person's soundscape in complex interacting ways \citep{Berglund2006Tool}. In order to predict how people would perceive an acoustic environment, it is essential to identify the underlying acoustic and non-acoustic properties of soundscape.

% From: Upgrade report
When attempting to apply soundscape in practical applications in the built environment, it is immediately apparent that a predictive model of the users' perceptual response to the acoustic environment is necessary. Whether to determine the impact of a design change, or to integrate a large scale data at neighbourhood and city levels, a mathematical model of the interacting factors will form a vital component of the implementation of the soundscape approach. This work is intended to identify methods for incorporating contextual and objective information into a useable and interpretable predictive model of urban soundscapes. In order to achieve this, a protocol for collecting the multi-level, multi-factor perceptual assessment data has been developed and implemented, resulting in a large soundscape database. Several avenues of investigation are then drawn from the database and addressed throughout this thesis. The primary research questions are:

% TODO: Try to move this to the end of the introduction.
\begin{enumerate}
  \item What are the primary acoustic features involved in soundscape formation and what are the driving interactions between acoustic features and soundscape assessment?
  \item How does the sound source composition in a complex sound environment mediate this interaction and how can this effect be simplified and modelled?
  \item How can the multiple levels of soundscape formation be simplified and integrated into a cohesive predictive model, and what interpretations about the cross-effects of these levels can be drawn from the model?
  \item In what ways and to what extent can predictive soundscape modelling be applied to address future urban design challenges? How can these methods best be integrated into policy, design, noise mapping, and engineering practice?
\end{enumerate}

Towards answering these questions, the results of five %TODO: check this number at the end.
peer-reviewed studies are presented. These studies represent a series of work to 

\begin{enumerate}
  \item Advance the conceptual development and practice of soundscape studies
  \item Develop a transparent and useful method of predicting soundscape assessments
  \item Investigate the various components which influence soundscape perception, including personal factors like psychological well-being, acoustical factors, and sound source specifics and to integrate these components into the predictive modelling methods.
\end{enumerate}

\section{The SSID Project}
The \gls{ssid} Project is a five-year, multi-disciplinary project funded by a Horizon 2020 European Research Council grant (no. 740696).

\subsection{Project collaborators}

\subsection{Motivation for the SSID Project}

\section{Research Aims}

\section{Soundscape Indices and Metrics}

\section{General Aim}

 % \chapter{Literature Review}
\label{ch:lit}

\section{Impact of Urban Noise on Health and Wellbeing}

\emph{Give a full formal background to why noise control is important for public health}.

\section{Current methods of Assessing and Addressing Urban Noise}

\subsection{Acoustical Parameters}

\subsection{EU Noise Mapping}

\subsection{Shortcomings}

\section{Soundscape Studies}

\subsection{World Soundscape Project}

\subsection{Swedish Soundscape Quality Protocol}

\section{Existing Predictive Models}

\cite{Lionello2020}


 % \chapter{Methods}
\label{chap:methods}

%%%%%%%%%%%%%%%%%%%%%%%%%%%%%
%FIXME: Find a better spot for this
The ability to predict the likely soundscape assessment of a space is crucial to implementing the soundscape concept in practical design. Current methods of assessing soundscapes are generally limited to a post-hoc assessment of the existing environment, where users of the space in question are surveyed regarding their experience of the acoustic environment \citep{Engel2018Review, Zhang2018Effect}. While this approach has proved useful in identifying the impacts of an existing environment, designers require the ability to predict how a change or proposed design will impact the soundscape of the space. To this end, a model that is built upon measurable or estimate-able quantities of the environment would represent a leap forward in the ability to design soundscapes.

\section{Questionnaires}

 The full protocol developed for this thesis is outlined in Chapter \ref{chap:protocol}. The development and presentation of this protocol involved a substantial development and testing phase, and represents a novel advancement in soundscape survey methodology. Therefore it was submitted and published as a peer-reviewed journal article in MDPI Applied Sciences as \citet{Mitchell2020Soundscape} and is presented as a stand-alone chapter within this thesis.

 \subsection{Likert Responses}

 \subsection{Circumplex Projection}

\section{Psychoacoustics and Auditory Perception}

 \subsection{Psychoacoustic Parameters}

   \subsubsection{Loudness}
   \emph{Zwicker and Fastl, Chap 8, see Mendeley notes and python-acoustics development notes.}
 \subsection{Feature Selection}

\section{Machine Learning and Regression Techniques}

 \subsection{Feature Selection}
   \subsubsection{Mutual Information}
   \draft{It appears that mutual information is related to the Bayes formula. I still need to read more into this, but it appears based on relative and overlapping probability distributions between the variables in question.}
   \paragraph*{From scholarpedia:}
   % http://www.scholarpedia.org/article/Mutual_information
   \draft{Based on entropy, where the uncertainty about a variable can be expressed as "the number of yes/no questions it takes to guess a random variable, given knowledge of the underlying distribution and taking the optimal question-asking strategy". "The mutual information is therefore the \emph{reduction} in uncertainty about variable $X$, or the expected reduction in the number of yes/no questions needed to guess $X$ after observing $Y$.". }

   \draft{"Mutual Information is just one way among many of measuring how related two variables are. However, it is a measure ideally suited for analyzing communication channels. Abstractly, a communication channel can be visualized as a transmission medium which receives an input $x$ and produces an output $y$. If the channel is \emph{noiseless}, the output will be equal to the input. However, in general, the transmission medium is noisy and an input $x$ is converted to an output $y$ with probability $P_{Y|X}(y|x)$. }
   \misc{This seems very useful for my conception of sound perception / auditory processing, where the perception system is a noisy communication channel.}

   \subsubsection{Conditional Mutual Information}
   The Mutual Information between two variables, given another variable as a control.

 \subsection{Clustering Analysis}
   \paragraph{K-means}
   \paragraph{nbclust}

 \subsection{Modelling Likert-type Data}

   \subsubsection{Multiple Linear Regression}

   \subsubsection{Ordinal Logistic Regression}

   \subsubsection{Multi-output Regression}

 \subsection{Multi-level Models}

 \subsection{Bayesian Regression}


 % \chapter{Characterizing the Temporal Behaviour of Dynamic Urban Soundscapes}
\label{ch:temp}

\section{Introduction}

\section{Methods}

\section{Results and Discussion}
  \subsection{Presence of 1/f in urban soundscapes}
  \subsection{Statistical relationship to pleasantness ratings}
  \subsection{Ordinal logistic models based on temporal and acoustic features}

\section{Conclusion}

 % \chapter{Combined Multi-level Regression Model for Predicting Soundscape}
\label{ch:mlm}

\section{Introduction}

\section{Methods}
  \subsection{Multi-level regression modelling}
  \subsection{Feature Selection}

\section{Results}
  \subsection{Simplified predictive soundscape models}
    \paragraph*{COVID-19 Model}

  \subsection{Multiple levels of soundscape formation}

  \subsection{Feature selection}
    \paragraph*{Acoustic features}
    \paragraph*{Non-acoustic factors}

  \subsection{Model design}

\section{Discussion}
  \subsection{Interpretation}
  \subsection{Implementation and use cases}



 % \chapter{The Influence of Sound Source Composition in Soundscape Formation}
\label{ch:ssp}

\section{Introduction}

\section{Methods}
  \subsection{Data collection}
  \subsection{Clustering analysis}

\section{Results}
  \subsection{Sound source profiles}
  \subsection{Perceived affective quality ratings}
  \subsection{Psychological well-being mediates soundscape formation within different sound source profiles}
  \subsection{Regression models}

\section{Discussion}

\section{Conclusion}


 % \chapter{A Proposal for a Future Predictive Model -- A Bayesian Hierarchical Approach}
  \label{ch:bayes}
% alt:A Bayesian Hierarchical Predictive Soundscape Model and a Proposed Soundscape Index
% alt: Probabilistic soundscape models including personal, contextual, environmental, and acoustic information

\section{Introduction}
\subsection{The problem with the pleasant-annoying paradigm}

As made clear by their name, noise annoyance studies have focussed on the relationship between noise (or acoustic) features and the perceived annoyance, to varying degrees of success. Within the soundscape circumplex framework, annoyance is the negative side of the pleasantness dimension, forming the \nth{1} primary component of soundscape perception. This means that, along with pleasantness, annoyance is that perceptual attribute which is most readily perceived and plays the largest part in differentiating between the perception of different soundscapes. This fact therefore makes this pleasantness dimension the prime target for addressing noise issues. 

However, \citet{Mitchell2021Investigating} (i.e. \cref{ch:lockdown}) and \citet{Aumond2022} have both recently demonstrated a fundamental difference in the statistical relationships connecting context and acoustic features with the perceived pleasantness and eventfulness of urban soundscapes. %TODO: Continue this discussion about including eventfulness. Can draw from that review I wrote

\section{Starting point} 
\draft{The lockdown model, as the most developed model so far.}

\section{Incorporating personal factors into a predictive soundscape model}
Although, as \citet{Droumeva2021sound} points out, each individual brings their own cultural and subjective aspects of listening to the stage of urban sound, when attempting to characterise the soundscape of a space, it is not a particular individual's aspects we should be concerned with. That individual forms a part of the collective perception of the space. Their cultural and subjective (i.e. personal) aspects mitigate their perception, but this perception then forms only a single component of the collective perception. How then should we consider these personal factors? Surely there is no suggestion to disregard their influence and importance within the soundscape approach? In my view, there are two approaches:

\begin{enumerate}
  \item Incorporate these personal factors as demographic statistics of a location; or
  \item An agent-based approach where each individual likely to use the space is simulated and modelled with their personal factors to then be included in the collective perception.
\end{enumerate}

Let's look at how these two approaches would be implemented in a multilevel acoustics-based predictive model, such as those presented in \crefrange{ch:mlmann}{ch:lockdown}.

%TODO: Expand
\subsection{Approach 1}
In the first, the demographic breakdown of the space under investigation is estimated, either through a census or by the designers' desired use case. This demographic breakdown can then be compared to the results presented above \citep{Erfanian2021Psychological} to derive weighting factors which adjust the predicted soundscape assessment. For instance, the results suggest that retired persons perceive the soundscape as \draft{XX\% or amount [need to check with results]} less pleasant than others. If the particular space under investigation has a large proportion of retired persons, say 65\% we could then apply an adjustment to the initial personal-factors-agnostic prediction to reflect this tendency. In this example, an initial location-level \gls{isopl} prediction of 0.36, with a 65\% retired population would be corrected by \draft{-XX [0.65 x result]} for a final demographics-corrected \gls{isopl} prediction of \draft{XX}.

\subsection{Approach 2}

\subsection{Benefits and downsides of each approach}

\section{Incorporating sound source information}

\section{Probabilistic predictions - A Bayesian Approach?}

\section{Discussion}

\section{Conclusion}


 % \chapter{Soundscape Modelling for Smart Cities: A case study}
\label{ch:smart}

\section{Introduction}

\section{Methods}
  \subsection{SSID Data Collection}
  \subsection{Sensor network and IFSTTAR data collection}

\section{Results}

\section{Discussion}



 \chapter{Conclusions}
\label{ch:conc}

\section{General Discussion}

\section{Implications}

\section{Limitations and Recommendations for Future Research}

\section{Concluding Remarks}

 %%%%%%%%%%%%%%%%%%%%%%%%%%%%%%%%%%%%%%%%%%%%%%%%%%%%%%%%%%%%%%%%%%%%%%%%%%%%%%%%%%%%%%%%%%%%%%%%%%


 \backmatter
 % A glossary and list of acronyms may go here
 % or may go in the front matter after the abstract.

 % The bibliography will go here.
 \bibliography{thesis_refs}

 %%%%%%%%%%%%%%%%%%%%%%%%%%%%%%%%%%%%%%%%%%%%%%%%%%%%%%%%%%%%%%%%%%%%%%%%%%%%%%%%%%%%%%%%%%%%%%%%%%

\end{document}
